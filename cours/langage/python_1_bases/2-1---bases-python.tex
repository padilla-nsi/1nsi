\documentclass[a4paper,17pt]{extarticle}


    \usepackage[sfdefault, condensed]{roboto} % police d'écriture plus moderne
\usepackage[french]{babel} % francisation
\usepackage[parfill]{parskip} %suppression indentation

\usepackage{fancyhdr}
\usepackage{multicol}

% figure non flotantes
\usepackage{float}
\let\origfigure\figure
\let\endorigfigure\endfigure
\renewenvironment{figure}[1][2] {
    \expandafter\origfigure\expandafter[H]
} {
    \endorigfigure
}

% mois/année
\usepackage{datetime}
\newdateformat{monthyeardate}{%
  \monthname[\THEMONTH] \THEYEAR}

% couleurs perso
\usepackage[table]{xcolor}
\definecolor{deepblue}{rgb}{0.3,0.3,0.8}
\definecolor{darkblue}{rgb}{0,0,0.3}
\definecolor{deepred}{rgb}{0.6,0,0}
\definecolor{iremred}{RGB}{204,35,50}
\definecolor{deepgreen}{rgb}{0,0.6,0}
\definecolor{backcolor}{rgb}{0.98,0.95,0.95}
\definecolor{grisClair}{rgb}{0.95,0.95,0.95}
\definecolor{orangeamu}{RGB}{250,178,11}
\definecolor{noiramu}{RGB}{35,31,32}
\definecolor{bleuamu}{RGB}{20,118,198}
\definecolor{bleuamudark}{RGB}{15,90,150}
\definecolor{cyanamu}{RGB}{77,198,244}


\usepackage{/home/bouscadilla/Documents/Code/nbconvert/template/latex/pdf_solution/xeboiboites}
%
% exemple
\newbreakabletheorem[
    small box style={fill=deepblue!90,draw=deepblue!15, rounded corners,line width=1pt},%
    big box style={fill=deepblue!5,draw=deepblue!15,thick,rounded corners,line width=1pt},%
    headfont={\color{white}\bfseries}
        ]{exemple}{Exemple}{}%{counterCo}
%
% remarque
\newbreakabletheorem[
    small box style={draw=ansi-green-intense!100,line width=2pt,fill=ansi-green-intense!0,rounded corners,decoration=penciline, decorate},%
	big box style={color=ansi-green-intense!90,fill=ansi-green-intense!10,thick,decoration={penciline},decorate},
    broken edges={draw=ansi-green-intense!90,thick,fill=orange!20!black!5, decoration={random steps, segment length=.5cm,amplitude=1.3mm},decorate},%
    other edges={decoration=penciline,decorate,thick},%
    headfont={\color{ansi-green-intense}\large\scshape\bfseries}
    ]{remarque}{Remarque}{}%{counterCa}
%
% formule (sans titre)
\newboxedequation[%
    big box style={fill=cyanamu!10,draw=cyanamu!100,thick,decoration=penciline,decorate}]%
    {form}
%
% Réponse
\newbreakabletheorem[
    small box style={fill=bleuamu!100, draw=bleuamu!60, line width=1pt,rounded corners,decorate},%
    big box style={fill=bleuamu!10,draw=bleuamu!30,thick,rounded corners,decorate},
    headfont={\color{white}\large\scshape\bfseries}
        ]{reponse}{Correction}{}
%

%
% À retenir
%\newbreakabletheorem[
%    small box style={fill=deepred!100, draw=deepred!80, line width=1pt,rounded corners,decorate},%
%    big box style={fill=deepred!10,draw=deepred!50,thick,rounded corners,decorate},
%    headfont={\color{white}\large\scshape\bfseries}
%        ]{retenir}{À retenir}{}
%
\newboxedequation[%
    big box style={fill=deepred!10,draw=deepred!0,thick,decoration=penciline,decorate}]%
    {retenir}



% astuce
\newspanning[
    image=/home/bouscadilla/Documents/Code/nbconvert/template/latex/pdf_solution/fig-idee,headfont=\bfseries,
    spanning style={very thick,decoration=penciline,decorate}
    ]{astuce}{Astuce}{}
%
% activité
\newbreakabletheorem[
    small box style={draw=orangeamu!100,line width=2pt,fill=orangeamu!100,rounded corners,decoration=penciline, decorate},%
	big box style={color=orangeamu!100,fill=orangeamu!5,thick,decoration={penciline},decorate},
    broken edges={draw=orangeamu!100,thick,fill=orangeamu!100, decoration={random steps, segment length=.5cm,amplitude=1.3mm},decorate},%
    other edges={decoration=penciline,decorate,thick},%
    headfont={\color{white}\large\scshape\bfseries}
    ]{activite}{\adjustimage{height=1cm, valign=m}{/home/bouscadilla/Documents/Code/nbconvert/template/latex/pdf_solution/papier_eleve_investigation.png}%
    Activité}{}%{counterCa}
%   
%   environnement élève
%
\newenvironment{eleve}%
%{\begin{activite}\large\\} % écrire plus gros
{\begin{activite}\color{noiramu}\\[-0.5cm]}
{\end{activite}}

\newenvironment{formule}%
%{\begin{activite}\large\\} % écrire plus gros
{\begin{form}\color{bleuamu}}
{\end{form}}


\usepackage[breakable]{tcolorbox}
    \usepackage{parskip} % Stop auto-indenting (to mimic markdown behaviour)
    
    \usepackage{iftex}
    \ifPDFTeX
    	\usepackage[T1]{fontenc}
    	\usepackage{mathpazo}
    \else
    	\usepackage{fontspec}
    \fi

    % Basic figure setup, for now with no caption control since it's done
    % automatically by Pandoc (which extracts ![](path) syntax from Markdown).
    \usepackage{graphicx}
    % Maintain compatibility with old templates. Remove in nbconvert 6.0
    \let\Oldincludegraphics\includegraphics
    % Ensure that by default, figures have no caption (until we provide a
    % proper Figure object with a Caption API and a way to capture that
    % in the conversion process - todo).
    \usepackage{caption}
    \DeclareCaptionFormat{nocaption}{}
    \captionsetup{format=nocaption,aboveskip=0pt,belowskip=0pt}

    \usepackage[Export]{adjustbox} % Used to constrain images to a maximum size
    \adjustboxset{max size={0.9\linewidth}{0.9\paperheight}}
    \usepackage{float}
    \floatplacement{figure}{H} % forces figures to be placed at the correct location
    \usepackage{xcolor} % Allow colors to be defined
    \usepackage{enumerate} % Needed for markdown enumerations to work
    \usepackage{geometry} % Used to adjust the document margins
    \usepackage{amsmath} % Equations
    \usepackage{amssymb} % Equations
    \usepackage{textcomp} % defines textquotesingle
    % Hack from http://tex.stackexchange.com/a/47451/13684:
    \AtBeginDocument{%
        \def\PYZsq{\textquotesingle}% Upright quotes in Pygmentized code
    }
    \usepackage{upquote} % Upright quotes for verbatim code
    \usepackage{eurosym} % defines \euro
    \usepackage[mathletters]{ucs} % Extended unicode (utf-8) support
    \usepackage{fancyvrb} % verbatim replacement that allows latex

    % The hyperref package gives us a pdf with properly built
    % internal navigation ('pdf bookmarks' for the table of contents,
    % internal cross-reference links, web links for URLs, etc.)
    \usepackage{hyperref}
    % The default LaTeX title has an obnoxious amount of whitespace. By default,
    % titling removes some of it. It also provides customization options.
    \usepackage{titling}
    \usepackage{longtable} % longtable support required by pandoc >1.10
    \usepackage{booktabs}  % table support for pandoc > 1.12.2
    \usepackage[inline]{enumitem} % IRkernel/repr support (it uses the enumerate* environment)
    \usepackage[normalem]{ulem} % ulem is needed to support strikethroughs (\sout)
                                % normalem makes italics be italics, not underlines
    \usepackage{mathrsfs}
    

    
    % Colors for the hyperref package
    \definecolor{urlcolor}{rgb}{0,.145,.698}
    \definecolor{linkcolor}{rgb}{.71,0.21,0.01}
    \definecolor{citecolor}{rgb}{.12,.54,.11}

    % ANSI colors
    \definecolor{ansi-black}{HTML}{3E424D}
    \definecolor{ansi-black-intense}{HTML}{282C36}
    \definecolor{ansi-red}{HTML}{E75C58}
    \definecolor{ansi-red-intense}{HTML}{B22B31}
    \definecolor{ansi-green}{HTML}{00A250}
    \definecolor{ansi-green-intense}{HTML}{007427}
    \definecolor{ansi-yellow}{HTML}{DDB62B}
    \definecolor{ansi-yellow-intense}{HTML}{B27D12}
    \definecolor{ansi-blue}{HTML}{208FFB}
    \definecolor{ansi-blue-intense}{HTML}{0065CA}
    \definecolor{ansi-magenta}{HTML}{D160C4}
    \definecolor{ansi-magenta-intense}{HTML}{A03196}
    \definecolor{ansi-cyan}{HTML}{60C6C8}
    \definecolor{ansi-cyan-intense}{HTML}{258F8F}
    \definecolor{ansi-white}{HTML}{C5C1B4}
    \definecolor{ansi-white-intense}{HTML}{A1A6B2}
    \definecolor{ansi-default-inverse-fg}{HTML}{FFFFFF}
    \definecolor{ansi-default-inverse-bg}{HTML}{000000}

    % commands and environments needed by pandoc snippets
    % extracted from the output of `pandoc -s`
    \providecommand{\tightlist}{%
      \setlength{\itemsep}{0pt}\setlength{\parskip}{0pt}}
    \DefineVerbatimEnvironment{Highlighting}{Verbatim}{commandchars=\\\{\}}
    % Add ',fontsize=\small' for more characters per line
    \newenvironment{Shaded}{}{}
    \newcommand{\KeywordTok}[1]{\textcolor[rgb]{0.00,0.44,0.13}{\textbf{{#1}}}}
    \newcommand{\DataTypeTok}[1]{\textcolor[rgb]{0.56,0.13,0.00}{{#1}}}
    \newcommand{\DecValTok}[1]{\textcolor[rgb]{0.25,0.63,0.44}{{#1}}}
    \newcommand{\BaseNTok}[1]{\textcolor[rgb]{0.25,0.63,0.44}{{#1}}}
    \newcommand{\FloatTok}[1]{\textcolor[rgb]{0.25,0.63,0.44}{{#1}}}
    \newcommand{\CharTok}[1]{\textcolor[rgb]{0.25,0.44,0.63}{{#1}}}
    \newcommand{\StringTok}[1]{\textcolor[rgb]{0.25,0.44,0.63}{{#1}}}
    \newcommand{\CommentTok}[1]{\textcolor[rgb]{0.38,0.63,0.69}{\textit{{#1}}}}
    \newcommand{\OtherTok}[1]{\textcolor[rgb]{0.00,0.44,0.13}{{#1}}}
    \newcommand{\AlertTok}[1]{\textcolor[rgb]{1.00,0.00,0.00}{\textbf{{#1}}}}
    \newcommand{\FunctionTok}[1]{\textcolor[rgb]{0.02,0.16,0.49}{{#1}}}
    \newcommand{\RegionMarkerTok}[1]{{#1}}
    \newcommand{\ErrorTok}[1]{\textcolor[rgb]{1.00,0.00,0.00}{\textbf{{#1}}}}
    \newcommand{\NormalTok}[1]{{#1}}
    
    % Additional commands for more recent versions of Pandoc
    \newcommand{\ConstantTok}[1]{\textcolor[rgb]{0.53,0.00,0.00}{{#1}}}
    \newcommand{\SpecialCharTok}[1]{\textcolor[rgb]{0.25,0.44,0.63}{{#1}}}
    \newcommand{\VerbatimStringTok}[1]{\textcolor[rgb]{0.25,0.44,0.63}{{#1}}}
    \newcommand{\SpecialStringTok}[1]{\textcolor[rgb]{0.73,0.40,0.53}{{#1}}}
    \newcommand{\ImportTok}[1]{{#1}}
    \newcommand{\DocumentationTok}[1]{\textcolor[rgb]{0.73,0.13,0.13}{\textit{{#1}}}}
    \newcommand{\AnnotationTok}[1]{\textcolor[rgb]{0.38,0.63,0.69}{\textbf{\textit{{#1}}}}}
    \newcommand{\CommentVarTok}[1]{\textcolor[rgb]{0.38,0.63,0.69}{\textbf{\textit{{#1}}}}}
    \newcommand{\VariableTok}[1]{\textcolor[rgb]{0.10,0.09,0.49}{{#1}}}
    \newcommand{\ControlFlowTok}[1]{\textcolor[rgb]{0.00,0.44,0.13}{\textbf{{#1}}}}
    \newcommand{\OperatorTok}[1]{\textcolor[rgb]{0.40,0.40,0.40}{{#1}}}
    \newcommand{\BuiltInTok}[1]{{#1}}
    \newcommand{\ExtensionTok}[1]{{#1}}
    \newcommand{\PreprocessorTok}[1]{\textcolor[rgb]{0.74,0.48,0.00}{{#1}}}
    \newcommand{\AttributeTok}[1]{\textcolor[rgb]{0.49,0.56,0.16}{{#1}}}
    \newcommand{\InformationTok}[1]{\textcolor[rgb]{0.38,0.63,0.69}{\textbf{\textit{{#1}}}}}
    \newcommand{\WarningTok}[1]{\textcolor[rgb]{0.38,0.63,0.69}{\textbf{\textit{{#1}}}}}
    
    
    % Define a nice break command that doesn't care if a line doesn't already
    % exist.
    \def\br{\hspace*{\fill} \\* }
    % Math Jax compatibility definitions
    \def\gt{>}
    \def\lt{<}
    \let\Oldtex\TeX
    \let\Oldlatex\LaTeX
    \renewcommand{\TeX}{\textrm{\Oldtex}}
    \renewcommand{\LaTeX}{\textrm{\Oldlatex}}
    % Document parameters
    % Document title
    \title{2-1---bases-python}
    
    
    
    
    
% Pygments definitions
\makeatletter
\def\PY@reset{\let\PY@it=\relax \let\PY@bf=\relax%
    \let\PY@ul=\relax \let\PY@tc=\relax%
    \let\PY@bc=\relax \let\PY@ff=\relax}
\def\PY@tok#1{\csname PY@tok@#1\endcsname}
\def\PY@toks#1+{\ifx\relax#1\empty\else%
    \PY@tok{#1}\expandafter\PY@toks\fi}
\def\PY@do#1{\PY@bc{\PY@tc{\PY@ul{%
    \PY@it{\PY@bf{\PY@ff{#1}}}}}}}
\def\PY#1#2{\PY@reset\PY@toks#1+\relax+\PY@do{#2}}

\expandafter\def\csname PY@tok@w\endcsname{\def\PY@tc##1{\textcolor[rgb]{0.73,0.73,0.73}{##1}}}
\expandafter\def\csname PY@tok@c\endcsname{\let\PY@it=\textit\def\PY@tc##1{\textcolor[rgb]{0.25,0.50,0.50}{##1}}}
\expandafter\def\csname PY@tok@cp\endcsname{\def\PY@tc##1{\textcolor[rgb]{0.74,0.48,0.00}{##1}}}
\expandafter\def\csname PY@tok@k\endcsname{\let\PY@bf=\textbf\def\PY@tc##1{\textcolor[rgb]{0.00,0.50,0.00}{##1}}}
\expandafter\def\csname PY@tok@kp\endcsname{\def\PY@tc##1{\textcolor[rgb]{0.00,0.50,0.00}{##1}}}
\expandafter\def\csname PY@tok@kt\endcsname{\def\PY@tc##1{\textcolor[rgb]{0.69,0.00,0.25}{##1}}}
\expandafter\def\csname PY@tok@o\endcsname{\def\PY@tc##1{\textcolor[rgb]{0.40,0.40,0.40}{##1}}}
\expandafter\def\csname PY@tok@ow\endcsname{\let\PY@bf=\textbf\def\PY@tc##1{\textcolor[rgb]{0.67,0.13,1.00}{##1}}}
\expandafter\def\csname PY@tok@nb\endcsname{\def\PY@tc##1{\textcolor[rgb]{0.00,0.50,0.00}{##1}}}
\expandafter\def\csname PY@tok@nf\endcsname{\def\PY@tc##1{\textcolor[rgb]{0.00,0.00,1.00}{##1}}}
\expandafter\def\csname PY@tok@nc\endcsname{\let\PY@bf=\textbf\def\PY@tc##1{\textcolor[rgb]{0.00,0.00,1.00}{##1}}}
\expandafter\def\csname PY@tok@nn\endcsname{\let\PY@bf=\textbf\def\PY@tc##1{\textcolor[rgb]{0.00,0.00,1.00}{##1}}}
\expandafter\def\csname PY@tok@ne\endcsname{\let\PY@bf=\textbf\def\PY@tc##1{\textcolor[rgb]{0.82,0.25,0.23}{##1}}}
\expandafter\def\csname PY@tok@nv\endcsname{\def\PY@tc##1{\textcolor[rgb]{0.10,0.09,0.49}{##1}}}
\expandafter\def\csname PY@tok@no\endcsname{\def\PY@tc##1{\textcolor[rgb]{0.53,0.00,0.00}{##1}}}
\expandafter\def\csname PY@tok@nl\endcsname{\def\PY@tc##1{\textcolor[rgb]{0.63,0.63,0.00}{##1}}}
\expandafter\def\csname PY@tok@ni\endcsname{\let\PY@bf=\textbf\def\PY@tc##1{\textcolor[rgb]{0.60,0.60,0.60}{##1}}}
\expandafter\def\csname PY@tok@na\endcsname{\def\PY@tc##1{\textcolor[rgb]{0.49,0.56,0.16}{##1}}}
\expandafter\def\csname PY@tok@nt\endcsname{\let\PY@bf=\textbf\def\PY@tc##1{\textcolor[rgb]{0.00,0.50,0.00}{##1}}}
\expandafter\def\csname PY@tok@nd\endcsname{\def\PY@tc##1{\textcolor[rgb]{0.67,0.13,1.00}{##1}}}
\expandafter\def\csname PY@tok@s\endcsname{\def\PY@tc##1{\textcolor[rgb]{0.73,0.13,0.13}{##1}}}
\expandafter\def\csname PY@tok@sd\endcsname{\let\PY@it=\textit\def\PY@tc##1{\textcolor[rgb]{0.73,0.13,0.13}{##1}}}
\expandafter\def\csname PY@tok@si\endcsname{\let\PY@bf=\textbf\def\PY@tc##1{\textcolor[rgb]{0.73,0.40,0.53}{##1}}}
\expandafter\def\csname PY@tok@se\endcsname{\let\PY@bf=\textbf\def\PY@tc##1{\textcolor[rgb]{0.73,0.40,0.13}{##1}}}
\expandafter\def\csname PY@tok@sr\endcsname{\def\PY@tc##1{\textcolor[rgb]{0.73,0.40,0.53}{##1}}}
\expandafter\def\csname PY@tok@ss\endcsname{\def\PY@tc##1{\textcolor[rgb]{0.10,0.09,0.49}{##1}}}
\expandafter\def\csname PY@tok@sx\endcsname{\def\PY@tc##1{\textcolor[rgb]{0.00,0.50,0.00}{##1}}}
\expandafter\def\csname PY@tok@m\endcsname{\def\PY@tc##1{\textcolor[rgb]{0.40,0.40,0.40}{##1}}}
\expandafter\def\csname PY@tok@gh\endcsname{\let\PY@bf=\textbf\def\PY@tc##1{\textcolor[rgb]{0.00,0.00,0.50}{##1}}}
\expandafter\def\csname PY@tok@gu\endcsname{\let\PY@bf=\textbf\def\PY@tc##1{\textcolor[rgb]{0.50,0.00,0.50}{##1}}}
\expandafter\def\csname PY@tok@gd\endcsname{\def\PY@tc##1{\textcolor[rgb]{0.63,0.00,0.00}{##1}}}
\expandafter\def\csname PY@tok@gi\endcsname{\def\PY@tc##1{\textcolor[rgb]{0.00,0.63,0.00}{##1}}}
\expandafter\def\csname PY@tok@gr\endcsname{\def\PY@tc##1{\textcolor[rgb]{1.00,0.00,0.00}{##1}}}
\expandafter\def\csname PY@tok@ge\endcsname{\let\PY@it=\textit}
\expandafter\def\csname PY@tok@gs\endcsname{\let\PY@bf=\textbf}
\expandafter\def\csname PY@tok@gp\endcsname{\let\PY@bf=\textbf\def\PY@tc##1{\textcolor[rgb]{0.00,0.00,0.50}{##1}}}
\expandafter\def\csname PY@tok@go\endcsname{\def\PY@tc##1{\textcolor[rgb]{0.53,0.53,0.53}{##1}}}
\expandafter\def\csname PY@tok@gt\endcsname{\def\PY@tc##1{\textcolor[rgb]{0.00,0.27,0.87}{##1}}}
\expandafter\def\csname PY@tok@err\endcsname{\def\PY@bc##1{\setlength{\fboxsep}{0pt}\fcolorbox[rgb]{1.00,0.00,0.00}{1,1,1}{\strut ##1}}}
\expandafter\def\csname PY@tok@kc\endcsname{\let\PY@bf=\textbf\def\PY@tc##1{\textcolor[rgb]{0.00,0.50,0.00}{##1}}}
\expandafter\def\csname PY@tok@kd\endcsname{\let\PY@bf=\textbf\def\PY@tc##1{\textcolor[rgb]{0.00,0.50,0.00}{##1}}}
\expandafter\def\csname PY@tok@kn\endcsname{\let\PY@bf=\textbf\def\PY@tc##1{\textcolor[rgb]{0.00,0.50,0.00}{##1}}}
\expandafter\def\csname PY@tok@kr\endcsname{\let\PY@bf=\textbf\def\PY@tc##1{\textcolor[rgb]{0.00,0.50,0.00}{##1}}}
\expandafter\def\csname PY@tok@bp\endcsname{\def\PY@tc##1{\textcolor[rgb]{0.00,0.50,0.00}{##1}}}
\expandafter\def\csname PY@tok@fm\endcsname{\def\PY@tc##1{\textcolor[rgb]{0.00,0.00,1.00}{##1}}}
\expandafter\def\csname PY@tok@vc\endcsname{\def\PY@tc##1{\textcolor[rgb]{0.10,0.09,0.49}{##1}}}
\expandafter\def\csname PY@tok@vg\endcsname{\def\PY@tc##1{\textcolor[rgb]{0.10,0.09,0.49}{##1}}}
\expandafter\def\csname PY@tok@vi\endcsname{\def\PY@tc##1{\textcolor[rgb]{0.10,0.09,0.49}{##1}}}
\expandafter\def\csname PY@tok@vm\endcsname{\def\PY@tc##1{\textcolor[rgb]{0.10,0.09,0.49}{##1}}}
\expandafter\def\csname PY@tok@sa\endcsname{\def\PY@tc##1{\textcolor[rgb]{0.73,0.13,0.13}{##1}}}
\expandafter\def\csname PY@tok@sb\endcsname{\def\PY@tc##1{\textcolor[rgb]{0.73,0.13,0.13}{##1}}}
\expandafter\def\csname PY@tok@sc\endcsname{\def\PY@tc##1{\textcolor[rgb]{0.73,0.13,0.13}{##1}}}
\expandafter\def\csname PY@tok@dl\endcsname{\def\PY@tc##1{\textcolor[rgb]{0.73,0.13,0.13}{##1}}}
\expandafter\def\csname PY@tok@s2\endcsname{\def\PY@tc##1{\textcolor[rgb]{0.73,0.13,0.13}{##1}}}
\expandafter\def\csname PY@tok@sh\endcsname{\def\PY@tc##1{\textcolor[rgb]{0.73,0.13,0.13}{##1}}}
\expandafter\def\csname PY@tok@s1\endcsname{\def\PY@tc##1{\textcolor[rgb]{0.73,0.13,0.13}{##1}}}
\expandafter\def\csname PY@tok@mb\endcsname{\def\PY@tc##1{\textcolor[rgb]{0.40,0.40,0.40}{##1}}}
\expandafter\def\csname PY@tok@mf\endcsname{\def\PY@tc##1{\textcolor[rgb]{0.40,0.40,0.40}{##1}}}
\expandafter\def\csname PY@tok@mh\endcsname{\def\PY@tc##1{\textcolor[rgb]{0.40,0.40,0.40}{##1}}}
\expandafter\def\csname PY@tok@mi\endcsname{\def\PY@tc##1{\textcolor[rgb]{0.40,0.40,0.40}{##1}}}
\expandafter\def\csname PY@tok@il\endcsname{\def\PY@tc##1{\textcolor[rgb]{0.40,0.40,0.40}{##1}}}
\expandafter\def\csname PY@tok@mo\endcsname{\def\PY@tc##1{\textcolor[rgb]{0.40,0.40,0.40}{##1}}}
\expandafter\def\csname PY@tok@ch\endcsname{\let\PY@it=\textit\def\PY@tc##1{\textcolor[rgb]{0.25,0.50,0.50}{##1}}}
\expandafter\def\csname PY@tok@cm\endcsname{\let\PY@it=\textit\def\PY@tc##1{\textcolor[rgb]{0.25,0.50,0.50}{##1}}}
\expandafter\def\csname PY@tok@cpf\endcsname{\let\PY@it=\textit\def\PY@tc##1{\textcolor[rgb]{0.25,0.50,0.50}{##1}}}
\expandafter\def\csname PY@tok@c1\endcsname{\let\PY@it=\textit\def\PY@tc##1{\textcolor[rgb]{0.25,0.50,0.50}{##1}}}
\expandafter\def\csname PY@tok@cs\endcsname{\let\PY@it=\textit\def\PY@tc##1{\textcolor[rgb]{0.25,0.50,0.50}{##1}}}

\def\PYZbs{\char`\\}
\def\PYZus{\char`\_}
\def\PYZob{\char`\{}
\def\PYZcb{\char`\}}
\def\PYZca{\char`\^}
\def\PYZam{\char`\&}
\def\PYZlt{\char`\<}
\def\PYZgt{\char`\>}
\def\PYZsh{\char`\#}
\def\PYZpc{\char`\%}
\def\PYZdl{\char`\$}
\def\PYZhy{\char`\-}
\def\PYZsq{\char`\'}
\def\PYZdq{\char`\"}
\def\PYZti{\char`\~}
% for compatibility with earlier versions
\def\PYZat{@}
\def\PYZlb{[}
\def\PYZrb{]}
\makeatother


    % For linebreaks inside Verbatim environment from package fancyvrb. 
    \makeatletter
        \newbox\Wrappedcontinuationbox 
        \newbox\Wrappedvisiblespacebox 
        \newcommand*\Wrappedvisiblespace {\textcolor{red}{\textvisiblespace}} 
        \newcommand*\Wrappedcontinuationsymbol {\textcolor{red}{\llap{\tiny$\m@th\hookrightarrow$}}} 
        \newcommand*\Wrappedcontinuationindent {3ex } 
        \newcommand*\Wrappedafterbreak {\kern\Wrappedcontinuationindent\copy\Wrappedcontinuationbox} 
        % Take advantage of the already applied Pygments mark-up to insert 
        % potential linebreaks for TeX processing. 
        %        {, <, #, %, $, ' and ": go to next line. 
        %        _, }, ^, &, >, - and ~: stay at end of broken line. 
        % Use of \textquotesingle for straight quote. 
        \newcommand*\Wrappedbreaksatspecials {% 
            \def\PYGZus{\discretionary{\char`\_}{\Wrappedafterbreak}{\char`\_}}% 
            \def\PYGZob{\discretionary{}{\Wrappedafterbreak\char`\{}{\char`\{}}% 
            \def\PYGZcb{\discretionary{\char`\}}{\Wrappedafterbreak}{\char`\}}}% 
            \def\PYGZca{\discretionary{\char`\^}{\Wrappedafterbreak}{\char`\^}}% 
            \def\PYGZam{\discretionary{\char`\&}{\Wrappedafterbreak}{\char`\&}}% 
            \def\PYGZlt{\discretionary{}{\Wrappedafterbreak\char`\<}{\char`\<}}% 
            \def\PYGZgt{\discretionary{\char`\>}{\Wrappedafterbreak}{\char`\>}}% 
            \def\PYGZsh{\discretionary{}{\Wrappedafterbreak\char`\#}{\char`\#}}% 
            \def\PYGZpc{\discretionary{}{\Wrappedafterbreak\char`\%}{\char`\%}}% 
            \def\PYGZdl{\discretionary{}{\Wrappedafterbreak\char`\$}{\char`\$}}% 
            \def\PYGZhy{\discretionary{\char`\-}{\Wrappedafterbreak}{\char`\-}}% 
            \def\PYGZsq{\discretionary{}{\Wrappedafterbreak\textquotesingle}{\textquotesingle}}% 
            \def\PYGZdq{\discretionary{}{\Wrappedafterbreak\char`\"}{\char`\"}}% 
            \def\PYGZti{\discretionary{\char`\~}{\Wrappedafterbreak}{\char`\~}}% 
        } 
        % Some characters . , ; ? ! / are not pygmentized. 
        % This macro makes them "active" and they will insert potential linebreaks 
        \newcommand*\Wrappedbreaksatpunct {% 
            \lccode`\~`\.\lowercase{\def~}{\discretionary{\hbox{\char`\.}}{\Wrappedafterbreak}{\hbox{\char`\.}}}% 
            \lccode`\~`\,\lowercase{\def~}{\discretionary{\hbox{\char`\,}}{\Wrappedafterbreak}{\hbox{\char`\,}}}% 
            \lccode`\~`\;\lowercase{\def~}{\discretionary{\hbox{\char`\;}}{\Wrappedafterbreak}{\hbox{\char`\;}}}% 
            \lccode`\~`\:\lowercase{\def~}{\discretionary{\hbox{\char`\:}}{\Wrappedafterbreak}{\hbox{\char`\:}}}% 
            \lccode`\~`\?\lowercase{\def~}{\discretionary{\hbox{\char`\?}}{\Wrappedafterbreak}{\hbox{\char`\?}}}% 
            \lccode`\~`\!\lowercase{\def~}{\discretionary{\hbox{\char`\!}}{\Wrappedafterbreak}{\hbox{\char`\!}}}% 
            \lccode`\~`\/\lowercase{\def~}{\discretionary{\hbox{\char`\/}}{\Wrappedafterbreak}{\hbox{\char`\/}}}% 
            \catcode`\.\active
            \catcode`\,\active 
            \catcode`\;\active
            \catcode`\:\active
            \catcode`\?\active
            \catcode`\!\active
            \catcode`\/\active 
            \lccode`\~`\~ 	
        }
    \makeatother

    \let\OriginalVerbatim=\Verbatim
    \makeatletter
    \renewcommand{\Verbatim}[1][1]{%
        %\parskip\z@skip
        \sbox\Wrappedcontinuationbox {\Wrappedcontinuationsymbol}%
        \sbox\Wrappedvisiblespacebox {\FV@SetupFont\Wrappedvisiblespace}%
        \def\FancyVerbFormatLine ##1{\hsize\linewidth
            \vtop{\raggedright\hyphenpenalty\z@\exhyphenpenalty\z@
                \doublehyphendemerits\z@\finalhyphendemerits\z@
                \strut ##1\strut}%
        }%
        % If the linebreak is at a space, the latter will be displayed as visible
        % space at end of first line, and a continuation symbol starts next line.
        % Stretch/shrink are however usually zero for typewriter font.
        \def\FV@Space {%
            \nobreak\hskip\z@ plus\fontdimen3\font minus\fontdimen4\font
            \discretionary{\copy\Wrappedvisiblespacebox}{\Wrappedafterbreak}
            {\kern\fontdimen2\font}%
        }%
        
        % Allow breaks at special characters using \PYG... macros.
        \Wrappedbreaksatspecials
        % Breaks at punctuation characters . , ; ? ! and / need catcode=\active 	
        \OriginalVerbatim[#1,codes*=\Wrappedbreaksatpunct]%
    }
    \makeatother

    % Exact colors from NB
    \definecolor{incolor}{HTML}{303F9F}
    \definecolor{outcolor}{HTML}{D84315}
    \definecolor{cellborder}{HTML}{CFCFCF}
    \definecolor{cellbackground}{HTML}{F7F7F7}
    
    % prompt
    \makeatletter
    \newcommand{\boxspacing}{\kern\kvtcb@left@rule\kern\kvtcb@boxsep}
    \makeatother
    \newcommand{\prompt}[4]{
        \ttfamily\llap{{\color{#2}[#3]:\hspace{3pt}#4}}\vspace{-\baselineskip}
    }
    

    
\setlength\headheight{30pt}
\setcounter{secnumdepth}{0} % Turns off numbering for sections

    % Prevent overflowing lines due to hard-to-break entities
    \sloppy 
    % Setup hyperref package
    \hypersetup{
      breaklinks=true,  % so long urls are correctly broken across lines
      colorlinks=true,
      urlcolor=urlcolor,
      linkcolor=linkcolor,
      citecolor=citecolor,
      }
    % Slightly bigger margins than the latex defaults
    \geometry{a4paper,tmargin=3cm,bmargin=2cm,lmargin=1cm,rmargin=1cm}\fancyhead[L]{Thème à définir}\fancyhead[L]{\adjustimage{height=1cm, valign=m}{/home/bouscadilla/Documents/Code/nbconvert/template/latex/pdf_solution/papier_eleve_ico_langage}\ttfamily\scshape Langage}\fancyhead[C]{\bfseries\MakeUppercase{2-1---bases-python}}\fancyhead[C]{\bfseries\MakeUppercase{2 --- Programmer en Python}}\fancyhead[R]{\monthyeardate\today}

    \fancyfoot[C]{\thepage}
    % #TODO ajouter les pages totales

    \pagestyle{fancy}
    


\begin{document}
    
    \title{2 --- Programmer en Python}
% \maketitle

    
    

    
    \hypertarget{chap.-2-programmer-en-python-pa.dilla.fr15}{%
\section[Chap. 2 --- Programmer en Python
(\href{https://pa.dilla.fr/15}{pa.dilla.fr/15} )]{\texorpdfstring{Chap.
2 --- Programmer en Python
(\href{https://pa.dilla.fr/15}{pa.dilla.fr/15}
\protect\includegraphics{res/qr-basthon.png})}{Chap. 2 --- Programmer en Python (pa.dilla.fr/15 )}}\label{chap.-2-programmer-en-python-pa.dilla.fr15}}
\begin{eleve}
    D'après toi, pourquoi HTML et CSS ne sont \textbf{pas} des langages de
programmation ?
        
        \end{eleve}\begin{reponse}
    HTML et CSS sont incapables d'effectuer le moindre calcul.

On peut aussi remarquer qu'il manque des instructions essentielles comme
les répétitions, les branchements conditionnels ou encore les
affectations.

        \end{reponse}
    Python est un langage de programmation qui peut être utilisé de
différentes façon. Dans ce cours tu verras :

\begin{itemize}
\tightlist
\item
  le mode \emph{interactif} dans des notebooks
\item
  le mode \emph{programme} dans l'interface de développement (IDE)
  VSCodium.
\end{itemize}

    \hypertarget{le-mode-interactif}{%
\subsection{1 --- Le mode interactif}\label{le-mode-interactif}}

    Le mode interactif se comporte un peu comme une calculatrice. Nous
allons explorer ce mode à travers les \textbf{notebooks}.

    \hypertarget{les-notebooks}{%
\subsubsection{1.1 --- Les notebooks}\label{les-notebooks}}

    Un \emph{notebook} est une page HTML qui contient des blocs de texte et
des blocs de code :

\begin{itemize}
\tightlist
\item
  les blocs de \textbf{texte} permettent d'afficher et de modifier du
  texte écrit en langage \texttt{Markdown}. Cette simplification du HTML
  permet d'afficher du texte (normal, gras, italique, etc.), des listes,
  des tableaux, des liens, du code ou encore des média comme les images
  et les vidéos.
\item
  les blocs de \textbf{code} permettent de saisir, modifier et
  d'exécuter du code Python. L'interprète Python est chargé avec la page
  HTML puis ensuite il calcule et affiche les résultats.
\end{itemize}
\begin{exemple}
    La page de ce cours consultable à l'adresse \url{https://pa.dilla.fr/15}
\textbf{est un notebook}.

        \end{exemple}\begin{exemple}
        {\scriptsize
    \begin{tcolorbox}[breakable, size=fbox, boxrule=1pt, pad at break*=1mm,colback=cellbackground, colframe=cellborder]
\prompt{In}{incolor}{ }{\boxspacing}
\begin{Verbatim}[commandchars=\\\{\}]
\PY{c+c1}{\PYZsh{} Ceci est un bloc de CODE.}
\PY{c+c1}{\PYZsh{} Les lignes qui commencent pas un \PYZsh{} sont des commentaires}
\PY{c+c1}{\PYZsh{} Les commentaires ne seront pas interprétés par Python}
\PY{c+c1}{\PYZsh{}}
\PY{c+c1}{\PYZsh{} Les blocs de code se reconnaissent facilement car ils}
\PY{c+c1}{\PYZsh{} possèdent une indication dans la marge \PYZdq{}In []\PYZdq{} ou \PYZdq{}Entrée []\PYZdq{}}
\PY{c+c1}{\PYZsh{} Si le bloc a été exécuté, alors un compteur apparait dans \PYZdq{}[...]\PYZdq{}}

\PY{n+nb}{print}\PY{p}{(}\PY{l+s+s2}{\PYZdq{}}\PY{l+s+s2}{Dans le notebook, ceci est un bloc de code}\PY{l+s+s2}{\PYZdq{}}\PY{p}{)}
\end{Verbatim}
\end{tcolorbox}
    }

        \end{exemple}
    Pour \textbf{modifier un bloc de texte ou de code} tu peux utiliser la
souris ou le clavier :

\begin{enumerate}
\def\labelenumi{\arabic{enumi}.}
\tightlist
\item
  méthode avec la souris :

  \begin{itemize}
  \tightlist
  \item
    double-cliquer sur ce bloc de texte pour l'\emph{éditer},
  \item
    modifier/corriger le texte
  \item
    cliquer sur le bouton \(\fbox{\texttt{Run}}\) ou
    \(\fbox{\texttt{Exécuter}}\) pour interpréter ce bloc et sortir du
    mode \emph{édition}
  \end{itemize}
\item
  méthode au clavier :

  \begin{itemize}
  \tightlist
  \item
    utiliser les flèches \texttt{\textless{}}↑\texttt{\textgreater{}} et
    \texttt{\textless{}}↓\texttt{\textgreater{}} pour sélectionner le
    bloc de texte à modifier
  \item
    appuyer sur \texttt{\textless{}Entrée\textgreater{}} pour
    \emph{éditer} le bloc
  \item
    modifier/corriger le texte
  \item
    pour interpréter le résultats tu peux utiliser au choix :

    \begin{itemize}
    \tightlist
    \item
      \texttt{\textless{}Ctrl\textgreater{}\ +\ \textless{}Entrée\textgreater{}}
      pour interpréter le bloc ou
    \item
      \texttt{\textless{}Shift\textgreater{}\ +\ \textless{}Entrée\textgreater{}}
      pour interpréter le bloc \emph{et} passer au bloc suivant.
    \end{itemize}
  \end{itemize}
\end{enumerate}
\begin{remarque}
    Souvent quand on ouvre un notebook, aucun bloc de code n'a été exécuté.
il faut donc passer \emph{dans l'ordre} par tous les blocs de code et
les exécuter.

C'est pourquoi, quand on lit notebook, il est pratique d'utiliser le
raccourcis clavier
\texttt{\textless{}Shift\textgreater{}\ +\ \textless{}Entrée\textgreater{}}
permettant de passer d'un bloc au suivant.

        \end{remarque}
    Pour \textbf{ajouter des blocs} (de code ou de texte) :

\begin{enumerate}
\def\labelenumi{\arabic{enumi}.}
\tightlist
\item
  tu peux utiliser dans les outils le bouton \texttt{+}
  \includegraphics{res/tuto-01.png}.
\item
  Si tu souhaites que le bloc soit du code, sélectionne \texttt{Code}
  dans la liste déroulante.
\item
  Si tu souhaites un bloc de texte, sélectionne alors \texttt{Markdown}.
\end{enumerate}

\includegraphics{res/tuto-02.png}
\begin{eleve}
    \textbf{Accède} à la version \emph{notebook} de ce chapitre
(\url{https://pa.dilla.fr/15}).

\begin{enumerate}
\def\labelenumi{\arabic{enumi}.}
\tightlist
\item
  \textbf{Corrige} la faute d'aurthografe de ce bloc de texte.
\item
  \textbf{Ajoute} juste en dessous un bloc de code et \textbf{écris} y :
  \texttt{1+2}.
\item
  Puis fait \textbf{calculer} le résultats par l'interprète Python.
\end{enumerate}
        
        \end{eleve}
    \hypertarget{arithmuxe9tique}{%
\subsubsection{1.2 --- Arithmétique}\label{arithmuxe9tique}}

    En Python, on peut saisir des combinaisons arbitraires d'opérations
arithmétiques, en utilisant notamment les quatre opérations les plus
communes.
\begin{exemple}
    \begin{Shaded}
\begin{Highlighting}[]
\DecValTok{2} \OperatorTok{+} \DecValTok{5} \OperatorTok{*}\NormalTok{ (}\DecValTok{10} \OperatorTok{{-}} \DecValTok{1} \OperatorTok{/} \DecValTok{2}\NormalTok{)}
\end{Highlighting}
\end{Shaded}

        \end{exemple}\begin{remarque}
    Quelques remarques

\begin{enumerate}
\def\labelenumi{\arabic{enumi}.}
\tightlist
\item
  Les symboles des quatre opérations sont \texttt{+} (addition),
  \texttt{-} (soustraction), \texttt{*} (multiplication) et \texttt{/}
  (division).
\item
  La priorité des opérations est usuelle et on peut utiliser les
  parenthèses.
\item
  Les espaces sont purement décoratifs : ils rendent plus lisibles mais
  n'agissent pas sur la priorité des opérations. Par exemple :
  \texttt{1+2\ \ \ \ \ *\ 3} renvoie \texttt{7} et non pas \texttt{9}!
\end{enumerate}

        \end{remarque}
    Remarque bien que l'interprète Python n'accepte pas les expressions qui
sont mal formées ou comportent des erreurs.
\begin{eleve}
    Dans le bloc de code ci-dessous, \textbf{écris} \texttt{1\ +\ *\ 2}.

\textbf{Indique} le type d'erreur affiché.
        
        \end{eleve}\begin{reponse}
    Lorsque l'on fait des erreurs de syntaxe car l'expression est mal
formée, l'interprète affiche \texttt{SyntaxError}. Pour aider
l'utilisateur, l'interprète montre l'endroit exact de l'erreur avec le
symbole \texttt{\^{}} qui pointe sous le \texttt{*}.

        \end{reponse}\begin{eleve}
    \emph{Pour répondre aux questions suivantes, tu peux ajouter des blocs
de texte ou des blocs de code\ldots{}}

En mathématique, est-il possible de \textbf{diviser un nombre par zéro}
?

Et en \textbf{Python} ?
        
        \end{eleve}
    \hypertarget{les-nombres}{%
\subsubsection{1.3 --- Les nombres}\label{les-nombres}}
\begin{retenir}
    En Python, les nombres sont de deux \emph{types} :

\begin{itemize}
\tightlist
\item
  soit des \textbf{entiers relatifs} (appelés \emph{entiers})
\item
  soit des \textbf{nombres décimaux} (appelés \emph{flottants}).
\end{itemize}

        \end{retenir}\begin{remarque}
    Attention, les nombres flottans s'écrivent à \emph{l'anglo-saxonne} avec
un point \texttt{3.14} à la place de la virgule \(3,14\).

        \end{remarque}\begin{eleve}
    \textbf{Indique} ce qu'affiche les commandes suivantes :

\begin{enumerate}
\def\labelenumi{\arabic{enumi}.}
\tightlist
\item
  Commande 1 : \texttt{1,2\ +\ 3,4}.
\item
  Commande 2 : \texttt{1,2\ *\ 3}
\end{enumerate}

\textbf{D'après toi}, à quoi sert le symbole virgule \texttt{,} en
Python ?
        
        \end{eleve}
    En Python :

\begin{itemize}
\tightlist
\item
  les \emph{entiers} sont de \textbf{taille arbitraire} ;
\item
  les \emph{flottants} en revanche ont une \textbf{capacité limitée} et
  sont souvent des approximations qui ne représentent alors qu'une
  partie des nombres décimaux.
\end{itemize}
\begin{retenir}
    Quand tu en as le choix, travaille de préférence avec les entiers.

        \end{retenir}
    \hypertarget{autres-opuxe9rations}{%
\subsubsection{1.4 --- Autres opérations}\label{autres-opuxe9rations}}
\begin{eleve}
    \textbf{Détermine} le type de nombre que l'on obtient lorsque l'on
divise entre eux deux nombres entiers.
        
        \end{eleve}\begin{reponse}
    La division avec l'opération \texttt{/} donne un nombre flottant. Ainsi
\texttt{10/2} donne \texttt{5.0} et \texttt{7/2} donne \texttt{3.5}. Ces
deux résultats sont de types \emph{flottant} car ils sont écrits avec
une virgule.

        \end{reponse}\begin{eleve}
    \textbf{Effectue} \emph{à la main} les divisions euclidiennes \(7 / 2\)
et \({98} / {10}\).
        
        \end{eleve}\begin{reponse}
    \(7/2\) donne \(3\) et il reste \(1\). L'égalité \(7 = 3 \times 2 + 1\)
est vérifiée.

\(98 / 10\) donne \(9\) et il reste \(8\). L'égalité
\(98 = 9 \times 10 + 8\) est vérifiée.

        \end{reponse}\begin{retenir}
    Pour effectuer la \textbf{division entière}, il faut utiliser
l'opération \texttt{//}.

Ainsi, \texttt{a\ //\ b} peut être vu comme le plus grand nombre entier
inférieur ou égal à \(a/b\).

        \end{retenir}\begin{retenir}
    Pour obtenir le \textbf{reste de la division euclidienne}, on utilise
l'opération \texttt{\%} (qui se dit \emph{``modulo''}).

        \end{retenir}\begin{eleve}
    \textbf{Vérifie} à l'aide des opérations \emph{division entière} et
\emph{modulo} les résultats de l'activité précédente.
        
        \end{eleve}\begin{reponse}
    Utiliser quatre blocs de code pour afficher les résultats des calculs
suivants :

\begin{Shaded}
\begin{Highlighting}[]
\DecValTok{7}\OperatorTok{//}\DecValTok{2}
\DecValTok{7} \OperatorTok{\%} \DecValTok{2}
\DecValTok{98} \OperatorTok{//} \DecValTok{10}
\DecValTok{98} \OperatorTok{\%} \DecValTok{10}
\end{Highlighting}
\end{Shaded}

        \end{reponse}\begin{retenir}
    L'opération \texttt{**} calcule la \textbf{puissance} d'un nombre.

Ainsi, \(10^{3}\) s'écrit \texttt{10**3}.

        \end{retenir}\begin{eleve}
    \textbf{Calcule} dans un bloc de code \(2^{1024}\) puis dans un autre
bloc de code : \((2.0) ^ {1024}\).

\textbf{Propose} une explication à tes observations.
        
        \end{eleve}\begin{reponse}
    \texttt{2**1024} est un nombre entier et se calcule correctement.

En revance, \texttt{2.0**1024} est un nombre flottant trop grand pour
être calculé. \texttt{OverfflowError} indique que le résultat est hors
intervalle.

        \end{reponse}
    \hypertarget{variables}{%
\subsubsection{1.5 --- Variables}\label{variables}}

    Les résultats calculés peuvent être \textbf{mémorisés} afin d'être
utilisés plus tard.

        {\scriptsize
    \begin{tcolorbox}[breakable, size=fbox, boxrule=1pt, pad at break*=1mm,colback=cellbackground, colframe=cellborder]
\prompt{In}{incolor}{ }{\boxspacing}
\begin{Verbatim}[commandchars=\\\{\}]
\PY{n}{a} \PY{o}{=} \PY{l+m+mi}{1} \PY{o}{+} \PY{l+m+mi}{1}
\end{Verbatim}
\end{tcolorbox}
    }
\begin{retenir}
    Une variable peut être imaginée comme un \textbf{emplacement en mémoire}
portant une \textbf{étiquette} et contenant une \textbf{valeur}.

        \end{retenir}\begin{exemple}
    \texttt{a\ =\ 2} se représente par un emplacement \texttt{a} contenant
la valeur \texttt{2}: \(\overset{\texttt{a}}{\fbox{2}}\)

        \end{exemple}
    La notation \texttt{a\ =} permet de donner un nom à l'information
mémorisée. Ici, l'interprète Python calcule le résultat de
\texttt{1\ +\ 1} et le mémorise dans la \textbf{variable} \texttt{a}.

Aucun résultat n'est affiché. On accède à la valeur mémorisée en
utilisant le nom \texttt{a}.

        {\scriptsize
    \begin{tcolorbox}[breakable, size=fbox, boxrule=1pt, pad at break*=1mm,colback=cellbackground, colframe=cellborder]
\prompt{In}{incolor}{ }{\boxspacing}
\begin{Verbatim}[commandchars=\\\{\}]
\PY{n}{a}
\end{Verbatim}
\end{tcolorbox}
    }

    Le bloc ci-dessous affiche le résultat du calcul
\texttt{a\ *\ (a\ +\ 1)} :

        {\scriptsize
    \begin{tcolorbox}[breakable, size=fbox, boxrule=1pt, pad at break*=1mm,colback=cellbackground, colframe=cellborder]
\prompt{In}{incolor}{ }{\boxspacing}
\begin{Verbatim}[commandchars=\\\{\}]
\PY{n}{a} \PY{o}{*} \PY{p}{(}\PY{n}{a}\PY{o}{+}\PY{l+m+mi}{1}\PY{p}{)}
\end{Verbatim}
\end{tcolorbox}
    }
\begin{retenir}
    Le symbole \texttt{=} utilisé en Python pour définir la variable
\texttt{a} désigne une opération d'\textbf{affectation}. En algorithme,
on note l'affectation par le symbole `\(\leftarrow\)', ce qui donne par
exemple : \(\texttt{a} \leftarrow \texttt{2}\).

Ce symbole attend à \emph{gauche} un nom de variable et à \emph{droite}
une expression.

On peut donner une nouvelle valeur à la variable \texttt{a} avec une
nouvelle affectation. Cette valeur \emph{remplace} la précédente.

        \end{retenir}
        {\scriptsize
    \begin{tcolorbox}[breakable, size=fbox, boxrule=1pt, pad at break*=1mm,colback=cellbackground, colframe=cellborder]
\prompt{In}{incolor}{ }{\boxspacing}
\begin{Verbatim}[commandchars=\\\{\}]
\PY{n}{a} \PY{o}{=} \PY{l+m+mi}{3}
\PY{n}{a} \PY{o}{*} \PY{p}{(}\PY{n}{a}\PY{o}{+}\PY{l+m+mi}{1}\PY{p}{)}
\end{Verbatim}
\end{tcolorbox}
    }

    Le calcul de la nouvelle valeur de \texttt{a} peut utiliser la valeur
courante de \texttt{a}.

        {\scriptsize
    \begin{tcolorbox}[breakable, size=fbox, boxrule=1pt, pad at break*=1mm,colback=cellbackground, colframe=cellborder]
\prompt{In}{incolor}{ }{\boxspacing}
\begin{Verbatim}[commandchars=\\\{\}]
\PY{n}{a} \PY{o}{=} \PY{n}{a} \PY{o}{+} \PY{l+m+mi}{1}
\PY{n}{a}
\end{Verbatim}
\end{tcolorbox}
    }
\begin{remarque}
    Un nom de variable peut être formé de plusieurs caractères (lettre,
chiffres et tiret bas). Il est recommandé de :

\begin{itemize}
\tightlist
\item
  ne pas utilisé de caractères accentués
\item
  se limiter aux caractères minuscules.
\end{itemize}

        \end{remarque}
        {\scriptsize
    \begin{tcolorbox}[breakable, size=fbox, boxrule=1pt, pad at break*=1mm,colback=cellbackground, colframe=cellborder]
\prompt{In}{incolor}{ }{\boxspacing}
\begin{Verbatim}[commandchars=\\\{\}]
\PY{n}{cube} \PY{o}{=} \PY{n}{a} \PY{o}{*} \PY{n}{a} \PY{o}{*} \PY{n}{a}
\PY{n}{ma\PYZus{}variable} \PY{o}{=} \PY{l+m+mi}{42}
\PY{n}{ma\PYZus{}variable2} \PY{o}{=} \PY{l+m+mi}{2021}
\end{Verbatim}
\end{tcolorbox}
    }

    \hypertarget{uxe9tat}{%
\subsubsection{1.6 --- État}\label{uxe9tat}}
\begin{retenir}
    L'ensemble des associations entre des noms de variables et des valeurs
constitue \textbf{l'état} de l'interprète Python.

        \end{retenir}
    L'état évolue en fonction des instructions exécutées. Les instructions
qui modifient l'état sont dites \emph{à effet de bord}.
\begin{eleve}
    À partir de l'état initial
\(\overset{\texttt{a}}{\fbox{1}}\overset{\texttt{b}}{\fbox{2}}\overset{\texttt{c}}{\fbox{3}}\overset{\texttt{d}}{\fbox{-12}}\),
les instructions suivantes sont exécutées :

\begin{Shaded}
\begin{Highlighting}[]
\NormalTok{a }\OperatorTok{=}\NormalTok{ c }\OperatorTok{{-}}\NormalTok{ a}
\NormalTok{e }\OperatorTok{=}\NormalTok{ b }\OperatorTok{+}\NormalTok{ c}
\NormalTok{d }\OperatorTok{=}\NormalTok{ a}
\NormalTok{a }\OperatorTok{=} \DecValTok{7}
\end{Highlighting}
\end{Shaded}

Après la première instruction \texttt{a\ =\ c\ -\ a}, l'état initial
devient alors :
\(\overset{\texttt{a}}{\fbox{2}}\overset{\texttt{b}}{\fbox{2}}\overset{\texttt{c}}{\fbox{3}}\overset{\texttt{d}}{\fbox{-12}}.\)

\textbf{Indique} les états suivants après l'exécution des trois
dernières instructions.
        
        \end{eleve}\begin{reponse}
    L'instruction \texttt{e\ =\ b\ +\ c} affecte la variable \texttt{e} pour
la première fois. Celle-ci fait donc son apparition dans l'état :

\[
\overset{\texttt{a}}{\fbox{2}}\overset{\texttt{b}}{\fbox{2}}\overset{\texttt{c}}{\fbox{3}}\overset{\texttt{d}}{\fbox{-12}}\overset{\texttt{e}}{\fbox{5}}.
\]

L'instructions \texttt{d\ =\ a} affecte à \texttt{d} la valeur pointée
par \texttt{a}. Il est important de comprendre que \texttt{a} et
\texttt{d} sont deux emplacements différents et donc indépendants.
Ainsi, l'instruction suivante n'aura pas d'effet sur la variable
\texttt{d}

\[
\overset{\texttt{a}}{\fbox{2}}\overset{\texttt{b}}{\fbox{2}}\overset{\texttt{c}}{\fbox{3}}\overset{\texttt{d}}{\fbox{2}}\overset{\texttt{e}}{\fbox{5}}.
\]

La dernière instruction \texttt{a\ =\ 7} affecte à la variable
\texttt{a} la valeur \texttt{7}.

\[
\overset{\texttt{a}}{\fbox{7}}\overset{\texttt{b}}{\fbox{2}}\overset{\texttt{c}}{\fbox{3}}\overset{\texttt{d}}{\fbox{2}}\overset{\texttt{e}}{\fbox{5}}.
\]

On peut voir une modélisation de l'état sur le site suivant :
\href{https://pythontutor.com/visualize.html\#code=a\%20\%3D\%201\%0Ab\%20\%3D\%202\%0Ac\%20\%3D\%203\%0Ad\%20\%3D\%20-12\%0A\%0Aa\%20\%3D\%20c\%20-\%20a\%0Ae\%20\%3D\%20b\%20\%2B\%20c\%0A\%0Ad\%20\%3D\%20a\%0Aa\%20\%3D\%207\&cumulative=true\&curInstr=0\&heapPrimitives=true\&mode=display\&origin=opt-frontend.js\&py=3\&rawInputLstJSON=\%5B\%5D\&textReferences=false}{https://pythontutor.com}

        \end{reponse}
    \hypertarget{le-mode-programme}{%
\subsection{2 --- Le mode programme}\label{le-mode-programme}}

Le mode programme de Python consiste à écrire une suite d'instructions
dans un fichier et à les faire exécuter par l'interprète Python.

Cette suite d'instruction s'appelle un \textbf{programme} ou encore un
\textbf{code source}.

    \hypertarget{vscodium}{%
\subsubsection{2.1 --- VSCodium}\label{vscodium}}

Pour écrire des programmes, le plus simple est d'utiliser un
\textbf{environnement de développement} (IDE). Cette année, nous allons
essentiellement utiliser \textbf{\texttt{VSCodium}}.

\begin{itemize}
\tightlist
\item
  Après avoir ouvert le logiciel, créer un nouveau fichier
  (\texttt{\textless{}Ctrl\textgreater{}\ +\ \textless{}N\textgreater{}}).
\item
  Sauvegarder le
  (\texttt{\textless{}Ctrl\textgreater{}\ +\ \textless{}S\textgreater{}})
  en donnant un nom qui a pour \emph{extension} \texttt{.py} (par
  exemple \texttt{test.py}).
\item
  Puis une fois votre programme écrit, pour l'exécuter :

  \begin{itemize}
  \tightlist
  \item
    en mode \emph{débogage} : \texttt{\textless{}F5\textgreater{}}
  \item
    en mode \emph{sans débogage} :
    \texttt{\textless{}Ctrl\textgreater{}\ +\ \textless{}F5\textgreater{}}
  \end{itemize}
\end{itemize}

    \hypertarget{affichage}{%
\subsubsection{2.2 --- Affichage}\label{affichage}}

En mode programme, les résultats calculés ne sont plus affichés. Il faut
utiliser pour ceci une instruction d'affichage.

En Python, elle s'appelle \texttt{print}. Ainsi le programme
\texttt{test.py} contenant l'unique ligne \texttt{print(3)} affichera
\texttt{3}.

    L'instruction \texttt{print} admet une expression arbitraire. Elle
commence d'abord par calculer le résultat de cette expression puis
l'affiche à l'écran.

Par exemple l'instruction \texttt{print(1+3)} calcule d'abord
l'expression \texttt{1+3} puis affiche \texttt{4} à l'écran.

    \hypertarget{affichage-des-textes}{%
\subsubsection{2.3 --- Affichage des
textes}\label{affichage-des-textes}}

On peut donner à l'instruction \texttt{print} un message à afficher,
\emph{écrit entre guillemets}.

L'instruction \texttt{print("Bonjour\ tout\ le\ monde")} affiche le
message \texttt{Bonjour\ tout\ le\ monde} à l'écran.
\begin{retenir}
    Le texte écrit entre guillemets est appelé une \textbf{chaîne de
caractères}.

        \end{retenir}\begin{remarque}
    Les guillemets englobants ne sont pas affichés.

Les guillemets peuvent être doubles \texttt{"..."} ou simples
\texttt{\textquotesingle{}...\textquotesingle{}}.

Les guillemets ouverts doivent être impérativement fermés sinon il y
aura une exception (erreur) de type \texttt{syntaxError}.

        \end{remarque}
    Une chaîne de caractère est arbitraire et n'est pas interprétée par
Python.
\begin{eleve}
    Indiquer ce qu'affichent les deux instructions suivantes :

\begin{itemize}
\tightlist
\item
  \texttt{print(1+3)}
\item
  \texttt{print("1+3")}
\end{itemize}
        
        \end{eleve}
    Attention aux opérations avec les chaînes de caractères.

\begin{itemize}
\tightlist
\item
  \texttt{"Hello"\ +\ 2} est une expression \textbf{invalide} car
  l'opération \texttt{+} entre une chaîne et un entier n'a aucun sens.
\item
  \texttt{"Hello"\ +\ "World"} est une expression \textbf{valide} car
  l'opération \texttt{+} entre deux chaînes de caractères les concatènes
  (c'est-à-dire les assemble et produit une chaîne de caractère
  contenant le texte \texttt{"HelloWorld"} (sans espace))
\item
  \texttt{"Hello"\ *\ "World"} est une expression \textbf{invalide} car
  l'opération \texttt{*} n'a pas de sens entres deux chaînes de
  caractères.
\item
  \texttt{"Hello"\ *\ 3} est une expression \textbf{valide} car
  l'opération \texttt{*} entre une chaîne et un entier est définie et
  est équivalente à concaténer \texttt{n} fois la chaîne de caractère.
  Ici l'expression est équivalente à
  \texttt{"Hello"\ +\ "Hello"\ +\ "Hello"} et après interprétation
  produit comme résultat \texttt{"HelloHelloHello"}.
\end{itemize}

    \hypertarget{suxe9quence-dinstructions}{%
\subsubsection{2.4 --- Séquence
d'instructions}\label{suxe9quence-dinstructions}}
\begin{retenir}
    Un programme est généralement constitué de plusieurs instructions.
Chaque instruction est écrite sur une ligne.

        \end{retenir}\begin{eleve}
    Détailler ce que produit l'exécution du programme suivant :

\begin{Shaded}
\begin{Highlighting}[]
\NormalTok{a }\OperatorTok{=} \DecValTok{34}
\NormalTok{b }\OperatorTok{=} \DecValTok{21} \OperatorTok{+}\NormalTok{ a}
\BuiltInTok{print}\NormalTok{(a)}
\BuiltInTok{print}\NormalTok{(b)}
\end{Highlighting}
\end{Shaded}
        
        \end{eleve}\begin{reponse}
    Ce programme affiche deux entiers à l'écran, sur deux lignes :

\begin{Shaded}
\begin{Highlighting}[]
\DecValTok{34}
\DecValTok{55}
\end{Highlighting}
\end{Shaded}

        \end{reponse}\begin{remarque}
    Pour afficher plusieurs expressions sur \textbf{une seule ligne}, il
suffit d'utiliser une seule instruction \texttt{print} et de mettre en
argument les expressions séparées par une virgule :

\begin{itemize}
\tightlist
\item
  \texttt{print(a,b)} affichera les deux éléments sur la même ligne :
  \texttt{34\ 55}
\item
  \texttt{print("la\ somme\ de",\ a,\ "et\ de",\ b,\ "vaut",\ a+b)}
  affichera sur une seule ligne les 6 expressions et donnera alors :
  \texttt{la\ somme\ de\ 34\ et\ de\ 55\ vaut\ 89}
\end{itemize}

        \end{remarque}
    \hypertarget{interagir-avec-lutilisateur}{%
\subsubsection{2.5 --- Interagir avec
l'utilisateur}\label{interagir-avec-lutilisateur}}
\begin{retenir}
    Pour permettre l'interaction du programme avec l'utilisateur, il faut
mettre en place une \textbf{interface}.

        \end{retenir}
    L'interface la plus simple consiste à utiliser l'instruction
\texttt{input} qui permet de récupérer des caractères tapés au clavier
par l'utilisateur.

Cette instruction interrompt l'exécution du programme et attend que la
saisie se termine lorsque la touche
\texttt{\textless{}Entrée\textgreater{}} est appuyée.

        {\scriptsize
    \begin{tcolorbox}[breakable, size=fbox, boxrule=1pt, pad at break*=1mm,colback=cellbackground, colframe=cellborder]
\prompt{In}{incolor}{ }{\boxspacing}
\begin{Verbatim}[commandchars=\\\{\}]
\PY{n}{saisie} \PY{o}{=} \PY{n+nb}{input}\PY{p}{(}\PY{p}{)}
\PY{n+nb}{print}\PY{p}{(}\PY{l+s+s2}{\PYZdq{}}\PY{l+s+s2}{la chaîne de caractère saisie est:}\PY{l+s+s2}{\PYZdq{}}\PY{p}{,} \PY{n}{saisie}\PY{p}{)}
\end{Verbatim}
\end{tcolorbox}
    }

    La chaîne de caractère ainsi récupérée pourra être convertie, si besoin,
en nombre entier en utilisant l'instruction \texttt{int}.

    Par exemple le programme ci-dessous demande l'âge de l'utilisateur. Le
nombre saisi est converti en nombre entier puis un calcul est effectué :

        {\scriptsize
    \begin{tcolorbox}[breakable, size=fbox, boxrule=1pt, pad at break*=1mm,colback=cellbackground, colframe=cellborder]
\prompt{In}{incolor}{ }{\boxspacing}
\begin{Verbatim}[commandchars=\\\{\}]
\PY{n}{texte} \PY{o}{=} \PY{n+nb}{input}\PY{p}{(}\PY{l+s+s2}{\PYZdq{}}\PY{l+s+s2}{ton âge ?}\PY{l+s+s2}{\PYZdq{}}\PY{p}{)}
\PY{n}{age} \PY{o}{=} \PY{n+nb}{int}\PY{p}{(}\PY{n}{texte}\PY{p}{)}
\PY{n+nb}{print}\PY{p}{(}\PY{l+s+s2}{\PYZdq{}}\PY{l+s+s2}{Dans 10 ans, tu auras}\PY{l+s+s2}{\PYZdq{}}\PY{p}{,} \PY{n}{age}\PY{o}{+}\PY{l+m+mi}{10}\PY{p}{)}
\end{Verbatim}
\end{tcolorbox}
    }
\begin{eleve}
    \textbf{Que renvoie} le programme précédent si on supprime la deuxième
ligne?
        
        \end{eleve}
    \hypertarget{un-programme-complet}{%
\subsubsection{2.6 --- Un programme
complet}\label{un-programme-complet}}
\begin{eleve}
    \textbf{Écris} un programme qui demande l'année de naissance à
l'utilisateur puis affiche l'âge qu'il aura en 2048.

Pour chaque ligne du programme, \textbf{représente} l'état de
l'interprète.
        
        \end{eleve}\begin{reponse}
        {\scriptsize
    \begin{tcolorbox}[breakable, size=fbox, boxrule=1pt, pad at break*=1mm,colback=cellbackground, colframe=cellborder]
\prompt{In}{incolor}{ }{\boxspacing}
\begin{Verbatim}[commandchars=\\\{\}]
\PY{c+c1}{\PYZsh{} Calcul de l\PYZsq{}âge en 2048}
\PY{n}{saisie} \PY{o}{=} \PY{n+nb}{input}\PY{p}{(}\PY{l+s+s2}{\PYZdq{}}\PY{l+s+s2}{Entrez votre année de naissance : }\PY{l+s+s2}{\PYZdq{}}\PY{p}{)}
\PY{n}{annee} \PY{o}{=} \PY{n+nb}{int}\PY{p}{(}\PY{n}{saisie}\PY{p}{)}
\PY{n}{age} \PY{o}{=} \PY{l+m+mi}{2048} \PY{o}{\PYZhy{}} \PY{n}{annee}
\PY{n+nb}{print}\PY{p}{(}\PY{l+s+s2}{\PYZdq{}}\PY{l+s+s2}{Vous aurez}\PY{l+s+s2}{\PYZdq{}}\PY{p}{,} \PY{n}{age}\PY{p}{,} \PY{l+s+s2}{\PYZdq{}}\PY{l+s+s2}{ans en 2048.}\PY{l+s+s2}{\PYZdq{}}\PY{p}{)}
\end{Verbatim}
\end{tcolorbox}
    }

        \end{reponse}\begin{reponse}
    \[
\begin{array}{|c|l|l|}
\hline
\text{\bf ligne} & \text{\bf état} & \text{\bf interface}\\
\hline
2 & \overset{\texttt{saisie}}{\fbox{\texttt{"2003"}}} & \text{affichage : \footnotesize\texttt{Entrez votre année de naissance : }}\\
& & \text{saisie : \texttt{2003}}\\
3 & \overset{\texttt{saisie}}{\fbox{\texttt{"2003"}}}\overset{\texttt{annee}}{\fbox{\texttt{2003}}} & \\
4 & \overset{\texttt{saisie}}{\fbox{\texttt{"2003"}}}\overset{\texttt{annee}}{\fbox{\texttt{2003}}}\overset{\texttt{age}}{\fbox{\texttt{45}}} & \\
5 & \overset{\texttt{saisie}}{\fbox{\texttt{"2003"}}}\overset{\texttt{annee}}{\fbox{\texttt{2003}}}\overset{\texttt{age}}{\fbox{\texttt{45}}} & \text{affichage : \texttt{Vous aurez 45 ans en 2048.}}\\
\hline
\end{array}
\]

        \end{reponse}

    % Add a bibliography block to the postdoc
    
    
    
\end{document}
