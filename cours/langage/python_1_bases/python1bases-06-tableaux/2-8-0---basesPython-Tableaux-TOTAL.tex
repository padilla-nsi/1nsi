% template pour la conversion ipynb → pdf
% author : Pascal Padilla
% source : ???
%
% Deux types de modifications : cellule et page
%   * formatage des cellules : par exemple :
%     '..."metadata": {"tags": ["retenir"]},...'
%       * "cacher"
%       * "exo"
%       * "solution"
%       * "reponse"
%       * "proposition"
%       * "remarque"
%       * "exemple"
%       * "retenir"
%
%   * formatage de la page : theme + titre     
%     ...
%     "metadata": {
%       ...,
%       "latex_metadata": {
%       "theme": "machine",
%       "title": "Représentation des données"
%       },
%     Thèmes possibles : 
%       * "interface"
%       * "machine"
%       * "langage"
%       * "algo"
%       * "struct"
%       * "data"
%       * "reseausocial"
%       * "ds"
%       * "iot"
% 
%
\documentclass[a4paper,17pt]{extarticle}


    %unicode lualatex
\usepackage{fontspec}
\usepackage[sfdefault, condensed]{roboto} % police d'écriture plus moderne
\usepackage[french]{babel} % francisation
\usepackage[parfill]{parskip} %suppression indentation




\usepackage{fancyhdr}
\usepackage{multicol}

% figure non flotantes
\usepackage{float}
\let\origfigure\figure
\let\endorigfigure\endfigure
\renewenvironment{figure}[1][2] {
    \expandafter\origfigure\expandafter[H]
} {
    \endorigfigure
}

% mois/année
\usepackage{datetime}
\newdateformat{monthyeardate}{%
  \monthname[\THEMONTH] \THEYEAR}

% couleurs perso
\usepackage[table]{xcolor}
\definecolor{deepblue}{rgb}{0.3,0.3,0.8}
\definecolor{darkblue}{rgb}{0,0,0.3}
\definecolor{deepred}{rgb}{0.6,0,0}
\definecolor{iremred}{RGB}{204,35,50}
\definecolor{deepgreen}{rgb}{0,0.6,0}
\definecolor{backcolor}{rgb}{0.98,0.95,0.95}
\definecolor{grisClair}{rgb}{0.95,0.95,0.95}
\definecolor{orangeamu}{RGB}{250,178,11}
\definecolor{noiramu}{RGB}{35,31,32}
\definecolor{bleuamu}{RGB}{20,118,198}
\definecolor{bleuamudark}{RGB}{15,90,150}
\definecolor{cyanamu}{RGB}{77,198,244}


\usepackage{/home/bouscadilla/Documents/Code/nbconvert/template/latex/pdf_solution/xeboiboites}
%
% exemple
\newbreakabletheorem[
    small box style={fill=deepblue!90,draw=deepblue!15, rounded corners,line width=1pt},%
    big box style={fill=deepblue!5,draw=deepblue!15,thick,rounded corners,line width=1pt},%
    headfont={\color{white}\bfseries}
        ]{exemple}{Exemple}{}%{counterCo}
%
% remarque
\newbreakabletheorem[
    small box style={draw=ansi-green-intense!100,line width=2pt,fill=ansi-green-intense!0,rounded corners,decoration=penciline, decorate},%
	big box style={color=ansi-green-intense!90,fill=ansi-green-intense!10,thick,decoration={penciline},decorate},
    broken edges={draw=ansi-green-intense!90,thick,fill=orange!20!black!5, decoration={random steps, segment length=.5cm,amplitude=1.3mm},decorate},%
    other edges={decoration=penciline,decorate,thick},%
    headfont={\color{ansi-green-intense}\large\scshape\bfseries}
    ]{remarque}{Remarque}{}%{counterCa}
%
% formule (sans titre)
\newboxedequation[%
    big box style={fill=cyanamu!10,draw=cyanamu!100,thick,decoration=penciline,decorate}]%
    {form}
%
% Réponse
\newbreakabletheorem[
    small box style={fill=bleuamu!100, draw=bleuamu!60, line width=1pt,rounded corners,decorate},%
    big box style={fill=bleuamu!10,draw=bleuamu!30,thick,rounded corners,decorate},
    headfont={\color{white}\large\scshape\bfseries}
        ]{reponse}{Correction}{}
%

%
% À retenir
%\newbreakabletheorem[
%    small box style={fill=deepred!100, draw=deepred!80, line width=1pt,rounded corners,decorate},%
%    big box style={fill=deepred!10,draw=deepred!50,thick,rounded corners,decorate},
%    headfont={\color{white}\large\scshape\bfseries}
%        ]{retenir}{À retenir}{}
%
\newboxedequation[%
    big box style={fill=deepred!10,draw=deepred!0,thick,decoration=penciline,decorate}]%
    {retenir}



% astuce
\newspanning[
    image=/home/bouscadilla/Documents/Code/nbconvert/template/latex/pdf_solution/fig-idee,headfont=\bfseries,
    spanning style={very thick,decoration=penciline,decorate}
    ]{astuce}{Astuce}{}
%
% activité

\newcounter{counterCa}
\newbreakabletheorem[
    small box style={draw=orangeamu!100,line width=2pt,fill=orangeamu!100,rounded corners,decoration=penciline, decorate},%
	big box style={color=orangeamu!100,fill=orangeamu!5,thick,decoration={penciline},decorate},
    broken edges={draw=orangeamu!100,thick,fill=orangeamu!100, decoration={random steps, segment length=.5cm,amplitude=1.3mm},decorate},%
    other edges={decoration=penciline,decorate,thick},%
    headfont={\color{white}\large\scshape\bfseries}
    ]{activite}{\adjustimage{height=1cm, valign=m}{/home/bouscadilla/Documents/Code/nbconvert/template/latex/pdf_solution/papier_eleve_investigation.png}%
    Activité}{counterCa}
%   
%   environnement élève
%
\newenvironment{eleve}%
%{\begin{activite}\large\\} % écrire plus gros
{\begin{activite}\color{noiramu}\\[-0.5cm]}
{\end{activite}}

\newenvironment{formule}%
%{\begin{activite}\large\\} % écrire plus gros
{\begin{form}\color{bleuamu}}
{\end{form}}


\usepackage[breakable]{tcolorbox}
    \usepackage{parskip} % Stop auto-indenting (to mimic markdown behaviour)
    
    \usepackage{iftex}
    \ifPDFTeX
    	\usepackage[T1]{fontenc}
    	\usepackage{mathpazo}
    \else
    	\usepackage{fontspec}
    \fi

    % Basic figure setup, for now with no caption control since it's done
    % automatically by Pandoc (which extracts ![](path) syntax from Markdown).
    \usepackage{graphicx}
    % Maintain compatibility with old templates. Remove in nbconvert 6.0
    \let\Oldincludegraphics\includegraphics
    % Ensure that by default, figures have no caption (until we provide a
    % proper Figure object with a Caption API and a way to capture that
    % in the conversion process - todo).
    \usepackage{caption}
    \DeclareCaptionFormat{nocaption}{}
    \captionsetup{format=nocaption,aboveskip=0pt,belowskip=0pt}

    \usepackage[Export]{adjustbox} % Used to constrain images to a maximum size
    \adjustboxset{max size={0.9\linewidth}{0.9\paperheight}}
    \usepackage{float}
    \floatplacement{figure}{H} % forces figures to be placed at the correct location
    \usepackage{xcolor} % Allow colors to be defined
    \usepackage{enumerate} % Needed for markdown enumerations to work
    \usepackage{geometry} % Used to adjust the document margins
    \usepackage{amsmath} % Equations
    \usepackage{amssymb} % Equations
    \usepackage{textcomp} % defines textquotesingle
    % Hack from http://tex.stackexchange.com/a/47451/13684:
    \AtBeginDocument{%
        \def\PYZsq{\textquotesingle}% Upright quotes in Pygmentized code
    }
    \usepackage{upquote} % Upright quotes for verbatim code
    \usepackage{eurosym} % defines \euro
    \usepackage[mathletters]{ucs} % Extended unicode (utf-8) support
    \usepackage{fancyvrb} % verbatim replacement that allows latex

    % The hyperref package gives us a pdf with properly built
    % internal navigation ('pdf bookmarks' for the table of contents,
    % internal cross-reference links, web links for URLs, etc.)
    \usepackage{hyperref}
    % The default LaTeX title has an obnoxious amount of whitespace. By default,
    % titling removes some of it. It also provides customization options.
    \usepackage{titling}
    \usepackage{longtable} % longtable support required by pandoc >1.10
    \usepackage{booktabs}  % table support for pandoc > 1.12.2
    \usepackage[inline]{enumitem} % IRkernel/repr support (it uses the enumerate* environment)
    \usepackage[normalem]{ulem} % ulem is needed to support strikethroughs (\sout)
                                % normalem makes italics be italics, not underlines
    \usepackage{mathrsfs}
    

    
    % Colors for the hyperref package
    \definecolor{urlcolor}{rgb}{0,.145,.698}
    \definecolor{linkcolor}{rgb}{.71,0.21,0.01}
    \definecolor{citecolor}{rgb}{.12,.54,.11}

    % ANSI colors
    \definecolor{ansi-black}{HTML}{3E424D}
    \definecolor{ansi-black-intense}{HTML}{282C36}
    \definecolor{ansi-red}{HTML}{E75C58}
    \definecolor{ansi-red-intense}{HTML}{B22B31}
    \definecolor{ansi-green}{HTML}{00A250}
    \definecolor{ansi-green-intense}{HTML}{007427}
    \definecolor{ansi-yellow}{HTML}{DDB62B}
    \definecolor{ansi-yellow-intense}{HTML}{B27D12}
    \definecolor{ansi-blue}{HTML}{208FFB}
    \definecolor{ansi-blue-intense}{HTML}{0065CA}
    \definecolor{ansi-magenta}{HTML}{D160C4}
    \definecolor{ansi-magenta-intense}{HTML}{A03196}
    \definecolor{ansi-cyan}{HTML}{60C6C8}
    \definecolor{ansi-cyan-intense}{HTML}{258F8F}
    \definecolor{ansi-white}{HTML}{C5C1B4}
    \definecolor{ansi-white-intense}{HTML}{A1A6B2}
    \definecolor{ansi-default-inverse-fg}{HTML}{FFFFFF}
    \definecolor{ansi-default-inverse-bg}{HTML}{000000}

    % commands and environments needed by pandoc snippets
    % extracted from the output of `pandoc -s`
    \providecommand{\tightlist}{%
      \setlength{\itemsep}{0pt}\setlength{\parskip}{0pt}}
    \DefineVerbatimEnvironment{Highlighting}{Verbatim}{commandchars=\\\{\}}
    % Add ',fontsize=\small' for more characters per line
    \newenvironment{Shaded}{}{}
    \newcommand{\KeywordTok}[1]{\textcolor[rgb]{0.00,0.44,0.13}{\textbf{{#1}}}}
    \newcommand{\DataTypeTok}[1]{\textcolor[rgb]{0.56,0.13,0.00}{{#1}}}
    \newcommand{\DecValTok}[1]{\textcolor[rgb]{0.25,0.63,0.44}{{#1}}}
    \newcommand{\BaseNTok}[1]{\textcolor[rgb]{0.25,0.63,0.44}{{#1}}}
    \newcommand{\FloatTok}[1]{\textcolor[rgb]{0.25,0.63,0.44}{{#1}}}
    \newcommand{\CharTok}[1]{\textcolor[rgb]{0.25,0.44,0.63}{{#1}}}
    \newcommand{\StringTok}[1]{\textcolor[rgb]{0.25,0.44,0.63}{{#1}}}
    \newcommand{\CommentTok}[1]{\textcolor[rgb]{0.38,0.63,0.69}{\textit{{#1}}}}
    \newcommand{\OtherTok}[1]{\textcolor[rgb]{0.00,0.44,0.13}{{#1}}}
    \newcommand{\AlertTok}[1]{\textcolor[rgb]{1.00,0.00,0.00}{\textbf{{#1}}}}
    \newcommand{\FunctionTok}[1]{\textcolor[rgb]{0.02,0.16,0.49}{{#1}}}
    \newcommand{\RegionMarkerTok}[1]{{#1}}
    \newcommand{\ErrorTok}[1]{\textcolor[rgb]{1.00,0.00,0.00}{\textbf{{#1}}}}
    \newcommand{\NormalTok}[1]{{#1}}
    
    % Additional commands for more recent versions of Pandoc
    \newcommand{\ConstantTok}[1]{\textcolor[rgb]{0.53,0.00,0.00}{{#1}}}
    \newcommand{\SpecialCharTok}[1]{\textcolor[rgb]{0.25,0.44,0.63}{{#1}}}
    \newcommand{\VerbatimStringTok}[1]{\textcolor[rgb]{0.25,0.44,0.63}{{#1}}}
    \newcommand{\SpecialStringTok}[1]{\textcolor[rgb]{0.73,0.40,0.53}{{#1}}}
    \newcommand{\ImportTok}[1]{{#1}}
    \newcommand{\DocumentationTok}[1]{\textcolor[rgb]{0.73,0.13,0.13}{\textit{{#1}}}}
    \newcommand{\AnnotationTok}[1]{\textcolor[rgb]{0.38,0.63,0.69}{\textbf{\textit{{#1}}}}}
    \newcommand{\CommentVarTok}[1]{\textcolor[rgb]{0.38,0.63,0.69}{\textbf{\textit{{#1}}}}}
    \newcommand{\VariableTok}[1]{\textcolor[rgb]{0.10,0.09,0.49}{{#1}}}
    \newcommand{\ControlFlowTok}[1]{\textcolor[rgb]{0.00,0.44,0.13}{\textbf{{#1}}}}
    \newcommand{\OperatorTok}[1]{\textcolor[rgb]{0.40,0.40,0.40}{{#1}}}
    \newcommand{\BuiltInTok}[1]{{#1}}
    \newcommand{\ExtensionTok}[1]{{#1}}
    \newcommand{\PreprocessorTok}[1]{\textcolor[rgb]{0.74,0.48,0.00}{{#1}}}
    \newcommand{\AttributeTok}[1]{\textcolor[rgb]{0.49,0.56,0.16}{{#1}}}
    \newcommand{\InformationTok}[1]{\textcolor[rgb]{0.38,0.63,0.69}{\textbf{\textit{{#1}}}}}
    \newcommand{\WarningTok}[1]{\textcolor[rgb]{0.38,0.63,0.69}{\textbf{\textit{{#1}}}}}
    
    
    % Define a nice break command that doesn't care if a line doesn't already
    % exist.
    \def\br{\hspace*{\fill} \\* }
    % Math Jax compatibility definitions
    \def\gt{>}
    \def\lt{<}
    \let\Oldtex\TeX
    \let\Oldlatex\LaTeX
    \renewcommand{\TeX}{\textrm{\Oldtex}}
    \renewcommand{\LaTeX}{\textrm{\Oldlatex}}
    % Document parameters
    % Document title
    \title{2-8-0---basesPython-Tableaux-TOTAL}
    
    
    
    
    
% Pygments definitions
\makeatletter
\def\PY@reset{\let\PY@it=\relax \let\PY@bf=\relax%
    \let\PY@ul=\relax \let\PY@tc=\relax%
    \let\PY@bc=\relax \let\PY@ff=\relax}
\def\PY@tok#1{\csname PY@tok@#1\endcsname}
\def\PY@toks#1+{\ifx\relax#1\empty\else%
    \PY@tok{#1}\expandafter\PY@toks\fi}
\def\PY@do#1{\PY@bc{\PY@tc{\PY@ul{%
    \PY@it{\PY@bf{\PY@ff{#1}}}}}}}
\def\PY#1#2{\PY@reset\PY@toks#1+\relax+\PY@do{#2}}

\expandafter\def\csname PY@tok@w\endcsname{\def\PY@tc##1{\textcolor[rgb]{0.73,0.73,0.73}{##1}}}
\expandafter\def\csname PY@tok@c\endcsname{\let\PY@it=\textit\def\PY@tc##1{\textcolor[rgb]{0.25,0.50,0.50}{##1}}}
\expandafter\def\csname PY@tok@cp\endcsname{\def\PY@tc##1{\textcolor[rgb]{0.74,0.48,0.00}{##1}}}
\expandafter\def\csname PY@tok@k\endcsname{\let\PY@bf=\textbf\def\PY@tc##1{\textcolor[rgb]{0.00,0.50,0.00}{##1}}}
\expandafter\def\csname PY@tok@kp\endcsname{\def\PY@tc##1{\textcolor[rgb]{0.00,0.50,0.00}{##1}}}
\expandafter\def\csname PY@tok@kt\endcsname{\def\PY@tc##1{\textcolor[rgb]{0.69,0.00,0.25}{##1}}}
\expandafter\def\csname PY@tok@o\endcsname{\def\PY@tc##1{\textcolor[rgb]{0.40,0.40,0.40}{##1}}}
\expandafter\def\csname PY@tok@ow\endcsname{\let\PY@bf=\textbf\def\PY@tc##1{\textcolor[rgb]{0.67,0.13,1.00}{##1}}}
\expandafter\def\csname PY@tok@nb\endcsname{\def\PY@tc##1{\textcolor[rgb]{0.00,0.50,0.00}{##1}}}
\expandafter\def\csname PY@tok@nf\endcsname{\def\PY@tc##1{\textcolor[rgb]{0.00,0.00,1.00}{##1}}}
\expandafter\def\csname PY@tok@nc\endcsname{\let\PY@bf=\textbf\def\PY@tc##1{\textcolor[rgb]{0.00,0.00,1.00}{##1}}}
\expandafter\def\csname PY@tok@nn\endcsname{\let\PY@bf=\textbf\def\PY@tc##1{\textcolor[rgb]{0.00,0.00,1.00}{##1}}}
\expandafter\def\csname PY@tok@ne\endcsname{\let\PY@bf=\textbf\def\PY@tc##1{\textcolor[rgb]{0.82,0.25,0.23}{##1}}}
\expandafter\def\csname PY@tok@nv\endcsname{\def\PY@tc##1{\textcolor[rgb]{0.10,0.09,0.49}{##1}}}
\expandafter\def\csname PY@tok@no\endcsname{\def\PY@tc##1{\textcolor[rgb]{0.53,0.00,0.00}{##1}}}
\expandafter\def\csname PY@tok@nl\endcsname{\def\PY@tc##1{\textcolor[rgb]{0.63,0.63,0.00}{##1}}}
\expandafter\def\csname PY@tok@ni\endcsname{\let\PY@bf=\textbf\def\PY@tc##1{\textcolor[rgb]{0.60,0.60,0.60}{##1}}}
\expandafter\def\csname PY@tok@na\endcsname{\def\PY@tc##1{\textcolor[rgb]{0.49,0.56,0.16}{##1}}}
\expandafter\def\csname PY@tok@nt\endcsname{\let\PY@bf=\textbf\def\PY@tc##1{\textcolor[rgb]{0.00,0.50,0.00}{##1}}}
\expandafter\def\csname PY@tok@nd\endcsname{\def\PY@tc##1{\textcolor[rgb]{0.67,0.13,1.00}{##1}}}
\expandafter\def\csname PY@tok@s\endcsname{\def\PY@tc##1{\textcolor[rgb]{0.73,0.13,0.13}{##1}}}
\expandafter\def\csname PY@tok@sd\endcsname{\let\PY@it=\textit\def\PY@tc##1{\textcolor[rgb]{0.73,0.13,0.13}{##1}}}
\expandafter\def\csname PY@tok@si\endcsname{\let\PY@bf=\textbf\def\PY@tc##1{\textcolor[rgb]{0.73,0.40,0.53}{##1}}}
\expandafter\def\csname PY@tok@se\endcsname{\let\PY@bf=\textbf\def\PY@tc##1{\textcolor[rgb]{0.73,0.40,0.13}{##1}}}
\expandafter\def\csname PY@tok@sr\endcsname{\def\PY@tc##1{\textcolor[rgb]{0.73,0.40,0.53}{##1}}}
\expandafter\def\csname PY@tok@ss\endcsname{\def\PY@tc##1{\textcolor[rgb]{0.10,0.09,0.49}{##1}}}
\expandafter\def\csname PY@tok@sx\endcsname{\def\PY@tc##1{\textcolor[rgb]{0.00,0.50,0.00}{##1}}}
\expandafter\def\csname PY@tok@m\endcsname{\def\PY@tc##1{\textcolor[rgb]{0.40,0.40,0.40}{##1}}}
\expandafter\def\csname PY@tok@gh\endcsname{\let\PY@bf=\textbf\def\PY@tc##1{\textcolor[rgb]{0.00,0.00,0.50}{##1}}}
\expandafter\def\csname PY@tok@gu\endcsname{\let\PY@bf=\textbf\def\PY@tc##1{\textcolor[rgb]{0.50,0.00,0.50}{##1}}}
\expandafter\def\csname PY@tok@gd\endcsname{\def\PY@tc##1{\textcolor[rgb]{0.63,0.00,0.00}{##1}}}
\expandafter\def\csname PY@tok@gi\endcsname{\def\PY@tc##1{\textcolor[rgb]{0.00,0.63,0.00}{##1}}}
\expandafter\def\csname PY@tok@gr\endcsname{\def\PY@tc##1{\textcolor[rgb]{1.00,0.00,0.00}{##1}}}
\expandafter\def\csname PY@tok@ge\endcsname{\let\PY@it=\textit}
\expandafter\def\csname PY@tok@gs\endcsname{\let\PY@bf=\textbf}
\expandafter\def\csname PY@tok@gp\endcsname{\let\PY@bf=\textbf\def\PY@tc##1{\textcolor[rgb]{0.00,0.00,0.50}{##1}}}
\expandafter\def\csname PY@tok@go\endcsname{\def\PY@tc##1{\textcolor[rgb]{0.53,0.53,0.53}{##1}}}
\expandafter\def\csname PY@tok@gt\endcsname{\def\PY@tc##1{\textcolor[rgb]{0.00,0.27,0.87}{##1}}}
\expandafter\def\csname PY@tok@err\endcsname{\def\PY@bc##1{\setlength{\fboxsep}{0pt}\fcolorbox[rgb]{1.00,0.00,0.00}{1,1,1}{\strut ##1}}}
\expandafter\def\csname PY@tok@kc\endcsname{\let\PY@bf=\textbf\def\PY@tc##1{\textcolor[rgb]{0.00,0.50,0.00}{##1}}}
\expandafter\def\csname PY@tok@kd\endcsname{\let\PY@bf=\textbf\def\PY@tc##1{\textcolor[rgb]{0.00,0.50,0.00}{##1}}}
\expandafter\def\csname PY@tok@kn\endcsname{\let\PY@bf=\textbf\def\PY@tc##1{\textcolor[rgb]{0.00,0.50,0.00}{##1}}}
\expandafter\def\csname PY@tok@kr\endcsname{\let\PY@bf=\textbf\def\PY@tc##1{\textcolor[rgb]{0.00,0.50,0.00}{##1}}}
\expandafter\def\csname PY@tok@bp\endcsname{\def\PY@tc##1{\textcolor[rgb]{0.00,0.50,0.00}{##1}}}
\expandafter\def\csname PY@tok@fm\endcsname{\def\PY@tc##1{\textcolor[rgb]{0.00,0.00,1.00}{##1}}}
\expandafter\def\csname PY@tok@vc\endcsname{\def\PY@tc##1{\textcolor[rgb]{0.10,0.09,0.49}{##1}}}
\expandafter\def\csname PY@tok@vg\endcsname{\def\PY@tc##1{\textcolor[rgb]{0.10,0.09,0.49}{##1}}}
\expandafter\def\csname PY@tok@vi\endcsname{\def\PY@tc##1{\textcolor[rgb]{0.10,0.09,0.49}{##1}}}
\expandafter\def\csname PY@tok@vm\endcsname{\def\PY@tc##1{\textcolor[rgb]{0.10,0.09,0.49}{##1}}}
\expandafter\def\csname PY@tok@sa\endcsname{\def\PY@tc##1{\textcolor[rgb]{0.73,0.13,0.13}{##1}}}
\expandafter\def\csname PY@tok@sb\endcsname{\def\PY@tc##1{\textcolor[rgb]{0.73,0.13,0.13}{##1}}}
\expandafter\def\csname PY@tok@sc\endcsname{\def\PY@tc##1{\textcolor[rgb]{0.73,0.13,0.13}{##1}}}
\expandafter\def\csname PY@tok@dl\endcsname{\def\PY@tc##1{\textcolor[rgb]{0.73,0.13,0.13}{##1}}}
\expandafter\def\csname PY@tok@s2\endcsname{\def\PY@tc##1{\textcolor[rgb]{0.73,0.13,0.13}{##1}}}
\expandafter\def\csname PY@tok@sh\endcsname{\def\PY@tc##1{\textcolor[rgb]{0.73,0.13,0.13}{##1}}}
\expandafter\def\csname PY@tok@s1\endcsname{\def\PY@tc##1{\textcolor[rgb]{0.73,0.13,0.13}{##1}}}
\expandafter\def\csname PY@tok@mb\endcsname{\def\PY@tc##1{\textcolor[rgb]{0.40,0.40,0.40}{##1}}}
\expandafter\def\csname PY@tok@mf\endcsname{\def\PY@tc##1{\textcolor[rgb]{0.40,0.40,0.40}{##1}}}
\expandafter\def\csname PY@tok@mh\endcsname{\def\PY@tc##1{\textcolor[rgb]{0.40,0.40,0.40}{##1}}}
\expandafter\def\csname PY@tok@mi\endcsname{\def\PY@tc##1{\textcolor[rgb]{0.40,0.40,0.40}{##1}}}
\expandafter\def\csname PY@tok@il\endcsname{\def\PY@tc##1{\textcolor[rgb]{0.40,0.40,0.40}{##1}}}
\expandafter\def\csname PY@tok@mo\endcsname{\def\PY@tc##1{\textcolor[rgb]{0.40,0.40,0.40}{##1}}}
\expandafter\def\csname PY@tok@ch\endcsname{\let\PY@it=\textit\def\PY@tc##1{\textcolor[rgb]{0.25,0.50,0.50}{##1}}}
\expandafter\def\csname PY@tok@cm\endcsname{\let\PY@it=\textit\def\PY@tc##1{\textcolor[rgb]{0.25,0.50,0.50}{##1}}}
\expandafter\def\csname PY@tok@cpf\endcsname{\let\PY@it=\textit\def\PY@tc##1{\textcolor[rgb]{0.25,0.50,0.50}{##1}}}
\expandafter\def\csname PY@tok@c1\endcsname{\let\PY@it=\textit\def\PY@tc##1{\textcolor[rgb]{0.25,0.50,0.50}{##1}}}
\expandafter\def\csname PY@tok@cs\endcsname{\let\PY@it=\textit\def\PY@tc##1{\textcolor[rgb]{0.25,0.50,0.50}{##1}}}

\def\PYZbs{\char`\\}
\def\PYZus{\char`\_}
\def\PYZob{\char`\{}
\def\PYZcb{\char`\}}
\def\PYZca{\char`\^}
\def\PYZam{\char`\&}
\def\PYZlt{\char`\<}
\def\PYZgt{\char`\>}
\def\PYZsh{\char`\#}
\def\PYZpc{\char`\%}
\def\PYZdl{\char`\$}
\def\PYZhy{\char`\-}
\def\PYZsq{\char`\'}
\def\PYZdq{\char`\"}
\def\PYZti{\char`\~}
% for compatibility with earlier versions
\def\PYZat{@}
\def\PYZlb{[}
\def\PYZrb{]}
\makeatother


    % For linebreaks inside Verbatim environment from package fancyvrb. 
    \makeatletter
        \newbox\Wrappedcontinuationbox 
        \newbox\Wrappedvisiblespacebox 
        \newcommand*\Wrappedvisiblespace {\textcolor{red}{\textvisiblespace}} 
        \newcommand*\Wrappedcontinuationsymbol {\textcolor{red}{\llap{\tiny$\m@th\hookrightarrow$}}} 
        \newcommand*\Wrappedcontinuationindent {3ex } 
        \newcommand*\Wrappedafterbreak {\kern\Wrappedcontinuationindent\copy\Wrappedcontinuationbox} 
        % Take advantage of the already applied Pygments mark-up to insert 
        % potential linebreaks for TeX processing. 
        %        {, <, #, %, $, ' and ": go to next line. 
        %        _, }, ^, &, >, - and ~: stay at end of broken line. 
        % Use of \textquotesingle for straight quote. 
        \newcommand*\Wrappedbreaksatspecials {% 
            \def\PYGZus{\discretionary{\char`\_}{\Wrappedafterbreak}{\char`\_}}% 
            \def\PYGZob{\discretionary{}{\Wrappedafterbreak\char`\{}{\char`\{}}% 
            \def\PYGZcb{\discretionary{\char`\}}{\Wrappedafterbreak}{\char`\}}}% 
            \def\PYGZca{\discretionary{\char`\^}{\Wrappedafterbreak}{\char`\^}}% 
            \def\PYGZam{\discretionary{\char`\&}{\Wrappedafterbreak}{\char`\&}}% 
            \def\PYGZlt{\discretionary{}{\Wrappedafterbreak\char`\<}{\char`\<}}% 
            \def\PYGZgt{\discretionary{\char`\>}{\Wrappedafterbreak}{\char`\>}}% 
            \def\PYGZsh{\discretionary{}{\Wrappedafterbreak\char`\#}{\char`\#}}% 
            \def\PYGZpc{\discretionary{}{\Wrappedafterbreak\char`\%}{\char`\%}}% 
            \def\PYGZdl{\discretionary{}{\Wrappedafterbreak\char`\$}{\char`\$}}% 
            \def\PYGZhy{\discretionary{\char`\-}{\Wrappedafterbreak}{\char`\-}}% 
            \def\PYGZsq{\discretionary{}{\Wrappedafterbreak\textquotesingle}{\textquotesingle}}% 
            \def\PYGZdq{\discretionary{}{\Wrappedafterbreak\char`\"}{\char`\"}}% 
            \def\PYGZti{\discretionary{\char`\~}{\Wrappedafterbreak}{\char`\~}}% 
        } 
        % Some characters . , ; ? ! / are not pygmentized. 
        % This macro makes them "active" and they will insert potential linebreaks 
        \newcommand*\Wrappedbreaksatpunct {% 
            \lccode`\~`\.\lowercase{\def~}{\discretionary{\hbox{\char`\.}}{\Wrappedafterbreak}{\hbox{\char`\.}}}% 
            \lccode`\~`\,\lowercase{\def~}{\discretionary{\hbox{\char`\,}}{\Wrappedafterbreak}{\hbox{\char`\,}}}% 
            \lccode`\~`\;\lowercase{\def~}{\discretionary{\hbox{\char`\;}}{\Wrappedafterbreak}{\hbox{\char`\;}}}% 
            \lccode`\~`\:\lowercase{\def~}{\discretionary{\hbox{\char`\:}}{\Wrappedafterbreak}{\hbox{\char`\:}}}% 
            \lccode`\~`\?\lowercase{\def~}{\discretionary{\hbox{\char`\?}}{\Wrappedafterbreak}{\hbox{\char`\?}}}% 
            \lccode`\~`\!\lowercase{\def~}{\discretionary{\hbox{\char`\!}}{\Wrappedafterbreak}{\hbox{\char`\!}}}% 
            \lccode`\~`\/\lowercase{\def~}{\discretionary{\hbox{\char`\/}}{\Wrappedafterbreak}{\hbox{\char`\/}}}% 
            \catcode`\.\active
            \catcode`\,\active 
            \catcode`\;\active
            \catcode`\:\active
            \catcode`\?\active
            \catcode`\!\active
            \catcode`\/\active 
            \lccode`\~`\~ 	
        }
    \makeatother

    \let\OriginalVerbatim=\Verbatim
    \makeatletter
    \renewcommand{\Verbatim}[1][1]{%
        %\parskip\z@skip
        \sbox\Wrappedcontinuationbox {\Wrappedcontinuationsymbol}%
        \sbox\Wrappedvisiblespacebox {\FV@SetupFont\Wrappedvisiblespace}%
        \def\FancyVerbFormatLine ##1{\hsize\linewidth
            \vtop{\raggedright\hyphenpenalty\z@\exhyphenpenalty\z@
                \doublehyphendemerits\z@\finalhyphendemerits\z@
                \strut ##1\strut}%
        }%
        % If the linebreak is at a space, the latter will be displayed as visible
        % space at end of first line, and a continuation symbol starts next line.
        % Stretch/shrink are however usually zero for typewriter font.
        \def\FV@Space {%
            \nobreak\hskip\z@ plus\fontdimen3\font minus\fontdimen4\font
            \discretionary{\copy\Wrappedvisiblespacebox}{\Wrappedafterbreak}
            {\kern\fontdimen2\font}%
        }%
        
        % Allow breaks at special characters using \PYG... macros.
        \Wrappedbreaksatspecials
        % Breaks at punctuation characters . , ; ? ! and / need catcode=\active 	
        \OriginalVerbatim[#1,codes*=\Wrappedbreaksatpunct]%
    }
    \makeatother

    % Exact colors from NB
    \definecolor{incolor}{HTML}{303F9F}
    \definecolor{outcolor}{HTML}{D84315}
    \definecolor{cellborder}{HTML}{CFCFCF}
    \definecolor{cellbackground}{HTML}{F7F7F7}
    
    % prompt
    \makeatletter
    \newcommand{\boxspacing}{\kern\kvtcb@left@rule\kern\kvtcb@boxsep}
    \makeatother
    \newcommand{\prompt}[4]{
        \ttfamily\llap{{\color{#2}[#3]:\hspace{3pt}#4}}\vspace{-\baselineskip}
    }
    

    
\setlength\headheight{30pt}
\setcounter{secnumdepth}{0} % Turns off numbering for sections

    % Prevent overflowing lines due to hard-to-break entities
    \sloppy 
    % Setup hyperref package
    \hypersetup{
      breaklinks=true,  % so long urls are correctly broken across lines
      colorlinks=true,
      urlcolor=urlcolor,
      linkcolor=linkcolor,
      citecolor=citecolor,
      }
    % Slightly bigger margins than the latex defaults
    \geometry{a4paper,tmargin=3cm,bmargin=2cm,lmargin=1cm,rmargin=1cm}\fancyhead[L]{Thème à définir}\fancyhead[L]{\adjustimage{height=1cm, valign=m}{/home/bouscadilla/Documents/Code/nbconvert/template/latex/pdf_solution/papier_eleve_ico_langage}\ttfamily\scshape Langage}\fancyhead[C]{\bfseries\MakeUppercase{2-8-0---basesPython-Tableaux-TOTAL}}\fancyhead[C]{\bfseries\MakeUppercase{8 --- Tableaux}}\fancyhead[R]{\monthyeardate\today}

    \fancyfoot[C]{\thepage}
    % #TODO ajouter les pages totales

    \pagestyle{fancy}
    


\begin{document}
    
    \title{8 --- Tableaux}
% \maketitle

    
    

    
    \hypertarget{tableaux}{%
\subsection{8 --- Tableaux}\label{tableaux}}

    \hypertarget{pyramide-des-uxe2ges}{%
\subsubsection{8.1 Pyramide des âges}\label{pyramide-des-uxe2ges}}

    Imaginons un programme qui stoke la pyramide des âges des français
(source : \href{https://www.insee.fr/fr/statistiques/2381472}{pyramide
des âges en 2021, source INSEE}).

    \begin{longtable}[]{@{}lll@{}}
\toprule
Année de naissance & Âge révolu & Ensemble\tabularnewline
\midrule
\endhead
2020 & 0 & 691 754\tabularnewline
2019 & 1 & 714 585\tabularnewline
2018 & 2 & 723 589\tabularnewline
2017 & 3 & 741 105\tabularnewline
2016 & 4 & 761 638\tabularnewline
2015 & 5 & 782 397\tabularnewline
2014 & 6 & 806 296\tabularnewline
2013 & 7 & 813 727\tabularnewline
2012 & 8 & 830 361\tabularnewline
2011 & 9 & 836 626\tabularnewline
2010 & 10 & 857 412\tabularnewline
2009 & 11 & 846 079\tabularnewline
\bottomrule
\end{longtable}

\ldots{}

    Ce programme pourrait avoir deux fonctionnalités :

\begin{enumerate}
\def\labelenumi{\arabic{enumi}.}
\tightlist
\item
  si l'utilisateur saisie un âge alors le programme affiche le nombre de
  français ayant cet âge-là
\item
  si l'utilisateur saisie deux âges, alors le programme affiche le
  nombre de français ayant un âge compris entre ces deux valeurs
\end{enumerate}

    Réaliser un programme qui réalise cela pour les âges compris entre 0
an(inclus) et 5 ans (exclus).

        {\scriptsize
    \begin{tcolorbox}[breakable, size=fbox, boxrule=1pt, pad at break*=1mm,colback=cellbackground, colframe=cellborder]
\prompt{In}{incolor}{1}{\boxspacing}
\begin{Verbatim}[commandchars=\\\{\}]
\PY{c+c1}{\PYZsh{} essayer d\PYZsq{}écrire le programme ici..}
\PY{o}{.}\PY{o}{.}\PY{o}{.}
\end{Verbatim}
\end{tcolorbox}
    }

    On voit bien qu'il faudra utiliser beaucoup de variables. Une variable
pour chaque année. C'est très pénible et heureusement, le langage Python
propose une alternative bien pratique\ldots{}

    \hypertarget{notion-de-tableau}{%
\subsubsection{8.2 Notion de tableau}\label{notion-de-tableau}}

    \hypertarget{cruxe9ation-dun-tableau}{%
\paragraph{Création d'un tableau}\label{cruxe9ation-dun-tableau}}

    Un tableau permet de stocker plusieurs valeurs dans une seule variable
et d'y accéder ensuite facilement. En Python, on construit un tableau en
énumérant ses valeurs entre crochets et séparées par des virgules.

\begin{Shaded}
\begin{Highlighting}[]
\OperatorTok{\textgreater{}\textgreater{}\textgreater{}}\NormalTok{ t }\OperatorTok{=}\NormalTok{ [}\DecValTok{2}\NormalTok{, }\DecValTok{3}\NormalTok{, }\DecValTok{5}\NormalTok{]}
\end{Highlighting}
\end{Shaded}

        {\scriptsize
    \begin{tcolorbox}[breakable, size=fbox, boxrule=1pt, pad at break*=1mm,colback=cellbackground, colframe=cellborder]
\prompt{In}{incolor}{2}{\boxspacing}
\begin{Verbatim}[commandchars=\\\{\}]
\PY{n}{t} \PY{o}{=} \PY{p}{[}\PY{l+m+mi}{2}\PY{p}{,} \PY{l+m+mi}{3}\PY{p}{,} \PY{l+m+mi}{5}\PY{p}{]}
\end{Verbatim}
\end{tcolorbox}
    }

    Ici, on a déclaré une variable \texttt{t} contenant un tableau. Ce
tableau contient trois entiers. Les valeurs sont \textbf{ordonnées}.

\begin{itemize}
\tightlist
\item
  la première valeur est \texttt{2}
\item
  la deuxième \texttt{3}
\item
  la troisième \texttt{5}
\end{itemize}

On peut se représenter un tableau comme des cases consécutives

\[
\begin{array}{|c|c|c|}
\hline
\texttt{2} & \texttt{3} & \texttt{5} \\
\hline
\end{array}
\]

C'est en effet ainsi qu'\textbf{un tableau est organisé dans la mémoire
de l'ordinateur} : ses valeurs y sont rangées consécutivement.

    \hypertarget{indices-dun-tableau}{%
\paragraph{Indices d'un tableau}\label{indices-dun-tableau}}

    Pour accéder à une valeur contenue dans le tableau \texttt{t}, il faut
utiliser la notation \texttt{t{[}i{]}} où \texttt{i} désigne la numéro
de la case à laquelle on veut accéder.

\textbf{Les cases sont numérotées à partir de zéro.}

Ainsi, la valeur contenue dans la première case du tableau est

\begin{Shaded}
\begin{Highlighting}[]
\OperatorTok{\textgreater{}\textgreater{}\textgreater{}}\NormalTok{ t[}\DecValTok{0}\NormalTok{]}
\DecValTok{2}
\end{Highlighting}
\end{Shaded}

        {\scriptsize
    \begin{tcolorbox}[breakable, size=fbox, boxrule=1pt, pad at break*=1mm,colback=cellbackground, colframe=cellborder]
\prompt{In}{incolor}{3}{\boxspacing}
\begin{Verbatim}[commandchars=\\\{\}]
\PY{n}{t}\PY{p}{[}\PY{l+m+mi}{0}\PY{p}{]}
\end{Verbatim}
\end{tcolorbox}
    }

            \begin{tcolorbox}[breakable, size=fbox, boxrule=.5pt, pad at break*=1mm, opacityfill=0]
\prompt{Out}{outcolor}{3}{\boxspacing}
\begin{Verbatim}[commandchars=\\\{\}]
2
\end{Verbatim}
\end{tcolorbox}
        
    Écrire un code qui essaye d'accéder à une case en dehors des limites du
tableau.

        {\scriptsize
    \begin{tcolorbox}[breakable, size=fbox, boxrule=1pt, pad at break*=1mm,colback=cellbackground, colframe=cellborder]
\prompt{In}{incolor}{4}{\boxspacing}
\begin{Verbatim}[commandchars=\\\{\}]
\PY{n}{t}\PY{p}{[}\PY{l+m+mi}{3}\PY{p}{]}
\end{Verbatim}
\end{tcolorbox}
    }

    \begin{Verbatim}[commandchars=\\\{\}]

        ---------------------------------------------------------------------------

        IndexError                                Traceback (most recent call last)

        /tmp/ipykernel\_118370/3611909921.py in <module>
    ----> 1 t[3]
    

        IndexError: list index out of range

    \end{Verbatim}

    On dit que 0 est l'\textbf{indice} de la première case.

La \textbf{taille} d'un tableau est son nombre de cases.

On obtient la taille d'un tableau avec l'opération \texttt{len(t)}.

\begin{Shaded}
\begin{Highlighting}[]
\OperatorTok{\textgreater{}\textgreater{}\textgreater{}} \BuiltInTok{len}\NormalTok{(t)}
\DecValTok{3}
\end{Highlighting}
\end{Shaded}

        {\scriptsize
    \begin{tcolorbox}[breakable, size=fbox, boxrule=1pt, pad at break*=1mm,colback=cellbackground, colframe=cellborder]
\prompt{In}{incolor}{5}{\boxspacing}
\begin{Verbatim}[commandchars=\\\{\}]
\PY{n+nb}{len}\PY{p}{(}\PY{n}{t}\PY{p}{)}
\end{Verbatim}
\end{tcolorbox}
    }

            \begin{tcolorbox}[breakable, size=fbox, boxrule=.5pt, pad at break*=1mm, opacityfill=0]
\prompt{Out}{outcolor}{5}{\boxspacing}
\begin{Verbatim}[commandchars=\\\{\}]
3
\end{Verbatim}
\end{tcolorbox}
        
    Les indices d'un tableau \texttt{t} prennent donc des valeurs comprises
entre \texttt{0} et \texttt{len(t)-1}.

Ce qui donne la représentation mentale suivante :

\[
\begin{array}{ccc}
0 & 1 & 2
\end{array}
\\
\begin{array}{|c|c|c|}
\hline
\texttt{2} & \texttt{3} & \texttt{5} \\
\hline
\end{array}
\]

Mais \textbf{attention}, seules les valeurs sont stockées en mémoire de
l'ordinateur. Les indices n'ont pas besoin d'être matérialisés.

    \hypertarget{retour-sur-la-pyramide-des-uxe2ges}{%
\paragraph{Retour sur la pyramide des
âges}\label{retour-sur-la-pyramide-des-uxe2ges}}

    Créer un tableau \texttt{pda} contenant les 10 premiers effectifs de la
pyramide des âges.

        {\scriptsize
    \begin{tcolorbox}[breakable, size=fbox, boxrule=1pt, pad at break*=1mm,colback=cellbackground, colframe=cellborder]
\prompt{In}{incolor}{6}{\boxspacing}
\begin{Verbatim}[commandchars=\\\{\}]
\PY{n}{pda} \PY{o}{=} \PY{p}{[}\PY{l+m+mi}{691754}\PY{p}{,} \PY{l+m+mi}{714585}\PY{p}{,} \PY{l+m+mi}{723589}\PY{p}{,} \PY{l+m+mi}{741105}\PY{p}{,} \PY{l+m+mi}{761638}\PY{p}{,} \PY{l+m+mi}{782397}\PY{p}{,} \PY{l+m+mi}{806296}\PY{p}{,} \PY{l+m+mi}{813727}\PY{p}{,} \PY{l+m+mi}{830361}\PY{p}{,} \PY{l+m+mi}{836626}\PY{p}{]}
\end{Verbatim}
\end{tcolorbox}
    }

    Dans la case \texttt{i} du tableau \texttt{pda} on retrouve le nombre de
français ayant exactement l'âge \texttt{i}.

    Créer un programme qui demande un âge à l'utilisateur et affiche le
nombre de français ayant cet âge-là.

    \hypertarget{modification-du-contenu-dun-tableau}{%
\paragraph{Modification du contenu d'un
tableau}\label{modification-du-contenu-dun-tableau}}

    Le contenu d'un tableau peut être modifié. Pour cela, on utilise un
\textbf{affectation} exactement comme on le ferait avec une variable.

Ainsi, pour modifier le contenu de la seconde case du tableau \texttt{t}
pour y remplacer la valeur 3 par la valeur 17 :

\begin{Shaded}
\begin{Highlighting}[]
\OperatorTok{\textgreater{}\textgreater{}\textgreater{}}\NormalTok{ t[}\DecValTok{1}\NormalTok{] }\OperatorTok{=} \DecValTok{17}
\end{Highlighting}
\end{Shaded}

        {\scriptsize
    \begin{tcolorbox}[breakable, size=fbox, boxrule=1pt, pad at break*=1mm,colback=cellbackground, colframe=cellborder]
\prompt{In}{incolor}{7}{\boxspacing}
\begin{Verbatim}[commandchars=\\\{\}]
\PY{n}{t}\PY{p}{[}\PY{l+m+mi}{1}\PY{p}{]} \PY{o}{=} \PY{l+m+mi}{17}
\end{Verbatim}
\end{tcolorbox}
    }

    Modifier le tableau \texttt{pda} pour y traduire une naissance.

        {\scriptsize
    \begin{tcolorbox}[breakable, size=fbox, boxrule=1pt, pad at break*=1mm,colback=cellbackground, colframe=cellborder]
\prompt{In}{incolor}{8}{\boxspacing}
\begin{Verbatim}[commandchars=\\\{\}]
\PY{n}{pda}\PY{p}{[}\PY{l+m+mi}{0}\PY{p}{]} \PY{o}{=} \PY{n}{pda}\PY{p}{[}\PY{l+m+mi}{0}\PY{p}{]} \PY{o}{+} \PY{l+m+mi}{1}
\PY{n}{pda}
\end{Verbatim}
\end{tcolorbox}
    }

            \begin{tcolorbox}[breakable, size=fbox, boxrule=.5pt, pad at break*=1mm, opacityfill=0]
\prompt{Out}{outcolor}{8}{\boxspacing}
\begin{Verbatim}[commandchars=\\\{\}]
[691755,
 714585,
 723589,
 741105,
 761638,
 782397,
 806296,
 813727,
 830361,
 836626]
\end{Verbatim}
\end{tcolorbox}
        
    \hypertarget{parcours-dun-tableau}{%
\subsubsection{8.3 --- Parcours d'un
tableau}\label{parcours-dun-tableau}}

    En utilisant le tableau \texttt{pda} de la partie précédente, proposer
un programme permettant d'obtenir le nombre de français ayant entre 1 et
8 ans.

        {\scriptsize
    \begin{tcolorbox}[breakable, size=fbox, boxrule=1pt, pad at break*=1mm,colback=cellbackground, colframe=cellborder]
\prompt{In}{incolor}{9}{\boxspacing}
\begin{Verbatim}[commandchars=\\\{\}]
\PY{n}{pda} \PY{o}{=} \PY{p}{[}\PY{l+m+mi}{691754}\PY{p}{,} \PY{l+m+mi}{714585}\PY{p}{,} \PY{l+m+mi}{723589}\PY{p}{,} \PY{l+m+mi}{741105}\PY{p}{,} \PY{l+m+mi}{761638}\PY{p}{,} \PY{l+m+mi}{782397}\PY{p}{,} \PY{l+m+mi}{806296}\PY{p}{,} \PY{l+m+mi}{813727}\PY{p}{,} \PY{l+m+mi}{830361}\PY{p}{,} \PY{l+m+mi}{836626}\PY{p}{]}
\PY{n}{pda}\PY{p}{[}\PY{l+m+mi}{1}\PY{p}{]} \PY{o}{+} \PY{n}{pda}\PY{p}{[}\PY{l+m+mi}{2}\PY{p}{]} \PY{o}{+} \PY{n}{pda}\PY{p}{[}\PY{l+m+mi}{3}\PY{p}{]} \PY{o}{+} \PY{n}{pda}\PY{p}{[}\PY{l+m+mi}{4}\PY{p}{]} \PY{o}{+} \PY{n}{pda}\PY{p}{[}\PY{l+m+mi}{5}\PY{p}{]} \PY{o}{+} \PY{n}{pda}\PY{p}{[}\PY{l+m+mi}{6}\PY{p}{]} \PY{o}{+} \PY{n}{pda}\PY{p}{[}\PY{l+m+mi}{7}\PY{p}{]} \PY{o}{+} \PY{n}{pda}\PY{p}{[}\PY{l+m+mi}{8}\PY{p}{]}
\end{Verbatim}
\end{tcolorbox}
    }

            \begin{tcolorbox}[breakable, size=fbox, boxrule=.5pt, pad at break*=1mm, opacityfill=0]
\prompt{Out}{outcolor}{9}{\boxspacing}
\begin{Verbatim}[commandchars=\\\{\}]
6173698
\end{Verbatim}
\end{tcolorbox}
        
    Comment s'y prendre quand le tableau est beaucoup plus long ?

Une meilleure solution consiste à utiliser une \textbf{boucle} pour
parcourir les cases du tableau concernées en \textbf{accumulant} le
nombre total dans une variable.

On commence par initialiser la variable \texttt{n} à \texttt{0} :

\begin{Shaded}
\begin{Highlighting}[]
\NormalTok{n }\OperatorTok{=} \DecValTok{0}
\end{Highlighting}
\end{Shaded}

    Puis on effectue une boucle \texttt{for} donnant successivement à la
variable \texttt{i} toutes les valeurs entre 1 et 8, et on ajoute à
chaque fois la valeur \texttt{pda{[}i{]}} à la variable \texttt{n} :

\begin{Shaded}
\begin{Highlighting}[]
\ControlFlowTok{for}\NormalTok{ i }\KeywordTok{in} \BuiltInTok{range}\NormalTok{(}\DecValTok{1}\NormalTok{, }\DecValTok{9}\NormalTok{):}
\NormalTok{    n }\OperatorTok{+=}\NormalTok{ pda[i]}
\end{Highlighting}
\end{Shaded}

On peut vérifier qu'après la boucle la variable \texttt{n} contient bien
la valeur précédente.

        {\scriptsize
    \begin{tcolorbox}[breakable, size=fbox, boxrule=1pt, pad at break*=1mm,colback=cellbackground, colframe=cellborder]
\prompt{In}{incolor}{10}{\boxspacing}
\begin{Verbatim}[commandchars=\\\{\}]
\PY{n}{n} \PY{o}{=} \PY{l+m+mi}{0}
\PY{k}{for} \PY{n}{i} \PY{o+ow}{in} \PY{n+nb}{range} \PY{p}{(}\PY{l+m+mi}{1}\PY{p}{,} \PY{l+m+mi}{9}\PY{p}{)}\PY{p}{:}
    \PY{n}{n} \PY{o}{+}\PY{o}{=} \PY{n}{pda}\PY{p}{[}\PY{n}{i}\PY{p}{]}

\PY{n}{n}
\end{Verbatim}
\end{tcolorbox}
    }

            \begin{tcolorbox}[breakable, size=fbox, boxrule=.5pt, pad at break*=1mm, opacityfill=0]
\prompt{Out}{outcolor}{10}{\boxspacing}
\begin{Verbatim}[commandchars=\\\{\}]
6173698
\end{Verbatim}
\end{tcolorbox}
        
    Écrire un programme qui demande à l'utilisateur

\begin{itemize}
\tightlist
\item
  un âge minimal
\item
  un âge maximal
\end{itemize}

et qui renvoie le nombre de français dont l'âge est compris entre le
minimal (inclu) et le maximal (exclu).

Par exemple :

\begin{verbatim}
    >>> âge minimum (inclu) : 1
    >>> âge maximum (exclu) : 9
    il y a 6173698 personnes qui ont entre 1 et 9 ans
\end{verbatim}

    

    Pour parcourir \textbf{toutes les cases} d'un tableau \texttt{t}, on
peut utiliser une boucle \texttt{for} allant de 0 inclus à
\texttt{len(t)} exclu.

    Proposer un programme qui affiche le nombre total de français contenus
dans votre tableau \texttt{pda}.

        {\scriptsize
    \begin{tcolorbox}[breakable, size=fbox, boxrule=1pt, pad at break*=1mm,colback=cellbackground, colframe=cellborder]
\prompt{In}{incolor}{11}{\boxspacing}
\begin{Verbatim}[commandchars=\\\{\}]
\PY{n}{population\PYZus{}totale} \PY{o}{=} \PY{l+m+mi}{0}
\PY{k}{for} \PY{n}{i} \PY{o+ow}{in} \PY{n+nb}{range} \PY{p}{(}\PY{l+m+mi}{0}\PY{p}{,} \PY{n+nb}{len}\PY{p}{(}\PY{n}{pda}\PY{p}{)}\PY{p}{)}\PY{p}{:}
    \PY{n}{population\PYZus{}totale} \PY{o}{+}\PY{o}{=} \PY{n}{pda}\PY{p}{[}\PY{n}{i}\PY{p}{]}

\PY{n+nb}{print}\PY{p}{(}\PY{n}{population\PYZus{}totale}\PY{p}{)}
\end{Verbatim}
\end{tcolorbox}
    }

    \begin{Verbatim}[commandchars=\\\{\}]
7702078
    \end{Verbatim}

    \hypertarget{construire-de-grands-tableaux}{%
\subsubsection{8.4 --- Construire de grands
tableaux}\label{construire-de-grands-tableaux}}

    Pour construire des tableaux d'une taille arbitraire, on peut utiliser
la syntaxe :

\begin{Shaded}
\begin{Highlighting}[]
\OperatorTok{\textgreater{}\textgreater{}\textgreater{}}\NormalTok{ t }\OperatorTok{=}\NormalTok{ [}\DecValTok{0}\NormalTok{] }\OperatorTok{*} \DecValTok{1000}
\end{Highlighting}
\end{Shaded}

On donne la taille du tableau après le symbole \texttt{*} (ici
\texttt{1000}) et entre crochet une valeur qui sera donnée à toutes les
cases du tableau (ici \texttt{0}).

On obtient donc un tableau de taille 1000 dont toutes les cases
contiennent pour l'instant la valeur \texttt{0}.

    Créer un tel tableau et vérifier sa taille.

        {\scriptsize
    \begin{tcolorbox}[breakable, size=fbox, boxrule=1pt, pad at break*=1mm,colback=cellbackground, colframe=cellborder]
\prompt{In}{incolor}{12}{\boxspacing}
\begin{Verbatim}[commandchars=\\\{\}]
\PY{n}{t} \PY{o}{=} \PY{p}{[}\PY{l+m+mi}{0}\PY{p}{]} \PY{o}{*} \PY{l+m+mi}{1000}
\PY{n+nb}{len}\PY{p}{(}\PY{n}{t}\PY{p}{)}
\end{Verbatim}
\end{tcolorbox}
    }

            \begin{tcolorbox}[breakable, size=fbox, boxrule=.5pt, pad at break*=1mm, opacityfill=0]
\prompt{Out}{outcolor}{12}{\boxspacing}
\begin{Verbatim}[commandchars=\\\{\}]
1000
\end{Verbatim}
\end{tcolorbox}
        
    Une fois le tableau créé, on peut le \textbf{remplir} avec des valeurs
de son choix.

    Remplir le tableau \texttt{t} avec la valeur des carrés des 1000
premiers entiers.

        {\scriptsize
    \begin{tcolorbox}[breakable, size=fbox, boxrule=1pt, pad at break*=1mm,colback=cellbackground, colframe=cellborder]
\prompt{In}{incolor}{13}{\boxspacing}
\begin{Verbatim}[commandchars=\\\{\}]
\PY{k}{for} \PY{n}{i} \PY{o+ow}{in} \PY{n+nb}{range}\PY{p}{(}\PY{l+m+mi}{0}\PY{p}{,} \PY{l+m+mi}{1000}\PY{p}{)}\PY{p}{:}
    \PY{n}{t}\PY{p}{[}\PY{n}{i}\PY{p}{]} \PY{o}{=} \PY{n}{i} \PY{o}{*} \PY{n}{i}
\end{Verbatim}
\end{tcolorbox}
    }

    \textbf{Attention}, l'opération \texttt{{[}0{]}*1000} n'implique par une
multiplication (même si on voit le symbole \texttt{*}). Il s'agit là
d'une opération spécifique aux tableaux.

    Il est possible de \emph{concaténer} deux tableaux, c'est-à-dire de
construire un nouveau tableau contenant bout à bout les éléments des
deux tableaux. On le fait avec l'opération \texttt{+} :

\begin{Shaded}
\begin{Highlighting}[]
\OperatorTok{\textgreater{}\textgreater{}\textgreater{}}\NormalTok{ [}\DecValTok{8}\NormalTok{, }\DecValTok{13}\NormalTok{, }\DecValTok{21}\NormalTok{] }\OperatorTok{+}\NormalTok{ [}\DecValTok{34}\NormalTok{, }\DecValTok{55}\NormalTok{]}
\NormalTok{[}\DecValTok{8}\NormalTok{, }\DecValTok{13}\NormalTok{, }\DecValTok{21}\NormalTok{, }\DecValTok{34}\NormalTok{, }\DecValTok{55}\NormalTok{]}
\end{Highlighting}
\end{Shaded}

    \hypertarget{tableaux-et-chauxeenes-de-caractuxe8res}{%
\paragraph{Tableaux et chaînes de
caractères}\label{tableaux-et-chauxeenes-de-caractuxe8res}}

    Les chaînes de caractères offrent certaines ressemblance avec les
tableaux.

\begin{itemize}
\tightlist
\item
  taille d'une chaîne de caractère : \texttt{len(ch)}
\item
  accès au i-ème caractère d'une chaîne : \texttt{ch{[}i{]}}
\item
  concaténation : \texttt{+}
\item
  répétition d'un valeur : \texttt{*}
\end{itemize}

    \begin{enumerate}
\def\labelenumi{\arabic{enumi}.}
\tightlist
\item
  Créer la variable \texttt{ch} initialisée à la chaîne de caractères
  \texttt{"bonjour"}.
\item
  Afficher la longueur de la chaîne.
\item
  Vérifier que le 3ème caractère de la chaîne est bien le \texttt{n}
\item
  Concaténer cette chaîne avec la deuxième chaîne
  \texttt{"\ le\ monde\ !"}
\end{enumerate}

        {\scriptsize
    \begin{tcolorbox}[breakable, size=fbox, boxrule=1pt, pad at break*=1mm,colback=cellbackground, colframe=cellborder]
\prompt{In}{incolor}{14}{\boxspacing}
\begin{Verbatim}[commandchars=\\\{\}]
\PY{n}{ch} \PY{o}{=} \PY{l+s+s2}{\PYZdq{}}\PY{l+s+s2}{bonjour}\PY{l+s+s2}{\PYZdq{}}
\PY{n+nb}{print}\PY{p}{(}\PY{n+nb}{len}\PY{p}{(}\PY{n}{ch}\PY{p}{)}\PY{p}{)}
\PY{n+nb}{print}\PY{p}{(}\PY{n}{ch}\PY{p}{[}\PY{l+m+mi}{2}\PY{p}{]}\PY{p}{)}
\PY{n}{ch} \PY{o}{=} \PY{n}{ch} \PY{o}{+} \PY{l+s+s2}{\PYZdq{}}\PY{l+s+s2}{ le monde !}\PY{l+s+s2}{\PYZdq{}}
\PY{n+nb}{print}\PY{p}{(}\PY{n}{ch}\PY{p}{)}
\end{Verbatim}
\end{tcolorbox}
    }

    \begin{Verbatim}[commandchars=\\\{\}]
7
n
bonjour le monde !
    \end{Verbatim}

    \textbf{Attention}, contrairement aux tableaux, les caractères d'une
chaîne ne peuvent pas être modifiés. On dit qu'une chaîne est immuable.

        {\scriptsize
    \begin{tcolorbox}[breakable, size=fbox, boxrule=1pt, pad at break*=1mm,colback=cellbackground, colframe=cellborder]
\prompt{In}{incolor}{15}{\boxspacing}
\begin{Verbatim}[commandchars=\\\{\}]
\PY{n}{ch}\PY{p}{[}\PY{l+m+mi}{2}\PY{p}{]} \PY{o}{=} \PY{l+s+s2}{\PYZdq{}}\PY{l+s+s2}{Z}\PY{l+s+s2}{\PYZdq{}}
\end{Verbatim}
\end{tcolorbox}
    }

    \begin{Verbatim}[commandchars=\\\{\}]

        ---------------------------------------------------------------------------

        TypeError                                 Traceback (most recent call last)

        /tmp/ipykernel\_118370/1875237351.py in <module>
    ----> 1 ch[2] = "Z"
    

        TypeError: 'str' object does not support item assignment

    \end{Verbatim}

    \hypertarget{tableaux-et-variables}{%
\subsubsection{8.5 --- Tableaux et
variables}\label{tableaux-et-variables}}

    \hypertarget{activituxe9}{%
\subsubsection{Activité}\label{activituxe9}}

Le bloc de code ci-dessous fait appel à un outil \textbf{indispensable}
: \href{https://pythontutor.com}{Python Tutor}. Je vous invite à
l'utiliser en autonomie chaque fois que vous voulez comprendre ou
visualiser un algorithme, une erreur, etc.

\begin{enumerate}
\def\labelenumi{\arabic{enumi}.}
\tightlist
\item
  Exécuter le bloc de code
\item
  Visualiser l'exécution du code pas à pas (en cliquant sur Forward
  \textgreater{}) et répondre aux questions suivantes :

  \begin{enumerate}
  \def\labelenumii{\arabic{enumii}.}
  \tightlist
  \item
    Comment s'appelle l'instruction de la ligne 1 ?
  \item
    Est-ce que l'instruction de la ligne 4 modifie le contenu de la
    variable \texttt{x} ?
  \item
    Même question pour l'instruction de la ligne 7.
  \item
    Proposer une explication\ldots{}
  \end{enumerate}
\end{enumerate}

        {\scriptsize
    \begin{tcolorbox}[breakable, size=fbox, boxrule=1pt, pad at break*=1mm,colback=cellbackground, colframe=cellborder]
\prompt{In}{incolor}{ }{\boxspacing}
\begin{Verbatim}[commandchars=\\\{\}]
\PY{k+kn}{from} \PY{n+nn}{tutor} \PY{k+kn}{import} \PY{n}{tutor}

\PY{n}{x} \PY{o}{=} \PY{l+m+mi}{1}
\PY{n}{x} \PY{o}{=} \PY{l+m+mi}{2}
\PY{n}{y} \PY{o}{=} \PY{n}{x}
\PY{n}{y} \PY{o}{=} \PY{l+m+mi}{3}
\PY{n}{t} \PY{o}{=} \PY{p}{[}\PY{l+m+mi}{1}\PY{p}{,} \PY{l+m+mi}{2}\PY{p}{,} \PY{l+m+mi}{3}\PY{p}{]}
\PY{n}{u} \PY{o}{=} \PY{n}{t}
\PY{n}{u}\PY{p}{[}\PY{l+m+mi}{2}\PY{p}{]} \PY{o}{=} \PY{l+m+mi}{7}
\PY{n+nb}{print}\PY{p}{(}\PY{n}{t}\PY{p}{[}\PY{l+m+mi}{2}\PY{p}{]}\PY{p}{)}

\PY{n}{tutor}\PY{p}{(}\PY{p}{)}
\end{Verbatim}
\end{tcolorbox}
    }

    \ldots{} bloc de réponse \ldots{}

    Petite explication de ce phénomène d'\textbf{effet de bords} lié aux
tableaux.

En réalité, ce n'est pas un tableau qui est stocké en mémoire, mais
l'\emph{adresse mémoire} du tableau (c'est pour cela que la
représentation adoptée est celle d'une flèche : l'adresse mémoire
désigne l'emplacement où se trouve le tableau).

Du coup, les deux variables \texttt{t} et \texttt{u} désignent
\textbf{le même} tableau ! Et donc toute modification du contenu du
tableau \texttt{u} sera visible dans le tableau \texttt{t}.

    \hypertarget{tableaux-et-fonctions}{%
\subsubsection{8-6 --- Tableaux et
fonctions}\label{tableaux-et-fonctions}}

    \hypertarget{tableau-en-argument}{%
\paragraph{Tableau en argument}\label{tableau-en-argument}}

    Dans le même ordre d'idée, nous allons passer un tableau en argument
d'une fonction dans l'exemple ci-dessous.

    \hypertarget{activituxe9}{%
\subsubsection{Activité}\label{activituxe9}}

\begin{enumerate}
\def\labelenumi{\arabic{enumi}.}
\tightlist
\item
  Observer le code ci-dessous et répondre aux questions suivantes :

  \begin{enumerate}
  \def\labelenumii{\arabic{enumii}.}
  \tightlist
  \item
    À quel type de valeur correspondent les variables \texttt{x} et
    \texttt{t} lorsqu'elles sont initialisées ?
  \item
    Quelles sont les valeurs initiales des variables \texttt{x} et
    \texttt{t} ?
  \item
    Que fait la fonction \texttt{f} ?
  \end{enumerate}
\item
  Exécuter le bloc de code ci-dessous.
\item
  Visualiser l'exécution du code pas à pas et répondre aux questions
  suivantes :

  \begin{enumerate}
  \def\labelenumii{\arabic{enumii}.}
  \tightlist
  \item
    Quelles sont les valeurs des variables \texttt{x} et \texttt{t}
    \textbf{après} l'exécution complète de l'instruction ligne 8 ?
  \item
    Expliquer pourquoi la variable \texttt{x} est inchangée alors que
    \texttt{t} a été modifiée ?
  \end{enumerate}
\end{enumerate}

        {\scriptsize
    \begin{tcolorbox}[breakable, size=fbox, boxrule=1pt, pad at break*=1mm,colback=cellbackground, colframe=cellborder]
\prompt{In}{incolor}{ }{\boxspacing}
\begin{Verbatim}[commandchars=\\\{\}]
\PY{k+kn}{from} \PY{n+nn}{tutor} \PY{k+kn}{import} \PY{n}{tutor}

\PY{n}{x} \PY{o}{=} \PY{l+m+mi}{1}
\PY{n}{t} \PY{o}{=} \PY{p}{[}\PY{l+m+mi}{1}\PY{p}{,} \PY{l+m+mi}{2}\PY{p}{,} \PY{l+m+mi}{3}\PY{p}{]}

\PY{k}{def} \PY{n+nf}{f}\PY{p}{(}\PY{n}{a}\PY{p}{,} \PY{n}{b}\PY{p}{)}\PY{p}{:}
    \PY{n}{a} \PY{o}{=} \PY{n}{a} \PY{o}{+} \PY{l+m+mi}{1}
    \PY{n}{b}\PY{p}{[}\PY{n}{a}\PY{p}{]} \PY{o}{=} \PY{l+m+mi}{7}
    
\PY{n}{f}\PY{p}{(}\PY{n}{x}\PY{p}{,} \PY{n}{t}\PY{p}{)}

\PY{n}{tutor}\PY{p}{(}\PY{p}{)}
\end{Verbatim}
\end{tcolorbox}
    }

    \ldots{} bloc de réponse \ldots{}

    Comme on le constate, le contenu du tableau \texttt{t} a été modifié par
la fonction, mais pas le contenu de la variable \texttt{x} : \textbf{une
fonction peut modifier le contenu d'un tableau qui lui est passé en
argument}.

Nous écrirons par la suite de nombreuses fonctions opérant ainsi sur les
tableaux. Par exemple en créant des fonctions qui trient le contenu d'un
tableau.

    \hypertarget{tableau-renvoyuxe9-par-une-fonction}{%
\paragraph{Tableau renvoyé par une
fonction}\label{tableau-renvoyuxe9-par-une-fonction}}

    Une fonction peut aussi renvoyer un tableau comme résultat.

Pour bien comprendre ce qu'il se passe, illustrons cela par un autre
exemple.

    \hypertarget{activituxe9}{%
\subsubsection{Activité}\label{activituxe9}}

\begin{enumerate}
\def\labelenumi{\arabic{enumi}.}
\tightlist
\item
  Lire le code ci-dessous et répondre aux questions suivantes :

  \begin{enumerate}
  \def\labelenumii{\arabic{enumii}.}
  \tightlist
  \item
    Que fait la fonction \texttt{carres(n)} lorsqu'on lui passe un
    nombre entier \texttt{n} en argument ?
  \item
    Lorsqu'on exécute l'instruction \texttt{c\ =\ carres(4)}, la
    fonction s'exécute. En arrivant à la ligne \texttt{return}, combien
    y a-t-il de variables locales à la fonction en mémoire ?
  \item
    Une fois que la fonction s'est complètement exécutée, combien
    reste-t-il de variables locales à la fonction en mémoire ?
  \end{enumerate}
\item
  Exécuter le bloc de code ci-dessous.
\item
  Visualiser l'exécution du code pas à pas et répondre aux questions
  suivantes :

  \begin{enumerate}
  \def\labelenumii{\arabic{enumii}.}
  \tightlist
  \item
    Est-ce que la variable \texttt{t} a survécu à l'appel à la fonction
    ?
  \item
    Est-ce que le contenu de la variable \texttt{t} a survécu ?
  \end{enumerate}
\end{enumerate}

        {\scriptsize
    \begin{tcolorbox}[breakable, size=fbox, boxrule=1pt, pad at break*=1mm,colback=cellbackground, colframe=cellborder]
\prompt{In}{incolor}{ }{\boxspacing}
\begin{Verbatim}[commandchars=\\\{\}]
\PY{k+kn}{from} \PY{n+nn}{tutor} \PY{k+kn}{import} \PY{n}{tutor}

\PY{k}{def} \PY{n+nf}{carres}\PY{p}{(}\PY{n}{n}\PY{p}{)}\PY{p}{:}
    \PY{n}{t} \PY{o}{=} \PY{p}{[}\PY{l+m+mi}{0}\PY{p}{]} \PY{o}{*} \PY{n}{n}
    \PY{k}{for} \PY{n}{i} \PY{o+ow}{in} \PY{n+nb}{range}\PY{p}{(}\PY{n}{n}\PY{p}{)}\PY{p}{:}
        \PY{n}{t}\PY{p}{[}\PY{n}{i}\PY{p}{]} \PY{o}{=} \PY{n}{i} \PY{o}{*} \PY{n}{i}
    \PY{k}{return} \PY{n}{t}

\PY{n}{c} \PY{o}{=} \PY{n}{carres}\PY{p}{(}\PY{l+m+mi}{4}\PY{p}{)}

\PY{n}{tutor}\PY{p}{(}\PY{p}{)}
\end{Verbatim}
\end{tcolorbox}
    }


    % Add a bibliography block to the postdoc
    
    
    
\end{document}
