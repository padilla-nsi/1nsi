\documentclass[a4paper,17pt]{extarticle}


    \usepackage[sfdefault, condensed]{roboto} % police d'écriture plus moderne
\usepackage[french]{babel} % francisation
\usepackage[parfill]{parskip} %suppression indentation

\usepackage{fancyhdr}
\usepackage{multicol}

% figure non flotantes
\usepackage{float}
\let\origfigure\figure
\let\endorigfigure\endfigure
\renewenvironment{figure}[1][2] {
    \expandafter\origfigure\expandafter[H]
} {
    \endorigfigure
}

% mois/année
\usepackage{datetime}
\newdateformat{monthyeardate}{%
  \monthname[\THEMONTH] \THEYEAR}

% couleurs perso
\usepackage[table]{xcolor}
\definecolor{deepblue}{rgb}{0.3,0.3,0.8}
\definecolor{darkblue}{rgb}{0,0,0.3}
\definecolor{deepred}{rgb}{0.6,0,0}
\definecolor{iremred}{RGB}{204,35,50}
\definecolor{deepgreen}{rgb}{0,0.6,0}
\definecolor{backcolor}{rgb}{0.98,0.95,0.95}
\definecolor{grisClair}{rgb}{0.95,0.95,0.95}
\definecolor{orangeamu}{RGB}{250,178,11}
\definecolor{noiramu}{RGB}{35,31,32}
\definecolor{bleuamu}{RGB}{20,118,198}
\definecolor{bleuamudark}{RGB}{15,90,150}
\definecolor{cyanamu}{RGB}{77,198,244}


\usepackage{/home/bouscadilla/Documents/Code/nbconvert/template/latex/pdf_solution/xeboiboites}
%
% exemple
\newbreakabletheorem[
    small box style={fill=deepblue!90,draw=deepblue!15, rounded corners,line width=1pt},%
    big box style={fill=deepblue!5,draw=deepblue!15,thick,rounded corners,line width=1pt},%
    headfont={\color{white}\bfseries}
        ]{exemple}{Exemple}{}%{counterCo}
%
% remarque
\newbreakabletheorem[
    small box style={draw=ansi-green-intense!100,line width=2pt,fill=ansi-green-intense!0,rounded corners,decoration=penciline, decorate},%
	big box style={color=ansi-green-intense!90,fill=ansi-green-intense!10,thick,decoration={penciline},decorate},
    broken edges={draw=ansi-green-intense!90,thick,fill=orange!20!black!5, decoration={random steps, segment length=.5cm,amplitude=1.3mm},decorate},%
    other edges={decoration=penciline,decorate,thick},%
    headfont={\color{ansi-green-intense}\large\scshape\bfseries}
    ]{remarque}{Remarque}{}%{counterCa}
%
% formule (sans titre)
\newboxedequation[%
    big box style={fill=cyanamu!10,draw=cyanamu!100,thick,decoration=penciline,decorate}]%
    {form}
%
% Réponse
\newbreakabletheorem[
    small box style={fill=bleuamu!100, draw=bleuamu!60, line width=1pt,rounded corners,decorate},%
    big box style={fill=bleuamu!10,draw=bleuamu!30,thick,rounded corners,decorate},
    headfont={\color{white}\large\scshape\bfseries}
        ]{reponse}{Correction}{}
%

%
% À retenir
%\newbreakabletheorem[
%    small box style={fill=deepred!100, draw=deepred!80, line width=1pt,rounded corners,decorate},%
%    big box style={fill=deepred!10,draw=deepred!50,thick,rounded corners,decorate},
%    headfont={\color{white}\large\scshape\bfseries}
%        ]{retenir}{À retenir}{}
%
\newboxedequation[%
    big box style={fill=deepred!10,draw=deepred!0,thick,decoration=penciline,decorate}]%
    {retenir}



% astuce
\newspanning[
    image=/home/bouscadilla/Documents/Code/nbconvert/template/latex/pdf_solution/fig-idee,headfont=\bfseries,
    spanning style={very thick,decoration=penciline,decorate}
    ]{astuce}{Astuce}{}
%
% activité

\newcounter{counterCa}
\newbreakabletheorem[
    small box style={draw=orangeamu!100,line width=2pt,fill=orangeamu!100,rounded corners,decoration=penciline, decorate},%
	big box style={color=orangeamu!100,fill=orangeamu!5,thick,decoration={penciline},decorate},
    broken edges={draw=orangeamu!100,thick,fill=orangeamu!100, decoration={random steps, segment length=.5cm,amplitude=1.3mm},decorate},%
    other edges={decoration=penciline,decorate,thick},%
    headfont={\color{white}\large\scshape\bfseries}
    ]{activite}{\adjustimage{height=1cm, valign=m}{/home/bouscadilla/Documents/Code/nbconvert/template/latex/pdf_solution/papier_eleve_investigation.png}%
    Activité}{counterCa}
%   
%   environnement élève
%
\newenvironment{eleve}%
%{\begin{activite}\large\\} % écrire plus gros
{\begin{activite}\color{noiramu}\\[-0.5cm]}
{\end{activite}}

\newenvironment{formule}%
%{\begin{activite}\large\\} % écrire plus gros
{\begin{form}\color{bleuamu}}
{\end{form}}


\usepackage[breakable]{tcolorbox}
    \usepackage{parskip} % Stop auto-indenting (to mimic markdown behaviour)
    
    \usepackage{iftex}
    \ifPDFTeX
    	\usepackage[T1]{fontenc}
    	\usepackage{mathpazo}
    \else
    	\usepackage{fontspec}
    \fi

    % Basic figure setup, for now with no caption control since it's done
    % automatically by Pandoc (which extracts ![](path) syntax from Markdown).
    \usepackage{graphicx}
    % Maintain compatibility with old templates. Remove in nbconvert 6.0
    \let\Oldincludegraphics\includegraphics
    % Ensure that by default, figures have no caption (until we provide a
    % proper Figure object with a Caption API and a way to capture that
    % in the conversion process - todo).
    \usepackage{caption}
    \DeclareCaptionFormat{nocaption}{}
    \captionsetup{format=nocaption,aboveskip=0pt,belowskip=0pt}

    \usepackage[Export]{adjustbox} % Used to constrain images to a maximum size
    \adjustboxset{max size={0.9\linewidth}{0.9\paperheight}}
    \usepackage{float}
    \floatplacement{figure}{H} % forces figures to be placed at the correct location
    \usepackage{xcolor} % Allow colors to be defined
    \usepackage{enumerate} % Needed for markdown enumerations to work
    \usepackage{geometry} % Used to adjust the document margins
    \usepackage{amsmath} % Equations
    \usepackage{amssymb} % Equations
    \usepackage{textcomp} % defines textquotesingle
    % Hack from http://tex.stackexchange.com/a/47451/13684:
    \AtBeginDocument{%
        \def\PYZsq{\textquotesingle}% Upright quotes in Pygmentized code
    }
    \usepackage{upquote} % Upright quotes for verbatim code
    \usepackage{eurosym} % defines \euro
    \usepackage[mathletters]{ucs} % Extended unicode (utf-8) support
    \usepackage{fancyvrb} % verbatim replacement that allows latex

    % The hyperref package gives us a pdf with properly built
    % internal navigation ('pdf bookmarks' for the table of contents,
    % internal cross-reference links, web links for URLs, etc.)
    \usepackage{hyperref}
    % The default LaTeX title has an obnoxious amount of whitespace. By default,
    % titling removes some of it. It also provides customization options.
    \usepackage{titling}
    \usepackage{longtable} % longtable support required by pandoc >1.10
    \usepackage{booktabs}  % table support for pandoc > 1.12.2
    \usepackage[inline]{enumitem} % IRkernel/repr support (it uses the enumerate* environment)
    \usepackage[normalem]{ulem} % ulem is needed to support strikethroughs (\sout)
                                % normalem makes italics be italics, not underlines
    \usepackage{mathrsfs}
    

    
    % Colors for the hyperref package
    \definecolor{urlcolor}{rgb}{0,.145,.698}
    \definecolor{linkcolor}{rgb}{.71,0.21,0.01}
    \definecolor{citecolor}{rgb}{.12,.54,.11}

    % ANSI colors
    \definecolor{ansi-black}{HTML}{3E424D}
    \definecolor{ansi-black-intense}{HTML}{282C36}
    \definecolor{ansi-red}{HTML}{E75C58}
    \definecolor{ansi-red-intense}{HTML}{B22B31}
    \definecolor{ansi-green}{HTML}{00A250}
    \definecolor{ansi-green-intense}{HTML}{007427}
    \definecolor{ansi-yellow}{HTML}{DDB62B}
    \definecolor{ansi-yellow-intense}{HTML}{B27D12}
    \definecolor{ansi-blue}{HTML}{208FFB}
    \definecolor{ansi-blue-intense}{HTML}{0065CA}
    \definecolor{ansi-magenta}{HTML}{D160C4}
    \definecolor{ansi-magenta-intense}{HTML}{A03196}
    \definecolor{ansi-cyan}{HTML}{60C6C8}
    \definecolor{ansi-cyan-intense}{HTML}{258F8F}
    \definecolor{ansi-white}{HTML}{C5C1B4}
    \definecolor{ansi-white-intense}{HTML}{A1A6B2}
    \definecolor{ansi-default-inverse-fg}{HTML}{FFFFFF}
    \definecolor{ansi-default-inverse-bg}{HTML}{000000}

    % commands and environments needed by pandoc snippets
    % extracted from the output of `pandoc -s`
    \providecommand{\tightlist}{%
      \setlength{\itemsep}{0pt}\setlength{\parskip}{0pt}}
    \DefineVerbatimEnvironment{Highlighting}{Verbatim}{commandchars=\\\{\}}
    % Add ',fontsize=\small' for more characters per line
    \newenvironment{Shaded}{}{}
    \newcommand{\KeywordTok}[1]{\textcolor[rgb]{0.00,0.44,0.13}{\textbf{{#1}}}}
    \newcommand{\DataTypeTok}[1]{\textcolor[rgb]{0.56,0.13,0.00}{{#1}}}
    \newcommand{\DecValTok}[1]{\textcolor[rgb]{0.25,0.63,0.44}{{#1}}}
    \newcommand{\BaseNTok}[1]{\textcolor[rgb]{0.25,0.63,0.44}{{#1}}}
    \newcommand{\FloatTok}[1]{\textcolor[rgb]{0.25,0.63,0.44}{{#1}}}
    \newcommand{\CharTok}[1]{\textcolor[rgb]{0.25,0.44,0.63}{{#1}}}
    \newcommand{\StringTok}[1]{\textcolor[rgb]{0.25,0.44,0.63}{{#1}}}
    \newcommand{\CommentTok}[1]{\textcolor[rgb]{0.38,0.63,0.69}{\textit{{#1}}}}
    \newcommand{\OtherTok}[1]{\textcolor[rgb]{0.00,0.44,0.13}{{#1}}}
    \newcommand{\AlertTok}[1]{\textcolor[rgb]{1.00,0.00,0.00}{\textbf{{#1}}}}
    \newcommand{\FunctionTok}[1]{\textcolor[rgb]{0.02,0.16,0.49}{{#1}}}
    \newcommand{\RegionMarkerTok}[1]{{#1}}
    \newcommand{\ErrorTok}[1]{\textcolor[rgb]{1.00,0.00,0.00}{\textbf{{#1}}}}
    \newcommand{\NormalTok}[1]{{#1}}
    
    % Additional commands for more recent versions of Pandoc
    \newcommand{\ConstantTok}[1]{\textcolor[rgb]{0.53,0.00,0.00}{{#1}}}
    \newcommand{\SpecialCharTok}[1]{\textcolor[rgb]{0.25,0.44,0.63}{{#1}}}
    \newcommand{\VerbatimStringTok}[1]{\textcolor[rgb]{0.25,0.44,0.63}{{#1}}}
    \newcommand{\SpecialStringTok}[1]{\textcolor[rgb]{0.73,0.40,0.53}{{#1}}}
    \newcommand{\ImportTok}[1]{{#1}}
    \newcommand{\DocumentationTok}[1]{\textcolor[rgb]{0.73,0.13,0.13}{\textit{{#1}}}}
    \newcommand{\AnnotationTok}[1]{\textcolor[rgb]{0.38,0.63,0.69}{\textbf{\textit{{#1}}}}}
    \newcommand{\CommentVarTok}[1]{\textcolor[rgb]{0.38,0.63,0.69}{\textbf{\textit{{#1}}}}}
    \newcommand{\VariableTok}[1]{\textcolor[rgb]{0.10,0.09,0.49}{{#1}}}
    \newcommand{\ControlFlowTok}[1]{\textcolor[rgb]{0.00,0.44,0.13}{\textbf{{#1}}}}
    \newcommand{\OperatorTok}[1]{\textcolor[rgb]{0.40,0.40,0.40}{{#1}}}
    \newcommand{\BuiltInTok}[1]{{#1}}
    \newcommand{\ExtensionTok}[1]{{#1}}
    \newcommand{\PreprocessorTok}[1]{\textcolor[rgb]{0.74,0.48,0.00}{{#1}}}
    \newcommand{\AttributeTok}[1]{\textcolor[rgb]{0.49,0.56,0.16}{{#1}}}
    \newcommand{\InformationTok}[1]{\textcolor[rgb]{0.38,0.63,0.69}{\textbf{\textit{{#1}}}}}
    \newcommand{\WarningTok}[1]{\textcolor[rgb]{0.38,0.63,0.69}{\textbf{\textit{{#1}}}}}
    
    
    % Define a nice break command that doesn't care if a line doesn't already
    % exist.
    \def\br{\hspace*{\fill} \\* }
    % Math Jax compatibility definitions
    \def\gt{>}
    \def\lt{<}
    \let\Oldtex\TeX
    \let\Oldlatex\LaTeX
    \renewcommand{\TeX}{\textrm{\Oldtex}}
    \renewcommand{\LaTeX}{\textrm{\Oldlatex}}
    % Document parameters
    % Document title
    \title{2-7-2---basesPython-Fonction-correctionExo}
    
    
    
    
    
% Pygments definitions
\makeatletter
\def\PY@reset{\let\PY@it=\relax \let\PY@bf=\relax%
    \let\PY@ul=\relax \let\PY@tc=\relax%
    \let\PY@bc=\relax \let\PY@ff=\relax}
\def\PY@tok#1{\csname PY@tok@#1\endcsname}
\def\PY@toks#1+{\ifx\relax#1\empty\else%
    \PY@tok{#1}\expandafter\PY@toks\fi}
\def\PY@do#1{\PY@bc{\PY@tc{\PY@ul{%
    \PY@it{\PY@bf{\PY@ff{#1}}}}}}}
\def\PY#1#2{\PY@reset\PY@toks#1+\relax+\PY@do{#2}}

\expandafter\def\csname PY@tok@w\endcsname{\def\PY@tc##1{\textcolor[rgb]{0.73,0.73,0.73}{##1}}}
\expandafter\def\csname PY@tok@c\endcsname{\let\PY@it=\textit\def\PY@tc##1{\textcolor[rgb]{0.25,0.50,0.50}{##1}}}
\expandafter\def\csname PY@tok@cp\endcsname{\def\PY@tc##1{\textcolor[rgb]{0.74,0.48,0.00}{##1}}}
\expandafter\def\csname PY@tok@k\endcsname{\let\PY@bf=\textbf\def\PY@tc##1{\textcolor[rgb]{0.00,0.50,0.00}{##1}}}
\expandafter\def\csname PY@tok@kp\endcsname{\def\PY@tc##1{\textcolor[rgb]{0.00,0.50,0.00}{##1}}}
\expandafter\def\csname PY@tok@kt\endcsname{\def\PY@tc##1{\textcolor[rgb]{0.69,0.00,0.25}{##1}}}
\expandafter\def\csname PY@tok@o\endcsname{\def\PY@tc##1{\textcolor[rgb]{0.40,0.40,0.40}{##1}}}
\expandafter\def\csname PY@tok@ow\endcsname{\let\PY@bf=\textbf\def\PY@tc##1{\textcolor[rgb]{0.67,0.13,1.00}{##1}}}
\expandafter\def\csname PY@tok@nb\endcsname{\def\PY@tc##1{\textcolor[rgb]{0.00,0.50,0.00}{##1}}}
\expandafter\def\csname PY@tok@nf\endcsname{\def\PY@tc##1{\textcolor[rgb]{0.00,0.00,1.00}{##1}}}
\expandafter\def\csname PY@tok@nc\endcsname{\let\PY@bf=\textbf\def\PY@tc##1{\textcolor[rgb]{0.00,0.00,1.00}{##1}}}
\expandafter\def\csname PY@tok@nn\endcsname{\let\PY@bf=\textbf\def\PY@tc##1{\textcolor[rgb]{0.00,0.00,1.00}{##1}}}
\expandafter\def\csname PY@tok@ne\endcsname{\let\PY@bf=\textbf\def\PY@tc##1{\textcolor[rgb]{0.82,0.25,0.23}{##1}}}
\expandafter\def\csname PY@tok@nv\endcsname{\def\PY@tc##1{\textcolor[rgb]{0.10,0.09,0.49}{##1}}}
\expandafter\def\csname PY@tok@no\endcsname{\def\PY@tc##1{\textcolor[rgb]{0.53,0.00,0.00}{##1}}}
\expandafter\def\csname PY@tok@nl\endcsname{\def\PY@tc##1{\textcolor[rgb]{0.63,0.63,0.00}{##1}}}
\expandafter\def\csname PY@tok@ni\endcsname{\let\PY@bf=\textbf\def\PY@tc##1{\textcolor[rgb]{0.60,0.60,0.60}{##1}}}
\expandafter\def\csname PY@tok@na\endcsname{\def\PY@tc##1{\textcolor[rgb]{0.49,0.56,0.16}{##1}}}
\expandafter\def\csname PY@tok@nt\endcsname{\let\PY@bf=\textbf\def\PY@tc##1{\textcolor[rgb]{0.00,0.50,0.00}{##1}}}
\expandafter\def\csname PY@tok@nd\endcsname{\def\PY@tc##1{\textcolor[rgb]{0.67,0.13,1.00}{##1}}}
\expandafter\def\csname PY@tok@s\endcsname{\def\PY@tc##1{\textcolor[rgb]{0.73,0.13,0.13}{##1}}}
\expandafter\def\csname PY@tok@sd\endcsname{\let\PY@it=\textit\def\PY@tc##1{\textcolor[rgb]{0.73,0.13,0.13}{##1}}}
\expandafter\def\csname PY@tok@si\endcsname{\let\PY@bf=\textbf\def\PY@tc##1{\textcolor[rgb]{0.73,0.40,0.53}{##1}}}
\expandafter\def\csname PY@tok@se\endcsname{\let\PY@bf=\textbf\def\PY@tc##1{\textcolor[rgb]{0.73,0.40,0.13}{##1}}}
\expandafter\def\csname PY@tok@sr\endcsname{\def\PY@tc##1{\textcolor[rgb]{0.73,0.40,0.53}{##1}}}
\expandafter\def\csname PY@tok@ss\endcsname{\def\PY@tc##1{\textcolor[rgb]{0.10,0.09,0.49}{##1}}}
\expandafter\def\csname PY@tok@sx\endcsname{\def\PY@tc##1{\textcolor[rgb]{0.00,0.50,0.00}{##1}}}
\expandafter\def\csname PY@tok@m\endcsname{\def\PY@tc##1{\textcolor[rgb]{0.40,0.40,0.40}{##1}}}
\expandafter\def\csname PY@tok@gh\endcsname{\let\PY@bf=\textbf\def\PY@tc##1{\textcolor[rgb]{0.00,0.00,0.50}{##1}}}
\expandafter\def\csname PY@tok@gu\endcsname{\let\PY@bf=\textbf\def\PY@tc##1{\textcolor[rgb]{0.50,0.00,0.50}{##1}}}
\expandafter\def\csname PY@tok@gd\endcsname{\def\PY@tc##1{\textcolor[rgb]{0.63,0.00,0.00}{##1}}}
\expandafter\def\csname PY@tok@gi\endcsname{\def\PY@tc##1{\textcolor[rgb]{0.00,0.63,0.00}{##1}}}
\expandafter\def\csname PY@tok@gr\endcsname{\def\PY@tc##1{\textcolor[rgb]{1.00,0.00,0.00}{##1}}}
\expandafter\def\csname PY@tok@ge\endcsname{\let\PY@it=\textit}
\expandafter\def\csname PY@tok@gs\endcsname{\let\PY@bf=\textbf}
\expandafter\def\csname PY@tok@gp\endcsname{\let\PY@bf=\textbf\def\PY@tc##1{\textcolor[rgb]{0.00,0.00,0.50}{##1}}}
\expandafter\def\csname PY@tok@go\endcsname{\def\PY@tc##1{\textcolor[rgb]{0.53,0.53,0.53}{##1}}}
\expandafter\def\csname PY@tok@gt\endcsname{\def\PY@tc##1{\textcolor[rgb]{0.00,0.27,0.87}{##1}}}
\expandafter\def\csname PY@tok@err\endcsname{\def\PY@bc##1{\setlength{\fboxsep}{0pt}\fcolorbox[rgb]{1.00,0.00,0.00}{1,1,1}{\strut ##1}}}
\expandafter\def\csname PY@tok@kc\endcsname{\let\PY@bf=\textbf\def\PY@tc##1{\textcolor[rgb]{0.00,0.50,0.00}{##1}}}
\expandafter\def\csname PY@tok@kd\endcsname{\let\PY@bf=\textbf\def\PY@tc##1{\textcolor[rgb]{0.00,0.50,0.00}{##1}}}
\expandafter\def\csname PY@tok@kn\endcsname{\let\PY@bf=\textbf\def\PY@tc##1{\textcolor[rgb]{0.00,0.50,0.00}{##1}}}
\expandafter\def\csname PY@tok@kr\endcsname{\let\PY@bf=\textbf\def\PY@tc##1{\textcolor[rgb]{0.00,0.50,0.00}{##1}}}
\expandafter\def\csname PY@tok@bp\endcsname{\def\PY@tc##1{\textcolor[rgb]{0.00,0.50,0.00}{##1}}}
\expandafter\def\csname PY@tok@fm\endcsname{\def\PY@tc##1{\textcolor[rgb]{0.00,0.00,1.00}{##1}}}
\expandafter\def\csname PY@tok@vc\endcsname{\def\PY@tc##1{\textcolor[rgb]{0.10,0.09,0.49}{##1}}}
\expandafter\def\csname PY@tok@vg\endcsname{\def\PY@tc##1{\textcolor[rgb]{0.10,0.09,0.49}{##1}}}
\expandafter\def\csname PY@tok@vi\endcsname{\def\PY@tc##1{\textcolor[rgb]{0.10,0.09,0.49}{##1}}}
\expandafter\def\csname PY@tok@vm\endcsname{\def\PY@tc##1{\textcolor[rgb]{0.10,0.09,0.49}{##1}}}
\expandafter\def\csname PY@tok@sa\endcsname{\def\PY@tc##1{\textcolor[rgb]{0.73,0.13,0.13}{##1}}}
\expandafter\def\csname PY@tok@sb\endcsname{\def\PY@tc##1{\textcolor[rgb]{0.73,0.13,0.13}{##1}}}
\expandafter\def\csname PY@tok@sc\endcsname{\def\PY@tc##1{\textcolor[rgb]{0.73,0.13,0.13}{##1}}}
\expandafter\def\csname PY@tok@dl\endcsname{\def\PY@tc##1{\textcolor[rgb]{0.73,0.13,0.13}{##1}}}
\expandafter\def\csname PY@tok@s2\endcsname{\def\PY@tc##1{\textcolor[rgb]{0.73,0.13,0.13}{##1}}}
\expandafter\def\csname PY@tok@sh\endcsname{\def\PY@tc##1{\textcolor[rgb]{0.73,0.13,0.13}{##1}}}
\expandafter\def\csname PY@tok@s1\endcsname{\def\PY@tc##1{\textcolor[rgb]{0.73,0.13,0.13}{##1}}}
\expandafter\def\csname PY@tok@mb\endcsname{\def\PY@tc##1{\textcolor[rgb]{0.40,0.40,0.40}{##1}}}
\expandafter\def\csname PY@tok@mf\endcsname{\def\PY@tc##1{\textcolor[rgb]{0.40,0.40,0.40}{##1}}}
\expandafter\def\csname PY@tok@mh\endcsname{\def\PY@tc##1{\textcolor[rgb]{0.40,0.40,0.40}{##1}}}
\expandafter\def\csname PY@tok@mi\endcsname{\def\PY@tc##1{\textcolor[rgb]{0.40,0.40,0.40}{##1}}}
\expandafter\def\csname PY@tok@il\endcsname{\def\PY@tc##1{\textcolor[rgb]{0.40,0.40,0.40}{##1}}}
\expandafter\def\csname PY@tok@mo\endcsname{\def\PY@tc##1{\textcolor[rgb]{0.40,0.40,0.40}{##1}}}
\expandafter\def\csname PY@tok@ch\endcsname{\let\PY@it=\textit\def\PY@tc##1{\textcolor[rgb]{0.25,0.50,0.50}{##1}}}
\expandafter\def\csname PY@tok@cm\endcsname{\let\PY@it=\textit\def\PY@tc##1{\textcolor[rgb]{0.25,0.50,0.50}{##1}}}
\expandafter\def\csname PY@tok@cpf\endcsname{\let\PY@it=\textit\def\PY@tc##1{\textcolor[rgb]{0.25,0.50,0.50}{##1}}}
\expandafter\def\csname PY@tok@c1\endcsname{\let\PY@it=\textit\def\PY@tc##1{\textcolor[rgb]{0.25,0.50,0.50}{##1}}}
\expandafter\def\csname PY@tok@cs\endcsname{\let\PY@it=\textit\def\PY@tc##1{\textcolor[rgb]{0.25,0.50,0.50}{##1}}}

\def\PYZbs{\char`\\}
\def\PYZus{\char`\_}
\def\PYZob{\char`\{}
\def\PYZcb{\char`\}}
\def\PYZca{\char`\^}
\def\PYZam{\char`\&}
\def\PYZlt{\char`\<}
\def\PYZgt{\char`\>}
\def\PYZsh{\char`\#}
\def\PYZpc{\char`\%}
\def\PYZdl{\char`\$}
\def\PYZhy{\char`\-}
\def\PYZsq{\char`\'}
\def\PYZdq{\char`\"}
\def\PYZti{\char`\~}
% for compatibility with earlier versions
\def\PYZat{@}
\def\PYZlb{[}
\def\PYZrb{]}
\makeatother


    % For linebreaks inside Verbatim environment from package fancyvrb. 
    \makeatletter
        \newbox\Wrappedcontinuationbox 
        \newbox\Wrappedvisiblespacebox 
        \newcommand*\Wrappedvisiblespace {\textcolor{red}{\textvisiblespace}} 
        \newcommand*\Wrappedcontinuationsymbol {\textcolor{red}{\llap{\tiny$\m@th\hookrightarrow$}}} 
        \newcommand*\Wrappedcontinuationindent {3ex } 
        \newcommand*\Wrappedafterbreak {\kern\Wrappedcontinuationindent\copy\Wrappedcontinuationbox} 
        % Take advantage of the already applied Pygments mark-up to insert 
        % potential linebreaks for TeX processing. 
        %        {, <, #, %, $, ' and ": go to next line. 
        %        _, }, ^, &, >, - and ~: stay at end of broken line. 
        % Use of \textquotesingle for straight quote. 
        \newcommand*\Wrappedbreaksatspecials {% 
            \def\PYGZus{\discretionary{\char`\_}{\Wrappedafterbreak}{\char`\_}}% 
            \def\PYGZob{\discretionary{}{\Wrappedafterbreak\char`\{}{\char`\{}}% 
            \def\PYGZcb{\discretionary{\char`\}}{\Wrappedafterbreak}{\char`\}}}% 
            \def\PYGZca{\discretionary{\char`\^}{\Wrappedafterbreak}{\char`\^}}% 
            \def\PYGZam{\discretionary{\char`\&}{\Wrappedafterbreak}{\char`\&}}% 
            \def\PYGZlt{\discretionary{}{\Wrappedafterbreak\char`\<}{\char`\<}}% 
            \def\PYGZgt{\discretionary{\char`\>}{\Wrappedafterbreak}{\char`\>}}% 
            \def\PYGZsh{\discretionary{}{\Wrappedafterbreak\char`\#}{\char`\#}}% 
            \def\PYGZpc{\discretionary{}{\Wrappedafterbreak\char`\%}{\char`\%}}% 
            \def\PYGZdl{\discretionary{}{\Wrappedafterbreak\char`\$}{\char`\$}}% 
            \def\PYGZhy{\discretionary{\char`\-}{\Wrappedafterbreak}{\char`\-}}% 
            \def\PYGZsq{\discretionary{}{\Wrappedafterbreak\textquotesingle}{\textquotesingle}}% 
            \def\PYGZdq{\discretionary{}{\Wrappedafterbreak\char`\"}{\char`\"}}% 
            \def\PYGZti{\discretionary{\char`\~}{\Wrappedafterbreak}{\char`\~}}% 
        } 
        % Some characters . , ; ? ! / are not pygmentized. 
        % This macro makes them "active" and they will insert potential linebreaks 
        \newcommand*\Wrappedbreaksatpunct {% 
            \lccode`\~`\.\lowercase{\def~}{\discretionary{\hbox{\char`\.}}{\Wrappedafterbreak}{\hbox{\char`\.}}}% 
            \lccode`\~`\,\lowercase{\def~}{\discretionary{\hbox{\char`\,}}{\Wrappedafterbreak}{\hbox{\char`\,}}}% 
            \lccode`\~`\;\lowercase{\def~}{\discretionary{\hbox{\char`\;}}{\Wrappedafterbreak}{\hbox{\char`\;}}}% 
            \lccode`\~`\:\lowercase{\def~}{\discretionary{\hbox{\char`\:}}{\Wrappedafterbreak}{\hbox{\char`\:}}}% 
            \lccode`\~`\?\lowercase{\def~}{\discretionary{\hbox{\char`\?}}{\Wrappedafterbreak}{\hbox{\char`\?}}}% 
            \lccode`\~`\!\lowercase{\def~}{\discretionary{\hbox{\char`\!}}{\Wrappedafterbreak}{\hbox{\char`\!}}}% 
            \lccode`\~`\/\lowercase{\def~}{\discretionary{\hbox{\char`\/}}{\Wrappedafterbreak}{\hbox{\char`\/}}}% 
            \catcode`\.\active
            \catcode`\,\active 
            \catcode`\;\active
            \catcode`\:\active
            \catcode`\?\active
            \catcode`\!\active
            \catcode`\/\active 
            \lccode`\~`\~ 	
        }
    \makeatother

    \let\OriginalVerbatim=\Verbatim
    \makeatletter
    \renewcommand{\Verbatim}[1][1]{%
        %\parskip\z@skip
        \sbox\Wrappedcontinuationbox {\Wrappedcontinuationsymbol}%
        \sbox\Wrappedvisiblespacebox {\FV@SetupFont\Wrappedvisiblespace}%
        \def\FancyVerbFormatLine ##1{\hsize\linewidth
            \vtop{\raggedright\hyphenpenalty\z@\exhyphenpenalty\z@
                \doublehyphendemerits\z@\finalhyphendemerits\z@
                \strut ##1\strut}%
        }%
        % If the linebreak is at a space, the latter will be displayed as visible
        % space at end of first line, and a continuation symbol starts next line.
        % Stretch/shrink are however usually zero for typewriter font.
        \def\FV@Space {%
            \nobreak\hskip\z@ plus\fontdimen3\font minus\fontdimen4\font
            \discretionary{\copy\Wrappedvisiblespacebox}{\Wrappedafterbreak}
            {\kern\fontdimen2\font}%
        }%
        
        % Allow breaks at special characters using \PYG... macros.
        \Wrappedbreaksatspecials
        % Breaks at punctuation characters . , ; ? ! and / need catcode=\active 	
        \OriginalVerbatim[#1,codes*=\Wrappedbreaksatpunct]%
    }
    \makeatother

    % Exact colors from NB
    \definecolor{incolor}{HTML}{303F9F}
    \definecolor{outcolor}{HTML}{D84315}
    \definecolor{cellborder}{HTML}{CFCFCF}
    \definecolor{cellbackground}{HTML}{F7F7F7}
    
    % prompt
    \makeatletter
    \newcommand{\boxspacing}{\kern\kvtcb@left@rule\kern\kvtcb@boxsep}
    \makeatother
    \newcommand{\prompt}[4]{
        \ttfamily\llap{{\color{#2}[#3]:\hspace{3pt}#4}}\vspace{-\baselineskip}
    }
    

    
\setlength\headheight{30pt}
\setcounter{secnumdepth}{0} % Turns off numbering for sections

    % Prevent overflowing lines due to hard-to-break entities
    \sloppy 
    % Setup hyperref package
    \hypersetup{
      breaklinks=true,  % so long urls are correctly broken across lines
      colorlinks=true,
      urlcolor=urlcolor,
      linkcolor=linkcolor,
      citecolor=citecolor,
      }
    % Slightly bigger margins than the latex defaults
    \geometry{a4paper,tmargin=3cm,bmargin=2cm,lmargin=1cm,rmargin=1cm}\fancyhead[L]{Thème à définir}\fancyhead[L]{\adjustimage{height=1cm, valign=m}{/home/bouscadilla/Documents/Code/nbconvert/template/latex/pdf_solution/papier_eleve_ico_langage}\ttfamily\scshape Langage}\fancyhead[C]{\bfseries\MakeUppercase{2-7-2---basesPython-Fonction-correctionExo}}\fancyhead[C]{\bfseries\MakeUppercase{7 --- Fonction}}\fancyhead[R]{\monthyeardate\today}

    \fancyfoot[C]{\thepage}
    % #TODO ajouter les pages totales

    \pagestyle{fancy}
    


\begin{document}
    
    \title{7 --- Fonction}
% \maketitle

    
    

    
    \hypertarget{fonctions---exercices-fonctions-correction}{%
\subsection{7 --- Fonctions - Exercices fonctions
(correction)}\label{fonctions---exercices-fonctions-correction}}
\begin{eleve}
    \textbf{Exercice 1}

Définir une fonction \texttt{test\_Pythagore} qui prend trois entiers
\(a\), \(b\) et \(c\) en arguments et renvoie un booléen indiquant si
\(a^2 + b^2 = c^2\), ou \(b^2 + c^2 = a^2\) ou \(c^2 + a^2 = b^2\).

\hypertarget{tests-et-exemples-dusage}{%
\paragraph{Tests et exemples d'usage :}\label{tests-et-exemples-dusage}}

\begin{Shaded}
\begin{Highlighting}[]
\OperatorTok{\textgreater{}\textgreater{}\textgreater{}} \BuiltInTok{print}\NormalTok{(test\_pythagore(}\DecValTok{3}\NormalTok{, }\DecValTok{4}\NormalTok{, }\DecValTok{5}\NormalTok{))}
\VariableTok{True}
\OperatorTok{\textgreater{}\textgreater{}\textgreater{}} \BuiltInTok{print}\NormalTok{(test\_pythagore(}\DecValTok{7}\NormalTok{, }\DecValTok{2}\NormalTok{, }\DecValTok{12}\NormalTok{))}
\VariableTok{False}
\OperatorTok{\textgreater{}\textgreater{}\textgreater{}} \BuiltInTok{print}\NormalTok{(test\_pythagore(}\DecValTok{3}\NormalTok{, }\DecValTok{5}\NormalTok{, }\DecValTok{4}\NormalTok{))}
\VariableTok{True}
\OperatorTok{\textgreater{}\textgreater{}\textgreater{}} \BuiltInTok{print}\NormalTok{(test\_pythagore(}\DecValTok{5}\NormalTok{, }\DecValTok{4}\NormalTok{, }\DecValTok{3}\NormalTok{))}
\VariableTok{True}
\end{Highlighting}
\end{Shaded}
        
        \end{eleve}
        {\scriptsize
    \begin{tcolorbox}[breakable, size=fbox, boxrule=1pt, pad at break*=1mm,colback=cellbackground, colframe=cellborder]
\prompt{In}{incolor}{1}{\boxspacing}
\begin{Verbatim}[commandchars=\\\{\}]
\PY{k}{def} \PY{n+nf}{test\PYZus{}pythagore}\PY{p}{(}\PY{n}{a}\PY{p}{,} \PY{n}{b}\PY{p}{,} \PY{n}{c}\PY{p}{)}\PY{p}{:}
    \PY{n}{est\PYZus{}rectangle} \PY{o}{=} \PY{n}{a}\PY{o}{*}\PY{n}{a} \PY{o}{+} \PY{n}{b}\PY{o}{*}\PY{n}{b} \PY{o}{==} \PY{n}{c}\PY{o}{*}\PY{n}{c} \PYZbs{}
                    \PY{o+ow}{or} \PY{n}{b}\PY{o}{*}\PY{n}{b} \PY{o}{+} \PY{n}{c}\PY{o}{*}\PY{n}{c} \PY{o}{==} \PY{n}{a}\PY{o}{*}\PY{n}{a} \PYZbs{}
                    \PY{o+ow}{or} \PY{n}{c}\PY{o}{*}\PY{n}{c} \PY{o}{+} \PY{n}{a}\PY{o}{*}\PY{n}{a} \PY{o}{==} \PY{n}{b}\PY{o}{*}\PY{n}{b}
    \PY{k}{return} \PY{n}{est\PYZus{}rectangle}


\PY{n+nb}{print}\PY{p}{(}\PY{n}{test\PYZus{}pythagore}\PY{p}{(}\PY{l+m+mi}{3}\PY{p}{,} \PY{l+m+mi}{4}\PY{p}{,} \PY{l+m+mi}{5}\PY{p}{)}\PY{p}{)}
\PY{n+nb}{print}\PY{p}{(}\PY{n}{test\PYZus{}pythagore}\PY{p}{(}\PY{l+m+mi}{7}\PY{p}{,} \PY{l+m+mi}{2}\PY{p}{,} \PY{l+m+mi}{12}\PY{p}{)}\PY{p}{)}
\PY{n+nb}{print}\PY{p}{(}\PY{n}{test\PYZus{}pythagore}\PY{p}{(}\PY{l+m+mi}{3}\PY{p}{,} \PY{l+m+mi}{5}\PY{p}{,} \PY{l+m+mi}{4}\PY{p}{)}\PY{p}{)}
\PY{n+nb}{print}\PY{p}{(}\PY{n}{test\PYZus{}pythagore}\PY{p}{(}\PY{l+m+mi}{5}\PY{p}{,} \PY{l+m+mi}{4}\PY{p}{,} \PY{l+m+mi}{3}\PY{p}{)}\PY{p}{)}
\end{Verbatim}
\end{tcolorbox}
    }

    \begin{Verbatim}[commandchars=\\\{\}]
True
False
True
True
    \end{Verbatim}
\begin{eleve}
    \textbf{Exercice 2}

Définir une fonction \texttt{valeur\_absolue} qui prend un entier en
argument et renvoie sa valeur absolue.

\hypertarget{exemples-et-tests}{%
\paragraph{Exemples et tests:}\label{exemples-et-tests}}

\begin{Shaded}
\begin{Highlighting}[]
\OperatorTok{\textgreater{}\textgreater{}\textgreater{}} \BuiltInTok{print}\NormalTok{(valeur\_absolue(}\DecValTok{5}\NormalTok{))}
\DecValTok{5}
\OperatorTok{\textgreater{}\textgreater{}\textgreater{}} \BuiltInTok{print}\NormalTok{(valeur\_absolue(}\OperatorTok{{-}}\DecValTok{5}\NormalTok{))}
\DecValTok{5}
\OperatorTok{\textgreater{}\textgreater{}\textgreater{}} \BuiltInTok{print}\NormalTok{(valeur\_absolue(}\DecValTok{0}\NormalTok{))}
\DecValTok{0}
\OperatorTok{\textgreater{}\textgreater{}\textgreater{}} \BuiltInTok{print}\NormalTok{(valeur\_absolue(}\OperatorTok{{-}}\DecValTok{42}\NormalTok{))}
\DecValTok{42}
\end{Highlighting}
\end{Shaded}
        
        \end{eleve}
        {\scriptsize
    \begin{tcolorbox}[breakable, size=fbox, boxrule=1pt, pad at break*=1mm,colback=cellbackground, colframe=cellborder]
\prompt{In}{incolor}{2}{\boxspacing}
\begin{Verbatim}[commandchars=\\\{\}]
\PY{k}{def} \PY{n+nf}{valeur\PYZus{}absolue}\PY{p}{(}\PY{n}{n}\PY{p}{)}\PY{p}{:}
    \PY{k}{if} \PY{n}{n} \PY{o}{\PYZgt{}}\PY{o}{=} \PY{l+m+mi}{0}\PY{p}{:}
        \PY{k}{return} \PY{n}{n}
        
    \PY{k}{return} \PY{o}{\PYZhy{}}\PY{n}{n}


\PY{n+nb}{print}\PY{p}{(}\PY{n}{valeur\PYZus{}absolue}\PY{p}{(}\PY{l+m+mi}{5}\PY{p}{)}\PY{p}{)}
\PY{n+nb}{print}\PY{p}{(}\PY{n}{valeur\PYZus{}absolue}\PY{p}{(}\PY{o}{\PYZhy{}}\PY{l+m+mi}{5}\PY{p}{)}\PY{p}{)}
\PY{n+nb}{print}\PY{p}{(}\PY{n}{valeur\PYZus{}absolue}\PY{p}{(}\PY{l+m+mi}{0}\PY{p}{)}\PY{p}{)}
\PY{n+nb}{print}\PY{p}{(}\PY{n}{valeur\PYZus{}absolue}\PY{p}{(}\PY{o}{\PYZhy{}}\PY{l+m+mi}{42}\PY{p}{)}\PY{p}{)}
\end{Verbatim}
\end{tcolorbox}
    }

    \begin{Verbatim}[commandchars=\\\{\}]
5
5
0
42
    \end{Verbatim}
\begin{eleve}
    \textbf{Exercice 3}

Créer une fonction \texttt{multiples} pour qu'elle prenne la limite en
argument plutôt que d'utiliser la valeur \(999\). En déduire une
fonction \texttt{multiples\_3\_ou\_5(borne\_sup)} qui renvoie la somme
des multiples de 3 ou 5 inférieurs ou égaux à \texttt{borne\_sup}.

\hypertarget{exemples-et-tests}{%
\paragraph{Exemples et tests:}\label{exemples-et-tests}}

\begin{Shaded}
\begin{Highlighting}[]
\OperatorTok{\textgreater{}\textgreater{}\textgreater{}} \BuiltInTok{print}\NormalTok{(multiples\_3\_ou\_5(}\DecValTok{2}\NormalTok{))}
\DecValTok{0}
\OperatorTok{\textgreater{}\textgreater{}\textgreater{}} \BuiltInTok{print}\NormalTok{(multiples\_3\_ou\_5(}\DecValTok{6}\NormalTok{))}
\DecValTok{14}
\OperatorTok{\textgreater{}\textgreater{}\textgreater{}} \BuiltInTok{print}\NormalTok{(multiples\_3\_ou\_5(}\DecValTok{10}\NormalTok{))}
\DecValTok{33}
\OperatorTok{\textgreater{}\textgreater{}\textgreater{}} \BuiltInTok{print}\NormalTok{(multiples\_3\_ou\_5(}\DecValTok{100}\NormalTok{))}
\DecValTok{2418}
\end{Highlighting}
\end{Shaded}
        
        \end{eleve}
        {\scriptsize
    \begin{tcolorbox}[breakable, size=fbox, boxrule=1pt, pad at break*=1mm,colback=cellbackground, colframe=cellborder]
\prompt{In}{incolor}{4}{\boxspacing}
\begin{Verbatim}[commandchars=\\\{\}]
\PY{k}{def} \PY{n+nf}{multiples\PYZus{}3\PYZus{}ou\PYZus{}5}\PY{p}{(}\PY{n}{borne\PYZus{}sup}\PY{p}{)}\PY{p}{:}
    \PY{n}{somme} \PY{o}{=} \PY{l+m+mi}{0}
    \PY{k}{for} \PY{n}{i} \PY{o+ow}{in} \PY{n+nb}{range}\PY{p}{(}\PY{n}{borne\PYZus{}sup}\PY{o}{+}\PY{l+m+mi}{1}\PY{p}{)}\PY{p}{:}
        \PY{k}{if} \PY{n}{i} \PY{o}{\PYZpc{}} \PY{l+m+mi}{3} \PY{o}{==} \PY{l+m+mi}{0} \PY{o+ow}{or} \PY{n}{i} \PY{o}{\PYZpc{}} \PY{l+m+mi}{5} \PY{o}{==} \PY{l+m+mi}{0}\PY{p}{:}
            \PY{n}{somme} \PY{o}{+}\PY{o}{=} \PY{n}{i}
    \PY{k}{return} \PY{n}{somme}


\PY{n+nb}{print}\PY{p}{(}\PY{n}{multiples\PYZus{}3\PYZus{}ou\PYZus{}5}\PY{p}{(}\PY{l+m+mi}{2}\PY{p}{)}\PY{p}{)}
\PY{n+nb}{print}\PY{p}{(}\PY{n}{multiples\PYZus{}3\PYZus{}ou\PYZus{}5}\PY{p}{(}\PY{l+m+mi}{6}\PY{p}{)}\PY{p}{)}
\PY{n+nb}{print}\PY{p}{(}\PY{n}{multiples\PYZus{}3\PYZus{}ou\PYZus{}5}\PY{p}{(}\PY{l+m+mi}{10}\PY{p}{)}\PY{p}{)}
\PY{n+nb}{print}\PY{p}{(}\PY{n}{multiples\PYZus{}3\PYZus{}ou\PYZus{}5}\PY{p}{(}\PY{l+m+mi}{100}\PY{p}{)}\PY{p}{)}
\end{Verbatim}
\end{tcolorbox}
    }

    \begin{Verbatim}[commandchars=\\\{\}]
0
14
33
2418
    \end{Verbatim}
\begin{eleve}
    \textbf{Exercice 4}

Écrire une fonction \texttt{max2(a,\ b)} qui renvoie le plus grand des
deux entiers \texttt{a} et \texttt{b}.

\hypertarget{exemples-et-tests}{%
\subsubsection{Exemples et tests:}\label{exemples-et-tests}}

\begin{Shaded}
\begin{Highlighting}[]
\OperatorTok{\textgreater{}\textgreater{}\textgreater{}} \BuiltInTok{print}\NormalTok{(}\BuiltInTok{max}\NormalTok{(}\DecValTok{0}\NormalTok{, }\DecValTok{0}\NormalTok{))}
\DecValTok{0}
\OperatorTok{\textgreater{}\textgreater{}\textgreater{}} \BuiltInTok{print}\NormalTok{(}\BuiltInTok{max}\NormalTok{(}\DecValTok{1}\NormalTok{, }\DecValTok{2}\NormalTok{))}
\DecValTok{2}
\OperatorTok{\textgreater{}\textgreater{}\textgreater{}} \BuiltInTok{print}\NormalTok{(}\BuiltInTok{max}\NormalTok{(}\OperatorTok{{-}}\DecValTok{2}\NormalTok{, }\OperatorTok{{-}}\DecValTok{1}\NormalTok{))}
\OperatorTok{{-}}\DecValTok{1}
\end{Highlighting}
\end{Shaded}
        
        \end{eleve}
        {\scriptsize
    \begin{tcolorbox}[breakable, size=fbox, boxrule=1pt, pad at break*=1mm,colback=cellbackground, colframe=cellborder]
\prompt{In}{incolor}{5}{\boxspacing}
\begin{Verbatim}[commandchars=\\\{\}]
\PY{k}{def} \PY{n+nf}{max2}\PY{p}{(}\PY{n}{a}\PY{p}{,} \PY{n}{b}\PY{p}{)}\PY{p}{:}
    \PY{k}{if} \PY{n}{a} \PY{o}{\PYZgt{}}\PY{o}{=} \PY{n}{b}\PY{p}{:}
        \PY{k}{return} \PY{n}{a}
    
    \PY{k}{return} \PY{n}{b}


\PY{n+nb}{print}\PY{p}{(}\PY{n+nb}{max}\PY{p}{(}\PY{l+m+mi}{0}\PY{p}{,}\PY{l+m+mi}{0}\PY{p}{)}\PY{p}{)}
\PY{n+nb}{print}\PY{p}{(}\PY{n+nb}{max}\PY{p}{(}\PY{l+m+mi}{1}\PY{p}{,}\PY{l+m+mi}{2}\PY{p}{)}\PY{p}{)}
\PY{n+nb}{print}\PY{p}{(}\PY{n+nb}{max}\PY{p}{(}\PY{o}{\PYZhy{}}\PY{l+m+mi}{2}\PY{p}{,}\PY{o}{\PYZhy{}}\PY{l+m+mi}{1}\PY{p}{)}\PY{p}{)}
\end{Verbatim}
\end{tcolorbox}
    }

    \begin{Verbatim}[commandchars=\\\{\}]
0
2
-1
    \end{Verbatim}
\begin{eleve}
    \textbf{Exercice 5}

Écrire une fonction \texttt{max3(a,\ b,\ c)} qui utilise la fonction
\texttt{max2} de l'exercice précédent et qui renvoie le plus grand des
trois entiers \texttt{a}, \texttt{b}, \texttt{c}.

La fonction \texttt{max2} est disponible et vous pouvez l'utiliser sans
l'implémenter.

\hypertarget{exemples-et-tests}{%
\paragraph{Exemples et tests:}\label{exemples-et-tests}}

\begin{Shaded}
\begin{Highlighting}[]
\OperatorTok{\textgreater{}\textgreater{}\textgreater{}}\NormalTok{ max3(}\DecValTok{0}\NormalTok{, }\DecValTok{0}\NormalTok{, }\DecValTok{0}\NormalTok{)}
\DecValTok{0}
\OperatorTok{\textgreater{}\textgreater{}\textgreater{}} \BuiltInTok{print}\NormalTok{(max3(}\DecValTok{0}\NormalTok{,}\DecValTok{5}\NormalTok{,}\DecValTok{2}\NormalTok{))}
\DecValTok{5}
\OperatorTok{\textgreater{}\textgreater{}\textgreater{}} \BuiltInTok{print}\NormalTok{(max3(}\OperatorTok{{-}}\DecValTok{10}\NormalTok{,}\DecValTok{5}\NormalTok{,}\DecValTok{21}\NormalTok{))}
\DecValTok{21}
\end{Highlighting}
\end{Shaded}
        
        \end{eleve}
        {\scriptsize
    \begin{tcolorbox}[breakable, size=fbox, boxrule=1pt, pad at break*=1mm,colback=cellbackground, colframe=cellborder]
\prompt{In}{incolor}{6}{\boxspacing}
\begin{Verbatim}[commandchars=\\\{\}]
\PY{k}{def} \PY{n+nf}{max3}\PY{p}{(}\PY{n}{a}\PY{p}{,} \PY{n}{b}\PY{p}{,} \PY{n}{c}\PY{p}{)}\PY{p}{:}
    \PY{k}{return} \PY{n}{max2}\PY{p}{(}\PY{n}{a}\PY{p}{,} \PY{n}{max2}\PY{p}{(}\PY{n}{b}\PY{p}{,} \PY{n}{c}\PY{p}{)}\PY{p}{)}


\PY{n+nb}{print}\PY{p}{(}\PY{n}{max3}\PY{p}{(}\PY{l+m+mi}{0}\PY{p}{,} \PY{l+m+mi}{0}\PY{p}{,} \PY{l+m+mi}{0}\PY{p}{)}\PY{p}{)}
\PY{n+nb}{print}\PY{p}{(}\PY{n}{max3}\PY{p}{(}\PY{l+m+mi}{0}\PY{p}{,}\PY{l+m+mi}{5}\PY{p}{,}\PY{l+m+mi}{2}\PY{p}{)}\PY{p}{)}
\PY{n+nb}{print}\PY{p}{(}\PY{n}{max3}\PY{p}{(}\PY{o}{\PYZhy{}}\PY{l+m+mi}{10}\PY{p}{,}\PY{l+m+mi}{5}\PY{p}{,}\PY{l+m+mi}{21}\PY{p}{)}\PY{p}{)}
\end{Verbatim}
\end{tcolorbox}
    }

    \begin{Verbatim}[commandchars=\\\{\}]
0
5
21
    \end{Verbatim}
\begin{eleve}
    \textbf{Exercice 6}

Écrire une fonction \texttt{puissance(x,\ k)} qui renvoie \texttt{x} à
la puissance \texttt{k} (utilisation de l'opérateur \texttt{**} interdit
! évidemment\ldots). On utilisera une boucle \texttt{for} pour faire le
calcul.

On suppose que \(k\geq 0\) et on rappelle que \(x^0 = 1\).

\hypertarget{exemples-et-tests}{%
\paragraph{Exemples et tests:}\label{exemples-et-tests}}

\begin{Shaded}
\begin{Highlighting}[]
\OperatorTok{\textgreater{}\textgreater{}\textgreater{}} \BuiltInTok{print}\NormalTok{(puissance(}\DecValTok{5}\NormalTok{, }\DecValTok{2}\NormalTok{))}
\DecValTok{25}
\OperatorTok{\textgreater{}\textgreater{}\textgreater{}} \BuiltInTok{print}\NormalTok{(puissance(}\DecValTok{100}\NormalTok{, }\DecValTok{0}\NormalTok{))}
\DecValTok{1}
\OperatorTok{\textgreater{}\textgreater{}\textgreater{}} \BuiltInTok{print}\NormalTok{(puissance(}\DecValTok{2}\NormalTok{, }\DecValTok{10}\NormalTok{))}
\DecValTok{1024}
\end{Highlighting}
\end{Shaded}
        
        \end{eleve}
        {\scriptsize
    \begin{tcolorbox}[breakable, size=fbox, boxrule=1pt, pad at break*=1mm,colback=cellbackground, colframe=cellborder]
\prompt{In}{incolor}{9}{\boxspacing}
\begin{Verbatim}[commandchars=\\\{\}]
\PY{k}{def} \PY{n+nf}{puissance}\PY{p}{(}\PY{n}{x}\PY{p}{,} \PY{n}{k}\PY{p}{)}\PY{p}{:}
    \PY{n}{resultat} \PY{o}{=} \PY{l+m+mi}{1}
    \PY{k}{for} \PY{n}{i} \PY{o+ow}{in} \PY{n+nb}{range}\PY{p}{(}\PY{n}{k}\PY{p}{)}\PY{p}{:}
        \PY{n}{resultat} \PY{o}{=} \PY{n}{resultat} \PY{o}{*} \PY{n}{x}
    
    \PY{k}{return} \PY{n}{resultat}


\PY{n+nb}{print}\PY{p}{(}\PY{n}{puissance}\PY{p}{(}\PY{l+m+mi}{5}\PY{p}{,} \PY{l+m+mi}{2}\PY{p}{)}\PY{p}{)}
\PY{n+nb}{print}\PY{p}{(}\PY{n}{puissance}\PY{p}{(}\PY{l+m+mi}{100}\PY{p}{,} \PY{l+m+mi}{0}\PY{p}{)}\PY{p}{)}
\PY{n+nb}{print}\PY{p}{(}\PY{n}{puissance}\PY{p}{(}\PY{l+m+mi}{2}\PY{p}{,} \PY{l+m+mi}{10}\PY{p}{)}\PY{p}{)}
\end{Verbatim}
\end{tcolorbox}
    }

    \begin{Verbatim}[commandchars=\\\{\}]
25
1
1024
    \end{Verbatim}
\begin{eleve}
    \textbf{Exercice 7}

Écrire une fonction \texttt{est\_bissextile(annee)} qui renvoie un
booléen indiquant si l'année \texttt{annee} est une année bissextile.

On rappelle qu'une année bissextile est une année multiple de 4 mais pas
de 100, ou multiple de 400.

\hypertarget{exemples-et-tests}{%
\paragraph{Exemples et tests:}\label{exemples-et-tests}}

\begin{Shaded}
\begin{Highlighting}[]
\OperatorTok{\textgreater{}\textgreater{}\textgreater{}} \BuiltInTok{print}\NormalTok{(est\_bissextile(}\DecValTok{1996}\NormalTok{))}
\VariableTok{True}
\OperatorTok{\textgreater{}\textgreater{}\textgreater{}} \BuiltInTok{print}\NormalTok{(est\_bissextile(}\DecValTok{2024}\NormalTok{))}
\VariableTok{True}
\OperatorTok{\textgreater{}\textgreater{}\textgreater{}} \BuiltInTok{print}\NormalTok{(est\_bissextile(}\DecValTok{2022}\NormalTok{))}
\VariableTok{False}
\end{Highlighting}
\end{Shaded}
        
        \end{eleve}
        {\scriptsize
    \begin{tcolorbox}[breakable, size=fbox, boxrule=1pt, pad at break*=1mm,colback=cellbackground, colframe=cellborder]
\prompt{In}{incolor}{10}{\boxspacing}
\begin{Verbatim}[commandchars=\\\{\}]
\PY{k}{def} \PY{n+nf}{est\PYZus{}bissextile}\PY{p}{(}\PY{n}{annee}\PY{p}{)}\PY{p}{:}
    \PY{n}{est\PYZus{}div\PYZus{}4}   \PY{o}{=} \PY{n}{annee} \PY{o}{\PYZpc{}} \PY{l+m+mi}{4}   \PY{o}{==} \PY{l+m+mi}{0}
    \PY{n}{est\PYZus{}div\PYZus{}100} \PY{o}{=} \PY{n}{annee} \PY{o}{\PYZpc{}} \PY{l+m+mi}{100} \PY{o}{==} \PY{l+m+mi}{0}
    \PY{n}{est\PYZus{}div\PYZus{}400} \PY{o}{=} \PY{n}{annee} \PY{o}{\PYZpc{}} \PY{l+m+mi}{400} \PY{o}{==} \PY{l+m+mi}{0}
    \PY{k}{if} \PY{p}{(}\PY{n}{est\PYZus{}div\PYZus{}4} \PY{o+ow}{and} \PY{o+ow}{not} \PY{n}{est\PYZus{}div\PYZus{}100}\PY{p}{)} \PYZbs{}
        \PY{o+ow}{or} \PY{n}{est\PYZus{}div\PYZus{}400}\PY{p}{:}
        \PY{k}{return} \PY{k+kc}{True}
    
    \PY{k}{return} \PY{k+kc}{False}


\PY{n+nb}{print}\PY{p}{(}\PY{n}{est\PYZus{}bissextile}\PY{p}{(}\PY{l+m+mi}{1996}\PY{p}{)}\PY{p}{)}
\PY{n+nb}{print}\PY{p}{(}\PY{n}{est\PYZus{}bissextile}\PY{p}{(}\PY{l+m+mi}{2024}\PY{p}{)}\PY{p}{)}
\PY{n+nb}{print}\PY{p}{(}\PY{n}{est\PYZus{}bissextile}\PY{p}{(}\PY{l+m+mi}{2022}\PY{p}{)}\PY{p}{)}
\end{Verbatim}
\end{tcolorbox}
    }

    \begin{Verbatim}[commandchars=\\\{\}]
True
True
False
    \end{Verbatim}
\begin{eleve}
    \textbf{Exercice 8}

Écrire une fonctions \texttt{nb\_jours\_annee(annee)} qui renvoie le
nombre de jour de l'année \texttt{annee}. La fonction
\texttt{est\_bissextile} de l'exercice précédent est disponible et vous
pouvez l'utiliser (ce qui est bien pratique quand même).

\hypertarget{exemples-et-tests}{%
\paragraph{Exemples et tests:}\label{exemples-et-tests}}

\begin{Shaded}
\begin{Highlighting}[]
\OperatorTok{\textgreater{}\textgreater{}\textgreater{}} \BuiltInTok{print}\NormalTok{(nb\_jours\_annee(}\DecValTok{1996}\NormalTok{))}
\DecValTok{366}
\OperatorTok{\textgreater{}\textgreater{}\textgreater{}} \BuiltInTok{print}\NormalTok{(nb\_jours\_annee(}\DecValTok{2024}\NormalTok{))}
\DecValTok{366}
\OperatorTok{\textgreater{}\textgreater{}\textgreater{}} \BuiltInTok{print}\NormalTok{(nb\_jours\_annee(}\DecValTok{2022}\NormalTok{))}
\DecValTok{365}
\end{Highlighting}
\end{Shaded}
        
        \end{eleve}
        {\scriptsize
    \begin{tcolorbox}[breakable, size=fbox, boxrule=1pt, pad at break*=1mm,colback=cellbackground, colframe=cellborder]
\prompt{In}{incolor}{11}{\boxspacing}
\begin{Verbatim}[commandchars=\\\{\}]
\PY{k}{def} \PY{n+nf}{nb\PYZus{}jours\PYZus{}annee}\PY{p}{(}\PY{n}{annee}\PY{p}{)}\PY{p}{:}
    \PY{k}{if} \PY{n}{est\PYZus{}bissextile}\PY{p}{(}\PY{n}{annee}\PY{p}{)}\PY{p}{:}
        \PY{k}{return} \PY{l+m+mi}{366}
    \PY{k}{return} \PY{l+m+mi}{365}


\PY{n+nb}{print}\PY{p}{(}\PY{n}{nb\PYZus{}jours\PYZus{}annee}\PY{p}{(}\PY{l+m+mi}{1996}\PY{p}{)}\PY{p}{)}
\PY{n+nb}{print}\PY{p}{(}\PY{n}{nb\PYZus{}jours\PYZus{}annee}\PY{p}{(}\PY{l+m+mi}{2024}\PY{p}{)}\PY{p}{)}
\PY{n+nb}{print}\PY{p}{(}\PY{n}{nb\PYZus{}jours\PYZus{}annee}\PY{p}{(}\PY{l+m+mi}{2022}\PY{p}{)}\PY{p}{)}
\end{Verbatim}
\end{tcolorbox}
    }

    \begin{Verbatim}[commandchars=\\\{\}]
366
366
365
    \end{Verbatim}
\begin{eleve}
    \textbf{Exercice 9}

Écrire une fonction \texttt{nb\_jours\_mois(annee,\ mois)} qui renvoie
le nombre de jours dans le mois \texttt{mois} de l'année \texttt{annee}.
La fonction \texttt{est\_bissextile} de l'exercice précédent est
disponible (ce qui est bien pratique).

On suppose que le mois \texttt{mois} est un entier compris entre
\texttt{1} (pour janvier) et \texttt{12} (décembre).

\hypertarget{exemples-et-tests}{%
\paragraph{Exemples et tests:}\label{exemples-et-tests}}

\begin{Shaded}
\begin{Highlighting}[]
\OperatorTok{\textgreater{}\textgreater{}\textgreater{}} \BuiltInTok{print}\NormalTok{(nb\_jours\_mois(}\DecValTok{1900}\NormalTok{, }\DecValTok{1}\NormalTok{))}
\DecValTok{31}
\OperatorTok{\textgreater{}\textgreater{}\textgreater{}} \BuiltInTok{print}\NormalTok{(nb\_jours\_mois(}\DecValTok{1900}\NormalTok{, }\DecValTok{2}\NormalTok{))}
\DecValTok{28}
\OperatorTok{\textgreater{}\textgreater{}\textgreater{}} \BuiltInTok{print}\NormalTok{(nb\_jours\_mois(}\DecValTok{1904}\NormalTok{, }\DecValTok{2}\NormalTok{))}
\DecValTok{29}
\OperatorTok{\textgreater{}\textgreater{}\textgreater{}} \BuiltInTok{print}\NormalTok{(nb\_jours\_mois(}\DecValTok{2021}\NormalTok{, }\DecValTok{9}\NormalTok{))}
\DecValTok{30}
\end{Highlighting}
\end{Shaded}
        
        \end{eleve}
        {\scriptsize
    \begin{tcolorbox}[breakable, size=fbox, boxrule=1pt, pad at break*=1mm,colback=cellbackground, colframe=cellborder]
\prompt{In}{incolor}{13}{\boxspacing}
\begin{Verbatim}[commandchars=\\\{\}]
\PY{k}{def} \PY{n+nf}{nb\PYZus{}jours\PYZus{}mois}\PY{p}{(}\PY{n}{annee}\PY{p}{,} \PY{n}{mois}\PY{p}{)}\PY{p}{:}
    \PY{k}{if} \PY{n}{mois} \PY{o}{==} \PY{l+m+mi}{2}\PY{p}{:}
        \PY{k}{if} \PY{n}{est\PYZus{}bissextile}\PY{p}{(}\PY{n}{annee}\PY{p}{)}\PY{p}{:}
            \PY{k}{return} \PY{l+m+mi}{29}
        \PY{k}{else}\PY{p}{:}
            \PY{k}{return} \PY{l+m+mi}{28}
    
    \PY{k}{if} \PY{n}{mois} \PY{o}{==} \PY{l+m+mi}{4} \PY{o+ow}{or} \PY{n}{mois} \PY{o}{==} \PY{l+m+mi}{6} \PY{o+ow}{or} \PY{n}{mois} \PY{o}{==} \PY{l+m+mi}{9} \PY{o+ow}{or} \PY{n}{mois} \PY{o}{==} \PY{l+m+mi}{11}\PY{p}{:}
        \PY{k}{return} \PY{l+m+mi}{30}
    
    \PY{k}{return} \PY{l+m+mi}{31}


\PY{n+nb}{print}\PY{p}{(}\PY{n}{nb\PYZus{}jours\PYZus{}mois}\PY{p}{(}\PY{l+m+mi}{1900}\PY{p}{,} \PY{l+m+mi}{1}\PY{p}{)}\PY{p}{)}
\PY{n+nb}{print}\PY{p}{(}\PY{n}{nb\PYZus{}jours\PYZus{}mois}\PY{p}{(}\PY{l+m+mi}{1900}\PY{p}{,} \PY{l+m+mi}{2}\PY{p}{)}\PY{p}{)}
\PY{n+nb}{print}\PY{p}{(}\PY{n}{nb\PYZus{}jours\PYZus{}mois}\PY{p}{(}\PY{l+m+mi}{1904}\PY{p}{,} \PY{l+m+mi}{2}\PY{p}{)}\PY{p}{)}
\PY{n+nb}{print}\PY{p}{(}\PY{n}{nb\PYZus{}jours\PYZus{}mois}\PY{p}{(}\PY{l+m+mi}{2021}\PY{p}{,} \PY{l+m+mi}{9}\PY{p}{)}\PY{p}{)}
\end{Verbatim}
\end{tcolorbox}
    }

    \begin{Verbatim}[commandchars=\\\{\}]
31
28
29
30
    \end{Verbatim}
\begin{eleve}
    \textbf{Exercice 10}

En utilisant les fonctions \texttt{nb\_jours\_annnee} et
\texttt{nb\_jours\_mois} (qui sont disponibles), écrire une fonction
\texttt{nb\_jours(j\_debut,\ m\_debut,\ a\_debut,\ j\_fin,\ m\_fin,\ a\_fin)}
qui renvoie le nombre de jours compris entre deux dates données
(incluses).

\hypertarget{exemples-et-tests}{%
\paragraph{Exemples et tests:}\label{exemples-et-tests}}

\begin{Shaded}
\begin{Highlighting}[]
\OperatorTok{\textgreater{}\textgreater{}\textgreater{}} \BuiltInTok{print}\NormalTok{(nb\_jours(}\DecValTok{1}\NormalTok{,}\DecValTok{1}\NormalTok{,}\DecValTok{2021}\NormalTok{,}\DecValTok{15}\NormalTok{,}\DecValTok{1}\NormalTok{,}\DecValTok{2021}\NormalTok{))}
\DecValTok{15}
\OperatorTok{\textgreater{}\textgreater{}\textgreater{}} \BuiltInTok{print}\NormalTok{(nb\_jours(}\DecValTok{1}\NormalTok{,}\DecValTok{1}\NormalTok{,}\DecValTok{2021}\NormalTok{,}\DecValTok{31}\NormalTok{,}\DecValTok{12}\NormalTok{,}\DecValTok{2021}\NormalTok{))}
\DecValTok{365}
\OperatorTok{\textgreater{}\textgreater{}\textgreater{}} \BuiltInTok{print}\NormalTok{(nb\_jours(}\DecValTok{1}\NormalTok{,}\DecValTok{1}\NormalTok{,}\DecValTok{2024}\NormalTok{,}\DecValTok{31}\NormalTok{,}\DecValTok{12}\NormalTok{,}\DecValTok{2024}\NormalTok{))}
\DecValTok{366}
\OperatorTok{\textgreater{}\textgreater{}\textgreater{}} \BuiltInTok{print}\NormalTok{(nb\_jours(}\DecValTok{1}\NormalTok{,}\DecValTok{1}\NormalTok{,}\DecValTok{1970}\NormalTok{,}\DecValTok{31}\NormalTok{,}\DecValTok{12}\NormalTok{,}\DecValTok{2021}\NormalTok{))}
\DecValTok{18993}
\end{Highlighting}
\end{Shaded}
        
        \end{eleve}
        {\scriptsize
    \begin{tcolorbox}[breakable, size=fbox, boxrule=1pt, pad at break*=1mm,colback=cellbackground, colframe=cellborder]
\prompt{In}{incolor}{18}{\boxspacing}
\begin{Verbatim}[commandchars=\\\{\}]
\PY{k}{def} \PY{n+nf}{nb\PYZus{}jours}\PY{p}{(}\PY{n}{j\PYZus{}debut}\PY{p}{,} \PY{n}{m\PYZus{}debut}\PY{p}{,} \PY{n}{a\PYZus{}debut}\PY{p}{,} \PY{n}{j\PYZus{}fin}\PY{p}{,} \PY{n}{m\PYZus{}fin}\PY{p}{,} \PY{n}{a\PYZus{}fin}\PY{p}{)}\PY{p}{:}
    \PY{c+c1}{\PYZsh{} nombre de jours avant la date de}
    \PY{c+c1}{\PYZsh{} départ pour compléter la première année}
    \PY{n}{nbjours\PYZus{}avant} \PY{o}{=} \PY{l+m+mi}{0}
    \PY{n}{nbjours\PYZus{}avant} \PY{o}{=} \PY{n}{nbjours\PYZus{}avant} \PY{o}{+} \PY{n}{j\PYZus{}debut} \PY{o}{\PYZhy{}} \PY{l+m+mi}{1}
    \PY{k}{for} \PY{n}{m} \PY{o+ow}{in} \PY{n+nb}{range} \PY{p}{(}\PY{l+m+mi}{1}\PY{p}{,} \PY{n}{m\PYZus{}debut}\PY{p}{)}\PY{p}{:}
        \PY{n}{nbjours\PYZus{}avant} \PY{o}{=} \PY{n}{nbjours\PYZus{}avant} \PY{o}{+} \PY{n}{nb\PYZus{}jours\PYZus{}mois}\PY{p}{(}\PY{n}{a\PYZus{}debut}\PY{p}{,} \PY{n}{m}\PY{p}{)}

    \PY{c+c1}{\PYZsh{} nombre de jours après la date de}
    \PY{c+c1}{\PYZsh{} fin nécessaires pour compléter la dernière année}
    \PY{n}{nbjours\PYZus{}apres} \PY{o}{=} \PY{l+m+mi}{0}
    \PY{n}{nbjours\PYZus{}apres} \PY{o}{=} \PY{n}{nbjours\PYZus{}apres} \PY{o}{+} \PY{n}{nb\PYZus{}jours\PYZus{}mois}\PY{p}{(}\PY{n}{a\PYZus{}fin}\PY{p}{,} \PY{n}{m\PYZus{}fin}\PY{p}{)} \PY{o}{\PYZhy{}} \PY{n}{j\PYZus{}fin}
    \PY{k}{for} \PY{n}{m} \PY{o+ow}{in} \PY{n+nb}{range}\PY{p}{(}\PY{n}{m\PYZus{}fin} \PY{o}{+} \PY{l+m+mi}{1}\PY{p}{,} \PY{l+m+mi}{13}\PY{p}{)}\PY{p}{:}
        \PY{n}{nbjours\PYZus{}apres} \PY{o}{=} \PY{n}{nbjours\PYZus{}apres} \PY{o}{+} \PY{n}{nb\PYZus{}jours\PYZus{}mois}\PY{p}{(}\PY{n}{a\PYZus{}fin}\PY{p}{,} \PY{n}{m}\PY{p}{)}

    \PY{c+c1}{\PYZsh{} nombres de jours entre l\PYZsq{}année de début et l\PYZsq{}année de fin}
    \PY{n}{total} \PY{o}{=} \PY{l+m+mi}{0}
    \PY{k}{for} \PY{n}{a} \PY{o+ow}{in} \PY{n+nb}{range}\PY{p}{(}\PY{n}{a\PYZus{}debut}\PY{p}{,} \PY{n}{a\PYZus{}fin}\PY{o}{+}\PY{l+m+mi}{1}\PY{p}{)}\PY{p}{:}
        \PY{n}{total} \PY{o}{=} \PY{n}{total} \PY{o}{+} \PY{n}{nb\PYZus{}jours\PYZus{}annee}\PY{p}{(}\PY{n}{a}\PY{p}{)}

    \PY{c+c1}{\PYZsh{} retirer les jours en trop des}
    \PY{c+c1}{\PYZsh{} premières et dernières années}
    \PY{k}{return} \PY{n}{total} \PY{o}{\PYZhy{}} \PY{n}{nbjours\PYZus{}avant} \PY{o}{\PYZhy{}} \PY{n}{nbjours\PYZus{}apres}
\end{Verbatim}
\end{tcolorbox}
    }

        {\scriptsize
    \begin{tcolorbox}[breakable, size=fbox, boxrule=1pt, pad at break*=1mm,colback=cellbackground, colframe=cellborder]
\prompt{In}{incolor}{16}{\boxspacing}
\begin{Verbatim}[commandchars=\\\{\}]
\PY{n+nb}{print}\PY{p}{(}\PY{n}{nb\PYZus{}jours}\PY{p}{(}\PY{l+m+mi}{1}\PY{p}{,}\PY{l+m+mi}{1}\PY{p}{,}\PY{l+m+mi}{2021}\PY{p}{,}\PY{l+m+mi}{15}\PY{p}{,}\PY{l+m+mi}{1}\PY{p}{,}\PY{l+m+mi}{2021}\PY{p}{)}\PY{p}{)}
\PY{n+nb}{print}\PY{p}{(}\PY{n}{nb\PYZus{}jours}\PY{p}{(}\PY{l+m+mi}{1}\PY{p}{,}\PY{l+m+mi}{1}\PY{p}{,}\PY{l+m+mi}{2021}\PY{p}{,}\PY{l+m+mi}{31}\PY{p}{,}\PY{l+m+mi}{12}\PY{p}{,}\PY{l+m+mi}{2021}\PY{p}{)}\PY{p}{)}
\PY{n+nb}{print}\PY{p}{(}\PY{n}{nb\PYZus{}jours}\PY{p}{(}\PY{l+m+mi}{1}\PY{p}{,}\PY{l+m+mi}{1}\PY{p}{,}\PY{l+m+mi}{2024}\PY{p}{,}\PY{l+m+mi}{31}\PY{p}{,}\PY{l+m+mi}{12}\PY{p}{,}\PY{l+m+mi}{2024}\PY{p}{)}\PY{p}{)}
\PY{n+nb}{print}\PY{p}{(}\PY{n}{nb\PYZus{}jours}\PY{p}{(}\PY{l+m+mi}{1}\PY{p}{,}\PY{l+m+mi}{1}\PY{p}{,}\PY{l+m+mi}{1970}\PY{p}{,}\PY{l+m+mi}{31}\PY{p}{,}\PY{l+m+mi}{12}\PY{p}{,}\PY{l+m+mi}{2021}\PY{p}{)}\PY{p}{)}
\end{Verbatim}
\end{tcolorbox}
    }

    \begin{Verbatim}[commandchars=\\\{\}]
15
365
366
18993
    \end{Verbatim}


    % Add a bibliography block to the postdoc
    
    
    
\end{document}
