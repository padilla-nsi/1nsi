\documentclass[a4paper,17pt]{extarticle}


    \usepackage[sfdefault, condensed]{roboto} % police d'écriture plus moderne
\usepackage[french]{babel} % francisation
\usepackage[parfill]{parskip} %suppression indentation

\usepackage{fancyhdr}
\usepackage{multicol}

% figure non flotantes
\usepackage{float}
\let\origfigure\figure
\let\endorigfigure\endfigure
\renewenvironment{figure}[1][2] {
    \expandafter\origfigure\expandafter[H]
} {
    \endorigfigure
}

% mois/année
\usepackage{datetime}
\newdateformat{monthyeardate}{%
  \monthname[\THEMONTH] \THEYEAR}

% couleurs perso
\usepackage[table]{xcolor}
\definecolor{deepblue}{rgb}{0.3,0.3,0.8}
\definecolor{darkblue}{rgb}{0,0,0.3}
\definecolor{deepred}{rgb}{0.6,0,0}
\definecolor{iremred}{RGB}{204,35,50}
\definecolor{deepgreen}{rgb}{0,0.6,0}
\definecolor{backcolor}{rgb}{0.98,0.95,0.95}
\definecolor{grisClair}{rgb}{0.95,0.95,0.95}
\definecolor{orangeamu}{RGB}{250,178,11}
\definecolor{noiramu}{RGB}{35,31,32}
\definecolor{bleuamu}{RGB}{20,118,198}
\definecolor{bleuamudark}{RGB}{15,90,150}
\definecolor{cyanamu}{RGB}{77,198,244}


\usepackage{/home/bouscadilla/Documents/Code/nbconvert/template/latex/pdf_solution/xeboiboites}
%
% exemple
\newbreakabletheorem[
    small box style={fill=deepblue!90,draw=deepblue!15, rounded corners,line width=1pt},%
    big box style={fill=deepblue!5,draw=deepblue!15,thick,rounded corners,line width=1pt},%
    headfont={\color{white}\bfseries}
        ]{exemple}{Exemple}{}%{counterCo}
%
% remarque
\newbreakabletheorem[
    small box style={draw=ansi-green-intense!100,line width=2pt,fill=ansi-green-intense!0,rounded corners,decoration=penciline, decorate},%
	big box style={color=ansi-green-intense!90,fill=ansi-green-intense!10,thick,decoration={penciline},decorate},
    broken edges={draw=ansi-green-intense!90,thick,fill=orange!20!black!5, decoration={random steps, segment length=.5cm,amplitude=1.3mm},decorate},%
    other edges={decoration=penciline,decorate,thick},%
    headfont={\color{ansi-green-intense}\large\scshape\bfseries}
    ]{remarque}{Remarque}{}%{counterCa}
%
% formule (sans titre)
\newboxedequation[%
    big box style={fill=cyanamu!10,draw=cyanamu!100,thick,decoration=penciline,decorate}]%
    {form}
%
% Réponse
\newbreakabletheorem[
    small box style={fill=bleuamu!100, draw=bleuamu!60, line width=1pt,rounded corners,decorate},%
    big box style={fill=bleuamu!10,draw=bleuamu!30,thick,rounded corners,decorate},
    headfont={\color{white}\large\scshape\bfseries}
        ]{reponse}{Correction}{}
%

%
% À retenir
%\newbreakabletheorem[
%    small box style={fill=deepred!100, draw=deepred!80, line width=1pt,rounded corners,decorate},%
%    big box style={fill=deepred!10,draw=deepred!50,thick,rounded corners,decorate},
%    headfont={\color{white}\large\scshape\bfseries}
%        ]{retenir}{À retenir}{}
%
\newboxedequation[%
    big box style={fill=deepred!10,draw=deepred!0,thick,decoration=penciline,decorate}]%
    {retenir}



% astuce
\newspanning[
    image=/home/bouscadilla/Documents/Code/nbconvert/template/latex/pdf_solution/fig-idee,headfont=\bfseries,
    spanning style={very thick,decoration=penciline,decorate}
    ]{astuce}{Astuce}{}
%
% activité

\newcounter{counterCa}
\newbreakabletheorem[
    small box style={draw=orangeamu!100,line width=2pt,fill=orangeamu!100,rounded corners,decoration=penciline, decorate},%
	big box style={color=orangeamu!100,fill=orangeamu!5,thick,decoration={penciline},decorate},
    broken edges={draw=orangeamu!100,thick,fill=orangeamu!100, decoration={random steps, segment length=.5cm,amplitude=1.3mm},decorate},%
    other edges={decoration=penciline,decorate,thick},%
    headfont={\color{white}\large\scshape\bfseries}
    ]{activite}{\adjustimage{height=1cm, valign=m}{/home/bouscadilla/Documents/Code/nbconvert/template/latex/pdf_solution/papier_eleve_investigation.png}%
    Activité}{counterCa}
%   
%   environnement élève
%
\newenvironment{eleve}%
%{\begin{activite}\large\\} % écrire plus gros
{\begin{activite}\color{noiramu}\\[-0.5cm]}
{\end{activite}}

\newenvironment{formule}%
%{\begin{activite}\large\\} % écrire plus gros
{\begin{form}\color{bleuamu}}
{\end{form}}


\usepackage[breakable]{tcolorbox}
    \usepackage{parskip} % Stop auto-indenting (to mimic markdown behaviour)
    
    \usepackage{iftex}
    \ifPDFTeX
    	\usepackage[T1]{fontenc}
    	\usepackage{mathpazo}
    \else
    	\usepackage{fontspec}
    \fi

    % Basic figure setup, for now with no caption control since it's done
    % automatically by Pandoc (which extracts ![](path) syntax from Markdown).
    \usepackage{graphicx}
    % Maintain compatibility with old templates. Remove in nbconvert 6.0
    \let\Oldincludegraphics\includegraphics
    % Ensure that by default, figures have no caption (until we provide a
    % proper Figure object with a Caption API and a way to capture that
    % in the conversion process - todo).
    \usepackage{caption}
    \DeclareCaptionFormat{nocaption}{}
    \captionsetup{format=nocaption,aboveskip=0pt,belowskip=0pt}

    \usepackage[Export]{adjustbox} % Used to constrain images to a maximum size
    \adjustboxset{max size={0.9\linewidth}{0.9\paperheight}}
    \usepackage{float}
    \floatplacement{figure}{H} % forces figures to be placed at the correct location
    \usepackage{xcolor} % Allow colors to be defined
    \usepackage{enumerate} % Needed for markdown enumerations to work
    \usepackage{geometry} % Used to adjust the document margins
    \usepackage{amsmath} % Equations
    \usepackage{amssymb} % Equations
    \usepackage{textcomp} % defines textquotesingle
    % Hack from http://tex.stackexchange.com/a/47451/13684:
    \AtBeginDocument{%
        \def\PYZsq{\textquotesingle}% Upright quotes in Pygmentized code
    }
    \usepackage{upquote} % Upright quotes for verbatim code
    \usepackage{eurosym} % defines \euro
    \usepackage[mathletters]{ucs} % Extended unicode (utf-8) support
    \usepackage{fancyvrb} % verbatim replacement that allows latex

    % The hyperref package gives us a pdf with properly built
    % internal navigation ('pdf bookmarks' for the table of contents,
    % internal cross-reference links, web links for URLs, etc.)
    \usepackage{hyperref}
    % The default LaTeX title has an obnoxious amount of whitespace. By default,
    % titling removes some of it. It also provides customization options.
    \usepackage{titling}
    \usepackage{longtable} % longtable support required by pandoc >1.10
    \usepackage{booktabs}  % table support for pandoc > 1.12.2
    \usepackage[inline]{enumitem} % IRkernel/repr support (it uses the enumerate* environment)
    \usepackage[normalem]{ulem} % ulem is needed to support strikethroughs (\sout)
                                % normalem makes italics be italics, not underlines
    \usepackage{mathrsfs}
    

    
    % Colors for the hyperref package
    \definecolor{urlcolor}{rgb}{0,.145,.698}
    \definecolor{linkcolor}{rgb}{.71,0.21,0.01}
    \definecolor{citecolor}{rgb}{.12,.54,.11}

    % ANSI colors
    \definecolor{ansi-black}{HTML}{3E424D}
    \definecolor{ansi-black-intense}{HTML}{282C36}
    \definecolor{ansi-red}{HTML}{E75C58}
    \definecolor{ansi-red-intense}{HTML}{B22B31}
    \definecolor{ansi-green}{HTML}{00A250}
    \definecolor{ansi-green-intense}{HTML}{007427}
    \definecolor{ansi-yellow}{HTML}{DDB62B}
    \definecolor{ansi-yellow-intense}{HTML}{B27D12}
    \definecolor{ansi-blue}{HTML}{208FFB}
    \definecolor{ansi-blue-intense}{HTML}{0065CA}
    \definecolor{ansi-magenta}{HTML}{D160C4}
    \definecolor{ansi-magenta-intense}{HTML}{A03196}
    \definecolor{ansi-cyan}{HTML}{60C6C8}
    \definecolor{ansi-cyan-intense}{HTML}{258F8F}
    \definecolor{ansi-white}{HTML}{C5C1B4}
    \definecolor{ansi-white-intense}{HTML}{A1A6B2}
    \definecolor{ansi-default-inverse-fg}{HTML}{FFFFFF}
    \definecolor{ansi-default-inverse-bg}{HTML}{000000}

    % commands and environments needed by pandoc snippets
    % extracted from the output of `pandoc -s`
    \providecommand{\tightlist}{%
      \setlength{\itemsep}{0pt}\setlength{\parskip}{0pt}}
    \DefineVerbatimEnvironment{Highlighting}{Verbatim}{commandchars=\\\{\}}
    % Add ',fontsize=\small' for more characters per line
    \newenvironment{Shaded}{}{}
    \newcommand{\KeywordTok}[1]{\textcolor[rgb]{0.00,0.44,0.13}{\textbf{{#1}}}}
    \newcommand{\DataTypeTok}[1]{\textcolor[rgb]{0.56,0.13,0.00}{{#1}}}
    \newcommand{\DecValTok}[1]{\textcolor[rgb]{0.25,0.63,0.44}{{#1}}}
    \newcommand{\BaseNTok}[1]{\textcolor[rgb]{0.25,0.63,0.44}{{#1}}}
    \newcommand{\FloatTok}[1]{\textcolor[rgb]{0.25,0.63,0.44}{{#1}}}
    \newcommand{\CharTok}[1]{\textcolor[rgb]{0.25,0.44,0.63}{{#1}}}
    \newcommand{\StringTok}[1]{\textcolor[rgb]{0.25,0.44,0.63}{{#1}}}
    \newcommand{\CommentTok}[1]{\textcolor[rgb]{0.38,0.63,0.69}{\textit{{#1}}}}
    \newcommand{\OtherTok}[1]{\textcolor[rgb]{0.00,0.44,0.13}{{#1}}}
    \newcommand{\AlertTok}[1]{\textcolor[rgb]{1.00,0.00,0.00}{\textbf{{#1}}}}
    \newcommand{\FunctionTok}[1]{\textcolor[rgb]{0.02,0.16,0.49}{{#1}}}
    \newcommand{\RegionMarkerTok}[1]{{#1}}
    \newcommand{\ErrorTok}[1]{\textcolor[rgb]{1.00,0.00,0.00}{\textbf{{#1}}}}
    \newcommand{\NormalTok}[1]{{#1}}
    
    % Additional commands for more recent versions of Pandoc
    \newcommand{\ConstantTok}[1]{\textcolor[rgb]{0.53,0.00,0.00}{{#1}}}
    \newcommand{\SpecialCharTok}[1]{\textcolor[rgb]{0.25,0.44,0.63}{{#1}}}
    \newcommand{\VerbatimStringTok}[1]{\textcolor[rgb]{0.25,0.44,0.63}{{#1}}}
    \newcommand{\SpecialStringTok}[1]{\textcolor[rgb]{0.73,0.40,0.53}{{#1}}}
    \newcommand{\ImportTok}[1]{{#1}}
    \newcommand{\DocumentationTok}[1]{\textcolor[rgb]{0.73,0.13,0.13}{\textit{{#1}}}}
    \newcommand{\AnnotationTok}[1]{\textcolor[rgb]{0.38,0.63,0.69}{\textbf{\textit{{#1}}}}}
    \newcommand{\CommentVarTok}[1]{\textcolor[rgb]{0.38,0.63,0.69}{\textbf{\textit{{#1}}}}}
    \newcommand{\VariableTok}[1]{\textcolor[rgb]{0.10,0.09,0.49}{{#1}}}
    \newcommand{\ControlFlowTok}[1]{\textcolor[rgb]{0.00,0.44,0.13}{\textbf{{#1}}}}
    \newcommand{\OperatorTok}[1]{\textcolor[rgb]{0.40,0.40,0.40}{{#1}}}
    \newcommand{\BuiltInTok}[1]{{#1}}
    \newcommand{\ExtensionTok}[1]{{#1}}
    \newcommand{\PreprocessorTok}[1]{\textcolor[rgb]{0.74,0.48,0.00}{{#1}}}
    \newcommand{\AttributeTok}[1]{\textcolor[rgb]{0.49,0.56,0.16}{{#1}}}
    \newcommand{\InformationTok}[1]{\textcolor[rgb]{0.38,0.63,0.69}{\textbf{\textit{{#1}}}}}
    \newcommand{\WarningTok}[1]{\textcolor[rgb]{0.38,0.63,0.69}{\textbf{\textit{{#1}}}}}
    
    
    % Define a nice break command that doesn't care if a line doesn't already
    % exist.
    \def\br{\hspace*{\fill} \\* }
    % Math Jax compatibility definitions
    \def\gt{>}
    \def\lt{<}
    \let\Oldtex\TeX
    \let\Oldlatex\LaTeX
    \renewcommand{\TeX}{\textrm{\Oldtex}}
    \renewcommand{\LaTeX}{\textrm{\Oldlatex}}
    % Document parameters
    % Document title
    \title{2-2---bases-python-act}
    
    
    
    
    
% Pygments definitions
\makeatletter
\def\PY@reset{\let\PY@it=\relax \let\PY@bf=\relax%
    \let\PY@ul=\relax \let\PY@tc=\relax%
    \let\PY@bc=\relax \let\PY@ff=\relax}
\def\PY@tok#1{\csname PY@tok@#1\endcsname}
\def\PY@toks#1+{\ifx\relax#1\empty\else%
    \PY@tok{#1}\expandafter\PY@toks\fi}
\def\PY@do#1{\PY@bc{\PY@tc{\PY@ul{%
    \PY@it{\PY@bf{\PY@ff{#1}}}}}}}
\def\PY#1#2{\PY@reset\PY@toks#1+\relax+\PY@do{#2}}

\expandafter\def\csname PY@tok@w\endcsname{\def\PY@tc##1{\textcolor[rgb]{0.73,0.73,0.73}{##1}}}
\expandafter\def\csname PY@tok@c\endcsname{\let\PY@it=\textit\def\PY@tc##1{\textcolor[rgb]{0.25,0.50,0.50}{##1}}}
\expandafter\def\csname PY@tok@cp\endcsname{\def\PY@tc##1{\textcolor[rgb]{0.74,0.48,0.00}{##1}}}
\expandafter\def\csname PY@tok@k\endcsname{\let\PY@bf=\textbf\def\PY@tc##1{\textcolor[rgb]{0.00,0.50,0.00}{##1}}}
\expandafter\def\csname PY@tok@kp\endcsname{\def\PY@tc##1{\textcolor[rgb]{0.00,0.50,0.00}{##1}}}
\expandafter\def\csname PY@tok@kt\endcsname{\def\PY@tc##1{\textcolor[rgb]{0.69,0.00,0.25}{##1}}}
\expandafter\def\csname PY@tok@o\endcsname{\def\PY@tc##1{\textcolor[rgb]{0.40,0.40,0.40}{##1}}}
\expandafter\def\csname PY@tok@ow\endcsname{\let\PY@bf=\textbf\def\PY@tc##1{\textcolor[rgb]{0.67,0.13,1.00}{##1}}}
\expandafter\def\csname PY@tok@nb\endcsname{\def\PY@tc##1{\textcolor[rgb]{0.00,0.50,0.00}{##1}}}
\expandafter\def\csname PY@tok@nf\endcsname{\def\PY@tc##1{\textcolor[rgb]{0.00,0.00,1.00}{##1}}}
\expandafter\def\csname PY@tok@nc\endcsname{\let\PY@bf=\textbf\def\PY@tc##1{\textcolor[rgb]{0.00,0.00,1.00}{##1}}}
\expandafter\def\csname PY@tok@nn\endcsname{\let\PY@bf=\textbf\def\PY@tc##1{\textcolor[rgb]{0.00,0.00,1.00}{##1}}}
\expandafter\def\csname PY@tok@ne\endcsname{\let\PY@bf=\textbf\def\PY@tc##1{\textcolor[rgb]{0.82,0.25,0.23}{##1}}}
\expandafter\def\csname PY@tok@nv\endcsname{\def\PY@tc##1{\textcolor[rgb]{0.10,0.09,0.49}{##1}}}
\expandafter\def\csname PY@tok@no\endcsname{\def\PY@tc##1{\textcolor[rgb]{0.53,0.00,0.00}{##1}}}
\expandafter\def\csname PY@tok@nl\endcsname{\def\PY@tc##1{\textcolor[rgb]{0.63,0.63,0.00}{##1}}}
\expandafter\def\csname PY@tok@ni\endcsname{\let\PY@bf=\textbf\def\PY@tc##1{\textcolor[rgb]{0.60,0.60,0.60}{##1}}}
\expandafter\def\csname PY@tok@na\endcsname{\def\PY@tc##1{\textcolor[rgb]{0.49,0.56,0.16}{##1}}}
\expandafter\def\csname PY@tok@nt\endcsname{\let\PY@bf=\textbf\def\PY@tc##1{\textcolor[rgb]{0.00,0.50,0.00}{##1}}}
\expandafter\def\csname PY@tok@nd\endcsname{\def\PY@tc##1{\textcolor[rgb]{0.67,0.13,1.00}{##1}}}
\expandafter\def\csname PY@tok@s\endcsname{\def\PY@tc##1{\textcolor[rgb]{0.73,0.13,0.13}{##1}}}
\expandafter\def\csname PY@tok@sd\endcsname{\let\PY@it=\textit\def\PY@tc##1{\textcolor[rgb]{0.73,0.13,0.13}{##1}}}
\expandafter\def\csname PY@tok@si\endcsname{\let\PY@bf=\textbf\def\PY@tc##1{\textcolor[rgb]{0.73,0.40,0.53}{##1}}}
\expandafter\def\csname PY@tok@se\endcsname{\let\PY@bf=\textbf\def\PY@tc##1{\textcolor[rgb]{0.73,0.40,0.13}{##1}}}
\expandafter\def\csname PY@tok@sr\endcsname{\def\PY@tc##1{\textcolor[rgb]{0.73,0.40,0.53}{##1}}}
\expandafter\def\csname PY@tok@ss\endcsname{\def\PY@tc##1{\textcolor[rgb]{0.10,0.09,0.49}{##1}}}
\expandafter\def\csname PY@tok@sx\endcsname{\def\PY@tc##1{\textcolor[rgb]{0.00,0.50,0.00}{##1}}}
\expandafter\def\csname PY@tok@m\endcsname{\def\PY@tc##1{\textcolor[rgb]{0.40,0.40,0.40}{##1}}}
\expandafter\def\csname PY@tok@gh\endcsname{\let\PY@bf=\textbf\def\PY@tc##1{\textcolor[rgb]{0.00,0.00,0.50}{##1}}}
\expandafter\def\csname PY@tok@gu\endcsname{\let\PY@bf=\textbf\def\PY@tc##1{\textcolor[rgb]{0.50,0.00,0.50}{##1}}}
\expandafter\def\csname PY@tok@gd\endcsname{\def\PY@tc##1{\textcolor[rgb]{0.63,0.00,0.00}{##1}}}
\expandafter\def\csname PY@tok@gi\endcsname{\def\PY@tc##1{\textcolor[rgb]{0.00,0.63,0.00}{##1}}}
\expandafter\def\csname PY@tok@gr\endcsname{\def\PY@tc##1{\textcolor[rgb]{1.00,0.00,0.00}{##1}}}
\expandafter\def\csname PY@tok@ge\endcsname{\let\PY@it=\textit}
\expandafter\def\csname PY@tok@gs\endcsname{\let\PY@bf=\textbf}
\expandafter\def\csname PY@tok@gp\endcsname{\let\PY@bf=\textbf\def\PY@tc##1{\textcolor[rgb]{0.00,0.00,0.50}{##1}}}
\expandafter\def\csname PY@tok@go\endcsname{\def\PY@tc##1{\textcolor[rgb]{0.53,0.53,0.53}{##1}}}
\expandafter\def\csname PY@tok@gt\endcsname{\def\PY@tc##1{\textcolor[rgb]{0.00,0.27,0.87}{##1}}}
\expandafter\def\csname PY@tok@err\endcsname{\def\PY@bc##1{\setlength{\fboxsep}{0pt}\fcolorbox[rgb]{1.00,0.00,0.00}{1,1,1}{\strut ##1}}}
\expandafter\def\csname PY@tok@kc\endcsname{\let\PY@bf=\textbf\def\PY@tc##1{\textcolor[rgb]{0.00,0.50,0.00}{##1}}}
\expandafter\def\csname PY@tok@kd\endcsname{\let\PY@bf=\textbf\def\PY@tc##1{\textcolor[rgb]{0.00,0.50,0.00}{##1}}}
\expandafter\def\csname PY@tok@kn\endcsname{\let\PY@bf=\textbf\def\PY@tc##1{\textcolor[rgb]{0.00,0.50,0.00}{##1}}}
\expandafter\def\csname PY@tok@kr\endcsname{\let\PY@bf=\textbf\def\PY@tc##1{\textcolor[rgb]{0.00,0.50,0.00}{##1}}}
\expandafter\def\csname PY@tok@bp\endcsname{\def\PY@tc##1{\textcolor[rgb]{0.00,0.50,0.00}{##1}}}
\expandafter\def\csname PY@tok@fm\endcsname{\def\PY@tc##1{\textcolor[rgb]{0.00,0.00,1.00}{##1}}}
\expandafter\def\csname PY@tok@vc\endcsname{\def\PY@tc##1{\textcolor[rgb]{0.10,0.09,0.49}{##1}}}
\expandafter\def\csname PY@tok@vg\endcsname{\def\PY@tc##1{\textcolor[rgb]{0.10,0.09,0.49}{##1}}}
\expandafter\def\csname PY@tok@vi\endcsname{\def\PY@tc##1{\textcolor[rgb]{0.10,0.09,0.49}{##1}}}
\expandafter\def\csname PY@tok@vm\endcsname{\def\PY@tc##1{\textcolor[rgb]{0.10,0.09,0.49}{##1}}}
\expandafter\def\csname PY@tok@sa\endcsname{\def\PY@tc##1{\textcolor[rgb]{0.73,0.13,0.13}{##1}}}
\expandafter\def\csname PY@tok@sb\endcsname{\def\PY@tc##1{\textcolor[rgb]{0.73,0.13,0.13}{##1}}}
\expandafter\def\csname PY@tok@sc\endcsname{\def\PY@tc##1{\textcolor[rgb]{0.73,0.13,0.13}{##1}}}
\expandafter\def\csname PY@tok@dl\endcsname{\def\PY@tc##1{\textcolor[rgb]{0.73,0.13,0.13}{##1}}}
\expandafter\def\csname PY@tok@s2\endcsname{\def\PY@tc##1{\textcolor[rgb]{0.73,0.13,0.13}{##1}}}
\expandafter\def\csname PY@tok@sh\endcsname{\def\PY@tc##1{\textcolor[rgb]{0.73,0.13,0.13}{##1}}}
\expandafter\def\csname PY@tok@s1\endcsname{\def\PY@tc##1{\textcolor[rgb]{0.73,0.13,0.13}{##1}}}
\expandafter\def\csname PY@tok@mb\endcsname{\def\PY@tc##1{\textcolor[rgb]{0.40,0.40,0.40}{##1}}}
\expandafter\def\csname PY@tok@mf\endcsname{\def\PY@tc##1{\textcolor[rgb]{0.40,0.40,0.40}{##1}}}
\expandafter\def\csname PY@tok@mh\endcsname{\def\PY@tc##1{\textcolor[rgb]{0.40,0.40,0.40}{##1}}}
\expandafter\def\csname PY@tok@mi\endcsname{\def\PY@tc##1{\textcolor[rgb]{0.40,0.40,0.40}{##1}}}
\expandafter\def\csname PY@tok@il\endcsname{\def\PY@tc##1{\textcolor[rgb]{0.40,0.40,0.40}{##1}}}
\expandafter\def\csname PY@tok@mo\endcsname{\def\PY@tc##1{\textcolor[rgb]{0.40,0.40,0.40}{##1}}}
\expandafter\def\csname PY@tok@ch\endcsname{\let\PY@it=\textit\def\PY@tc##1{\textcolor[rgb]{0.25,0.50,0.50}{##1}}}
\expandafter\def\csname PY@tok@cm\endcsname{\let\PY@it=\textit\def\PY@tc##1{\textcolor[rgb]{0.25,0.50,0.50}{##1}}}
\expandafter\def\csname PY@tok@cpf\endcsname{\let\PY@it=\textit\def\PY@tc##1{\textcolor[rgb]{0.25,0.50,0.50}{##1}}}
\expandafter\def\csname PY@tok@c1\endcsname{\let\PY@it=\textit\def\PY@tc##1{\textcolor[rgb]{0.25,0.50,0.50}{##1}}}
\expandafter\def\csname PY@tok@cs\endcsname{\let\PY@it=\textit\def\PY@tc##1{\textcolor[rgb]{0.25,0.50,0.50}{##1}}}

\def\PYZbs{\char`\\}
\def\PYZus{\char`\_}
\def\PYZob{\char`\{}
\def\PYZcb{\char`\}}
\def\PYZca{\char`\^}
\def\PYZam{\char`\&}
\def\PYZlt{\char`\<}
\def\PYZgt{\char`\>}
\def\PYZsh{\char`\#}
\def\PYZpc{\char`\%}
\def\PYZdl{\char`\$}
\def\PYZhy{\char`\-}
\def\PYZsq{\char`\'}
\def\PYZdq{\char`\"}
\def\PYZti{\char`\~}
% for compatibility with earlier versions
\def\PYZat{@}
\def\PYZlb{[}
\def\PYZrb{]}
\makeatother


    % For linebreaks inside Verbatim environment from package fancyvrb. 
    \makeatletter
        \newbox\Wrappedcontinuationbox 
        \newbox\Wrappedvisiblespacebox 
        \newcommand*\Wrappedvisiblespace {\textcolor{red}{\textvisiblespace}} 
        \newcommand*\Wrappedcontinuationsymbol {\textcolor{red}{\llap{\tiny$\m@th\hookrightarrow$}}} 
        \newcommand*\Wrappedcontinuationindent {3ex } 
        \newcommand*\Wrappedafterbreak {\kern\Wrappedcontinuationindent\copy\Wrappedcontinuationbox} 
        % Take advantage of the already applied Pygments mark-up to insert 
        % potential linebreaks for TeX processing. 
        %        {, <, #, %, $, ' and ": go to next line. 
        %        _, }, ^, &, >, - and ~: stay at end of broken line. 
        % Use of \textquotesingle for straight quote. 
        \newcommand*\Wrappedbreaksatspecials {% 
            \def\PYGZus{\discretionary{\char`\_}{\Wrappedafterbreak}{\char`\_}}% 
            \def\PYGZob{\discretionary{}{\Wrappedafterbreak\char`\{}{\char`\{}}% 
            \def\PYGZcb{\discretionary{\char`\}}{\Wrappedafterbreak}{\char`\}}}% 
            \def\PYGZca{\discretionary{\char`\^}{\Wrappedafterbreak}{\char`\^}}% 
            \def\PYGZam{\discretionary{\char`\&}{\Wrappedafterbreak}{\char`\&}}% 
            \def\PYGZlt{\discretionary{}{\Wrappedafterbreak\char`\<}{\char`\<}}% 
            \def\PYGZgt{\discretionary{\char`\>}{\Wrappedafterbreak}{\char`\>}}% 
            \def\PYGZsh{\discretionary{}{\Wrappedafterbreak\char`\#}{\char`\#}}% 
            \def\PYGZpc{\discretionary{}{\Wrappedafterbreak\char`\%}{\char`\%}}% 
            \def\PYGZdl{\discretionary{}{\Wrappedafterbreak\char`\$}{\char`\$}}% 
            \def\PYGZhy{\discretionary{\char`\-}{\Wrappedafterbreak}{\char`\-}}% 
            \def\PYGZsq{\discretionary{}{\Wrappedafterbreak\textquotesingle}{\textquotesingle}}% 
            \def\PYGZdq{\discretionary{}{\Wrappedafterbreak\char`\"}{\char`\"}}% 
            \def\PYGZti{\discretionary{\char`\~}{\Wrappedafterbreak}{\char`\~}}% 
        } 
        % Some characters . , ; ? ! / are not pygmentized. 
        % This macro makes them "active" and they will insert potential linebreaks 
        \newcommand*\Wrappedbreaksatpunct {% 
            \lccode`\~`\.\lowercase{\def~}{\discretionary{\hbox{\char`\.}}{\Wrappedafterbreak}{\hbox{\char`\.}}}% 
            \lccode`\~`\,\lowercase{\def~}{\discretionary{\hbox{\char`\,}}{\Wrappedafterbreak}{\hbox{\char`\,}}}% 
            \lccode`\~`\;\lowercase{\def~}{\discretionary{\hbox{\char`\;}}{\Wrappedafterbreak}{\hbox{\char`\;}}}% 
            \lccode`\~`\:\lowercase{\def~}{\discretionary{\hbox{\char`\:}}{\Wrappedafterbreak}{\hbox{\char`\:}}}% 
            \lccode`\~`\?\lowercase{\def~}{\discretionary{\hbox{\char`\?}}{\Wrappedafterbreak}{\hbox{\char`\?}}}% 
            \lccode`\~`\!\lowercase{\def~}{\discretionary{\hbox{\char`\!}}{\Wrappedafterbreak}{\hbox{\char`\!}}}% 
            \lccode`\~`\/\lowercase{\def~}{\discretionary{\hbox{\char`\/}}{\Wrappedafterbreak}{\hbox{\char`\/}}}% 
            \catcode`\.\active
            \catcode`\,\active 
            \catcode`\;\active
            \catcode`\:\active
            \catcode`\?\active
            \catcode`\!\active
            \catcode`\/\active 
            \lccode`\~`\~ 	
        }
    \makeatother

    \let\OriginalVerbatim=\Verbatim
    \makeatletter
    \renewcommand{\Verbatim}[1][1]{%
        %\parskip\z@skip
        \sbox\Wrappedcontinuationbox {\Wrappedcontinuationsymbol}%
        \sbox\Wrappedvisiblespacebox {\FV@SetupFont\Wrappedvisiblespace}%
        \def\FancyVerbFormatLine ##1{\hsize\linewidth
            \vtop{\raggedright\hyphenpenalty\z@\exhyphenpenalty\z@
                \doublehyphendemerits\z@\finalhyphendemerits\z@
                \strut ##1\strut}%
        }%
        % If the linebreak is at a space, the latter will be displayed as visible
        % space at end of first line, and a continuation symbol starts next line.
        % Stretch/shrink are however usually zero for typewriter font.
        \def\FV@Space {%
            \nobreak\hskip\z@ plus\fontdimen3\font minus\fontdimen4\font
            \discretionary{\copy\Wrappedvisiblespacebox}{\Wrappedafterbreak}
            {\kern\fontdimen2\font}%
        }%
        
        % Allow breaks at special characters using \PYG... macros.
        \Wrappedbreaksatspecials
        % Breaks at punctuation characters . , ; ? ! and / need catcode=\active 	
        \OriginalVerbatim[#1,codes*=\Wrappedbreaksatpunct]%
    }
    \makeatother

    % Exact colors from NB
    \definecolor{incolor}{HTML}{303F9F}
    \definecolor{outcolor}{HTML}{D84315}
    \definecolor{cellborder}{HTML}{CFCFCF}
    \definecolor{cellbackground}{HTML}{F7F7F7}
    
    % prompt
    \makeatletter
    \newcommand{\boxspacing}{\kern\kvtcb@left@rule\kern\kvtcb@boxsep}
    \makeatother
    \newcommand{\prompt}[4]{
        \ttfamily\llap{{\color{#2}[#3]:\hspace{3pt}#4}}\vspace{-\baselineskip}
    }
    

    
\setlength\headheight{30pt}
\setcounter{secnumdepth}{0} % Turns off numbering for sections

    % Prevent overflowing lines due to hard-to-break entities
    \sloppy 
    % Setup hyperref package
    \hypersetup{
      breaklinks=true,  % so long urls are correctly broken across lines
      colorlinks=true,
      urlcolor=urlcolor,
      linkcolor=linkcolor,
      citecolor=citecolor,
      }
    % Slightly bigger margins than the latex defaults
    \geometry{a4paper,tmargin=3cm,bmargin=2cm,lmargin=1cm,rmargin=1cm}\fancyhead[L]{Thème à définir}\fancyhead[L]{\adjustimage{height=1cm, valign=m}{/home/bouscadilla/Documents/Code/nbconvert/template/latex/pdf_solution/papier_eleve_ico_langage}\ttfamily\scshape Langage}\fancyhead[C]{\bfseries\MakeUppercase{2-2---bases-python-act}}\fancyhead[C]{\bfseries\MakeUppercase{2 --- Programmer en Python}}\fancyhead[R]{\monthyeardate\today}

    \fancyfoot[C]{\thepage}
    % #TODO ajouter les pages totales

    \pagestyle{fancy}
    


\begin{document}
    
    \title{2 --- Programmer en Python}
% \maketitle

    
    

    
    \hypertarget{exercices}{%
\subsection{3 --- Exercices}\label{exercices}}
\begin{eleve}
    \textbf{Observer} les résultats obtenus par l'expression
\texttt{5\ -\ 3\ -\ 2} et par l'expression \texttt{1\ /\ 2\ /\ 2}.

\textbf{En déduire} la manière dont sont interprétées les soustractions
et les divisions enchaînées.
        
        \end{eleve}\begin{eleve}
    \textbf{Réécrire} les expressions suivantes en explicitant toutes les
parenthèses :

\begin{enumerate}
\def\labelenumi{\arabic{enumi}.}
\tightlist
\item
  \texttt{1\ +\ 2\ *\ 3\ -\ 4}
\item
  \texttt{1+2\ /\ 4*3}
\item
  \texttt{1-a+a*a/2-a*a*a/6+a*a*a*a/24}
\end{enumerate}
        
        \end{eleve}\begin{reponse}
    \begin{enumerate}
\def\labelenumi{\arabic{enumi}.}
\tightlist
\item
  \texttt{(1\ +\ (2*3))\ -\ 4}
\item
  \texttt{1\ +\ ((2/4)*3)}
\item
  \texttt{(((1\ -\ a)\ +\ ((a*a)/2))\ -\ ((a*a*a)/6))\ +\ ((a*a*a*a)/24)}
\end{enumerate}

        \end{reponse}\begin{eleve}
    \textbf{Réécrire} les expressions suivantes en utilisant aussi peu de
parenthèses que possible sans changer le résultat.

\begin{enumerate}
\def\labelenumi{\arabic{enumi}.}
\tightlist
\item
  \texttt{1+(2*(3-4))}
\item
  \texttt{(1+2)+((5*3)+4)}
\item
  \texttt{(1-((2-3)+4))+(((5-6)+((7-8)/2)))}
\end{enumerate}
        
        \end{eleve}\begin{reponse}
    \begin{enumerate}
\def\labelenumi{\arabic{enumi}.}
\tightlist
\item
  \texttt{1\ +\ 2\ *\ (3\ -\ 4)}
\item
  \texttt{1\ +\ 2\ +\ 5*3\ +\ 4}
\item
  \texttt{1\ -\ (2\ -\ 3\ +\ 4)\ +\ 5\ -\ 6\ +\ (7-8)/2}
\end{enumerate}

        \end{reponse}\begin{eleve}
    \textbf{Déterminer} la valeur affichée par l'interprète Python après la
séquence d'instructions suivante :

\begin{Shaded}
\begin{Highlighting}[]
\NormalTok{a }\OperatorTok{=} \DecValTok{3}
\NormalTok{a }\OperatorTok{=} \DecValTok{4}
\NormalTok{a }\OperatorTok{=}\NormalTok{ a}\OperatorTok{+}\DecValTok{2}
\NormalTok{a}
\end{Highlighting}
\end{Shaded}
        
        \end{eleve}\begin{reponse}
    Après ces instruction, l'interprète affiche \texttt{6}. Voici les états
ligne par ligne :

\[
\overset{\text{\#1}}\longrightarrow
\overset{\texttt{a}}{\fbox{\texttt{3}}}
\overset{\text{\#2}}\longrightarrow
\overset{\texttt{a}}{\fbox{\texttt{4}}}
\overset{\text{\#3}}\longrightarrow
\overset{\texttt{a}}{\fbox{\texttt{6}}}
\]

        \end{reponse}\begin{eleve}
    \textbf{Déterminer} la valeur affichée par l'interprète Python après la
séquence d'instructions suivante :

\begin{Shaded}
\begin{Highlighting}[]
\NormalTok{a }\OperatorTok{=} \DecValTok{2}
\NormalTok{b }\OperatorTok{=}\NormalTok{ a}\OperatorTok{*}\NormalTok{a}
\NormalTok{b }\OperatorTok{=}\NormalTok{ a}\OperatorTok{*}\NormalTok{b}
\NormalTok{b }\OperatorTok{=}\NormalTok{ b}\OperatorTok{*}\NormalTok{b}
\NormalTok{b}
\end{Highlighting}
\end{Shaded}
        
        \end{eleve}\begin{reponse}
    L'interprète affiche la valeur de la variable \texttt{b}, soit
\texttt{64}. Voici les états successifs de l'interprète :

\[
\overset{\text{\#1}}\longrightarrow
\overset{\texttt{a}}{\fbox{\texttt{2}}}
\overset{\text{\#2}}\longrightarrow
\overset{\texttt{a}}{\fbox{\texttt{2}}}
\overset{\texttt{b}}{\fbox{\texttt{4}}}
\overset{\text{\#3}}\longrightarrow
\overset{\texttt{a}}{\fbox{\texttt{2}}}
\overset{\texttt{b}}{\fbox{\texttt{8}}}
\overset{\text{\#4}}\longrightarrow
\overset{\texttt{a}}{\fbox{\texttt{2}}}
\overset{\texttt{b}}{\fbox{\texttt{64}}}
\]

        \end{reponse}\begin{eleve}
    (Capytale : 3f7e-77330) Dans un notebook, \textbf{initialiser} une
variable \texttt{a} avec la valeur 2, puis \textbf{répéter} dix fois
l'instruction \texttt{a\ =\ a\ *\ a}.

\textbf{Observer} le résultat. Quelle puissance de 2 a-t-on ainsi
calculé ?

\textbf{Recommencer} en affectant cette fois-ci la valeur \texttt{2.0} à
la variable \texttt{a}. \textbf{Observer} puis \textbf{interpréter} le
résultat.
        
        \end{eleve}\begin{reponse}
    En procédant ainsi l'interprète a calculé \(2^{1024}\).

En affectant \texttt{2.0} à la variable \texttt{a}, on obtient
\texttt{inf} qui signifie \emph{infini} et indique que le nombre
flottant n'est pas représentable car il est trop grand.

        \end{reponse}\begin{eleve}
    \textbf{Indiquer} ce qu'affichent les instructions suivantes
\texttt{print("1+")} et \texttt{print(1+)}.
        
        \end{eleve}\begin{reponse}
    Dans le premier cas la chaîne de caractère \texttt{1+} est affichée.
Dans le second cas, l'expression (addition entre un nombre entier et
rien du tout) comporte une erreur de syntaxe. L'exception
\texttt{SyntaxError} est levée.

        \end{reponse}\begin{eleve}
    \textbf{Indiquer} ce qu'il se passe quand on exécute le code suivant :

\begin{Shaded}
\begin{Highlighting}[]
\NormalTok{a }\OperatorTok{=} \BuiltInTok{input}\NormalTok{(}\StringTok{"saisir un nombre : "}\NormalTok{)}
\BuiltInTok{print}\NormalTok{(}\StringTok{"le nombre suivant est "}\NormalTok{, a}\OperatorTok{+}\DecValTok{1}\NormalTok{)}
\end{Highlighting}
\end{Shaded}

\textbf{Rectifier} si nécessaire.
        
        \end{eleve}\begin{reponse}
    L'expression \texttt{a+1} est incorrecte puisqu'elle demande d'effectuer
une addition entre \texttt{a} qui est une chaîne de caractère et le
nombre entier \texttt{1}. Cette opération n'est pas définie en Python.

Pour corriger ce code, il faut par exemple ajouter dans une ligne
intermédiaire le code \texttt{a\ =\ int(a)}.

        \end{reponse}\begin{eleve}
    \textbf{Indiquer} ce que fait la séquence d'instruction suivante en
supposant qu'à l'origine les variables \texttt{a} et \texttt{b}
contiennent un nombre entier.

\begin{Shaded}
\begin{Highlighting}[]
\NormalTok{tmp }\OperatorTok{=}\NormalTok{ a}
\NormalTok{a }\OperatorTok{=}\NormalTok{ b}
\NormalTok{b }\OperatorTok{=}\NormalTok{ tmp}
\end{Highlighting}
\end{Shaded}
        
        \end{eleve}\begin{reponse}
    Détaillons les états successifs de l'interprète en supposant que la
variable \texttt{a} contienne la valeur \(n_1\) et que la variable
\texttt{b} contienne la valeur \(n_2\).

\[
\overset{\texttt{a}}{\fbox{$n_1$}}
\overset{\texttt{b}}{\fbox{$n_2$}}
\overset{\text{\#1}}\longrightarrow
\overset{\texttt{a}}{\fbox{$n_1$}}
\overset{\texttt{b}}{\fbox{$n_2$}}
\overset{\texttt{tmp}}{\fbox{$n_1$}}
\overset{\text{\#2}}\longrightarrow
\overset{\texttt{a}}{\fbox{$n_2$}}
\overset{\texttt{b}}{\fbox{$n_2$}}
\overset{\texttt{tmp}}{\fbox{$n_1$}}
\overset{\text{\#3}}\longrightarrow
\overset{\texttt{a}}{\fbox{$n_2$}}
\overset{\texttt{b}}{\fbox{$n_1$}}
\overset{\texttt{tmp}}{\fbox{$n_1$}}
\]

À la fin de l'instruction, les valeurs enregistrées dans les variables
\texttt{a} et \texttt{b} ont été \textbf{permutées}.

        \end{reponse}\begin{eleve}
    (Capytale : f6db-77338) On met deux entiers dans deux variables
\texttt{a} et \texttt{b}, par exemple 55 et 89. On remplace le contenu
de \texttt{a} par la somme de celui de \texttt{a} et de \texttt{b}. Puis
on remplace le contenu de \texttt{b} par le contenu de \texttt{a} moins
le contenu de \texttt{b}? Enfin on remplace le contenu de \texttt{a} par
son contenu moins celui de \texttt{b}.

\textbf{Que contiennent} \texttt{a} et \texttt{b} à la fin de ces
opérations ?

\textbf{Programme} cet algorithme en Python.
        
        \end{eleve}\begin{reponse}
    Notons \texttt{va} et \texttt{vb} les valeurs initiales des variables
\texttt{a} et \texttt{b}.

\[
\overset{\texttt{a}}{\fbox{\texttt{va}}}
\overset{\texttt{b}}{\fbox{\texttt{vb}}}
\longrightarrow
\overset{\texttt{a}}{\fbox{\texttt{va+vb}}}
\overset{\texttt{b}}{\fbox{\texttt{vb}}}
\longrightarrow
\overset{\texttt{a}}{\fbox{\texttt{va+vb}}}
\overset{\texttt{b}}{\fbox{\texttt{va}}}
\longrightarrow
\overset{\texttt{a}}{\fbox{\texttt{vb}}}
\overset{\texttt{b}}{\fbox{\texttt{va}}}
\]

À la fin de cette séquence d'instructions, les valeurs de \texttt{a} et
de \texttt{b} ont été permutées.

        \end{reponse}
        {\scriptsize
    \begin{tcolorbox}[breakable, size=fbox, boxrule=1pt, pad at break*=1mm,colback=cellbackground, colframe=cellborder]
\prompt{In}{incolor}{ }{\boxspacing}
\begin{Verbatim}[commandchars=\\\{\}]
\PY{n}{a} \PY{o}{=} \PY{l+m+mi}{55}
\PY{n}{b} \PY{o}{=} \PY{l+m+mi}{89}

\PY{n}{a} \PY{o}{=} \PY{n}{a} \PY{o}{+} \PY{n}{b}
\PY{n}{b} \PY{o}{=} \PY{n}{a} \PY{o}{\PYZhy{}} \PY{n}{b}
\PY{n}{a} \PY{o}{=} \PY{n}{a} \PY{o}{\PYZhy{}} \PY{n}{b}
\PY{n+nb}{print}\PY{p}{(}\PY{l+s+s2}{\PYZdq{}}\PY{l+s+s2}{a vaut}\PY{l+s+s2}{\PYZdq{}}\PY{p}{,} \PY{n}{a}\PY{p}{,} \PY{l+s+s2}{\PYZdq{}}\PY{l+s+s2}{et b vaut}\PY{l+s+s2}{\PYZdq{}}\PY{p}{,} \PY{n}{b}\PY{p}{)}
\end{Verbatim}
\end{tcolorbox}
    }
\begin{eleve}
    (Capytale : d677-77366) \textbf{Écrire} un programme qui demande à
l'utilisateur les longueurs (entières) des deux côtés d'un rectangle et
affiche son aire.
        
        \end{eleve}
        {\scriptsize
    \begin{tcolorbox}[breakable, size=fbox, boxrule=1pt, pad at break*=1mm,colback=cellbackground, colframe=cellborder]
\prompt{In}{incolor}{ }{\boxspacing}
\begin{Verbatim}[commandchars=\\\{\}]
\PY{n}{texte\PYZus{}l1} \PY{o}{=} \PY{n+nb}{input}\PY{p}{(}\PY{l+s+s2}{\PYZdq{}}\PY{l+s+s2}{Saisir la longueur du premier côté :}\PY{l+s+s2}{\PYZdq{}}\PY{p}{)}
\PY{n}{texte\PYZus{}l2} \PY{o}{=} \PY{n+nb}{input}\PY{p}{(}\PY{l+s+s2}{\PYZdq{}}\PY{l+s+s2}{Saisir la longueur du second  côté :}\PY{l+s+s2}{\PYZdq{}}\PY{p}{)}
\PY{n}{l1} \PY{o}{=} \PY{n+nb}{int}\PY{p}{(}\PY{n}{texte\PYZus{}l1}\PY{p}{)}
\PY{n}{l2} \PY{o}{=} \PY{n+nb}{int}\PY{p}{(}\PY{n}{texte\PYZus{}l2}\PY{p}{)}
\PY{n}{aire} \PY{o}{=} \PY{n}{l1} \PY{o}{*} \PY{n}{l2}
\PY{n+nb}{print}\PY{p}{(}\PY{l+s+s2}{\PYZdq{}}\PY{l+s+s2}{L}\PY{l+s+s2}{\PYZsq{}}\PY{l+s+s2}{aire du rectangle vaut}\PY{l+s+s2}{\PYZdq{}}\PY{p}{,} \PY{n}{aire}\PY{p}{)}
\end{Verbatim}
\end{tcolorbox}
    }
\begin{eleve}
    (Capytale : 5b27-77369) \textbf{Écrire} un programme qui demande
d'entrer une base (entre 2 et 36) et un nombre dans cette base et qui
affiche ce nombre en base 10.

La notation \texttt{int(chaine,\ base)} permet de convertir une
\texttt{chaîne} représentant un entier dans une \texttt{base} donnée en
un entier Python.
        
        \end{eleve}
        {\scriptsize
    \begin{tcolorbox}[breakable, size=fbox, boxrule=1pt, pad at break*=1mm,colback=cellbackground, colframe=cellborder]
\prompt{In}{incolor}{ }{\boxspacing}
\begin{Verbatim}[commandchars=\\\{\}]
\PY{n}{txt\PYZus{}base} \PY{o}{=} \PY{n+nb}{input}\PY{p}{(}\PY{l+s+s2}{\PYZdq{}}\PY{l+s+s2}{Saisir la base :}\PY{l+s+s2}{\PYZdq{}}\PY{p}{)}
\PY{n}{txt\PYZus{}nb}   \PY{o}{=} \PY{n+nb}{input}\PY{p}{(}\PY{l+s+s2}{\PYZdq{}}\PY{l+s+s2}{Saisir le nombre :}\PY{l+s+s2}{\PYZdq{}}\PY{p}{)}
\PY{n}{base} \PY{o}{=} \PY{n+nb}{int}\PY{p}{(}\PY{n}{txt\PYZus{}base}\PY{p}{)}
\PY{n}{nb} \PY{o}{=} \PY{n+nb}{int}\PY{p}{(}\PY{n}{txt\PYZus{}nb}\PY{p}{,} \PY{n}{base}\PY{p}{)}
\PY{n+nb}{print}\PY{p}{(}\PY{l+s+s2}{\PYZdq{}}\PY{l+s+s2}{le nombre}\PY{l+s+s2}{\PYZdq{}}\PY{p}{,}\PY{n}{txt\PYZus{}nb}\PY{p}{,}\PY{l+s+s2}{\PYZdq{}}\PY{l+s+s2}{écrit en base}\PY{l+s+s2}{\PYZdq{}}\PY{p}{,}
     \PY{n}{txt\PYZus{}base}\PY{p}{,}\PY{l+s+s2}{\PYZdq{}}\PY{l+s+s2}{s}\PY{l+s+s2}{\PYZsq{}}\PY{l+s+s2}{écrit en base 10 :}\PY{l+s+s2}{\PYZdq{}}\PY{p}{,} \PY{n}{nb}\PY{p}{)}
\end{Verbatim}
\end{tcolorbox}
    }
\begin{eleve}
    (Capytale : c513-77371) \textbf{Écrire} un programme qui demande à
l'utilisateur d'entrer un nombre de secondes et qui l'affiche sous la
forme d'heures/minutes/secondes.
        
        \end{eleve}
        {\scriptsize
    \begin{tcolorbox}[breakable, size=fbox, boxrule=1pt, pad at break*=1mm,colback=cellbackground, colframe=cellborder]
\prompt{In}{incolor}{ }{\boxspacing}
\begin{Verbatim}[commandchars=\\\{\}]
\PY{n}{txt\PYZus{}seconde} \PY{o}{=} \PY{n+nb}{input}\PY{p}{(}\PY{l+s+s2}{\PYZdq{}}\PY{l+s+s2}{Saisir un nombre de secondes :}\PY{l+s+s2}{\PYZdq{}}\PY{p}{)}
\PY{n}{seconde} \PY{o}{=} \PY{n+nb}{int}\PY{p}{(}\PY{n}{txt\PYZus{}seconde}\PY{p}{)}
\PY{n}{minute} \PY{o}{=} \PY{n}{seconde} \PY{o}{/}\PY{o}{/} \PY{l+m+mi}{60}
\PY{n}{seconde} \PY{o}{=} \PY{n}{seconde} \PY{o}{\PYZpc{}} \PY{l+m+mi}{60}

\PY{n}{heure} \PY{o}{=} \PY{n}{minute} \PY{o}{/}\PY{o}{/} \PY{l+m+mi}{60}
\PY{n}{minute} \PY{o}{=} \PY{n}{minute} \PY{o}{\PYZpc{}} \PY{l+m+mi}{60}

\PY{n+nb}{print}\PY{p}{(}\PY{n}{heure}\PY{p}{,}\PY{l+s+s2}{\PYZdq{}}\PY{l+s+s2}{h}\PY{l+s+s2}{\PYZdq{}}\PY{p}{,} \PY{n}{minute}\PY{p}{,}\PY{l+s+s2}{\PYZdq{}}\PY{l+s+s2}{min}\PY{l+s+s2}{\PYZdq{}}\PY{p}{,} \PY{n}{seconde}\PY{p}{,}\PY{l+s+s2}{\PYZdq{}}\PY{l+s+s2}{s}\PY{l+s+s2}{\PYZdq{}}\PY{p}{)}
\end{Verbatim}
\end{tcolorbox}
    }
\begin{eleve}
    (Capytale : 70d6-77374) On souhaite écrire un programme qui demande à
l'utilisateur un nombre d'œufs et affiche le nombre de boîtes de 6 œufs
nécessaires à leur transport. On considère ce programme qui utilise la
division euclidienne.

\begin{Shaded}
\begin{Highlighting}[]
\NormalTok{n }\OperatorTok{=} \BuiltInTok{int}\NormalTok{(}\BuiltInTok{input}\NormalTok{(}\StringTok{"combien d\textquotesingle{}œufs : "}\NormalTok{))}
\BuiltInTok{print}\NormalTok{(n}\OperatorTok{//}\DecValTok{6}\NormalTok{)}
\end{Highlighting}
\end{Shaded}

\textbf{Tester} ce programme sur différentes entrées.

\begin{enumerate}
\def\labelenumi{\arabic{enumi}.}
\tightlist
\item
  Sur quelles valeurs de \texttt{n} ce programme est-il correct ?
\item
  Pourquoi n'est-il pas correct de remplacer \texttt{n\ //\ 6} par
  \texttt{n\ //\ 6\ +\ 1} ?
\item
  Proposer une solution correcte.
\end{enumerate}
        
        \end{eleve}\begin{reponse}
    \begin{enumerate}
\def\labelenumi{\arabic{enumi}.}
\tightlist
\item
  Ce programme n'est correct que pour les nombres \texttt{n} multiples
  de 6.
\item
  Avec la modification proposée, le programme est correct pour les
  valeurs qui ne sont pas multiples de 6, mais est devenu incorrect pour
  les valeurs multiples de 6.
\item
  Un programme correct est :
\end{enumerate}

\begin{Shaded}
\begin{Highlighting}[]
\NormalTok{n }\OperatorTok{=} \BuiltInTok{int}\NormalTok{(}\BuiltInTok{input}\NormalTok{(}\StringTok{"combien d\textquotesingle{}œufs : "}\NormalTok{))}
\BuiltInTok{print}\NormalTok{((n}\OperatorTok{+}\DecValTok{5}\NormalTok{) }\OperatorTok{//} \DecValTok{6}\NormalTok{)}
\end{Highlighting}
\end{Shaded}

        \end{reponse}

    % Add a bibliography block to the postdoc
    
    
    
\end{document}
